\documentclass[12pt]{book}

% not needed with polyglossia
\usepackage[utf8]{inputenc}
\usepackage[T1]{fontenc}

%\usepackage{polyglossia}
%\setdefaultlanguage{icelandic}


\usepackage{graphics,amsmath,amsfonts,amsbsy,amssymb,amsthm}
\usepackage{fancyvrb}
\usepackage[a4paper]{geometry}
\usepackage{graphicx}
\usepackage{hyperref}
\usepackage{datatool}
\usepackage{float}
\usepackage{mdframed}
\usepackage{listingsutf8}
\usepackage{enumerate}
\usepackage{comment}
\usepackage{epstopdf}
\usepackage{caption}
\usepackage{subcaption}
\usepackage{tikz}
\usepackage{enumitem}
\usepackage{mathtools}
\usepackage{tabu}

\usepackage{accents}

\setlength{\parskip}{8pt plus 1pt minus 1pt}
%Verdur ad vera her, sumir pakkar dependa a thetta.
\usepackage[icelandic]{babel}

%viljum ekki númeraða kafla á dæmum

\newcommand{\nonums}{\setcounter{secnumdepth}{-1}}

%flýtiskipanir
\newcommand{\e}{\textbf}
\newcommand{\R}{\mathbb{R}}

\newcommand{\X}{\mathbb{X}}
\newcommand{\Y}{\mathbb{Y}}


%\newcommand{\R}{\Real}
%\newcommand{\C}{\Complex}
%\newcommand{\Z}{\Integer}
%\newcommand{\N}{\Natural}
%\newcommand{\Q}{\Rational}

\newcommand{\K}{\mathbb{K}}
\newcommand{\C}{\mathbb{C}}
\newcommand{\Con}{\mathcal{C}}
\newcommand{\Z}{\mathbb{Z}}
\newcommand{\N}{\mathbb{N}}
\newcommand{\Q}{\mathbb{Q}}
\newcommand{\f}{\frac}
\newcommand{\1}{\frac{1}}
\newcommand{\eps}{\f{\epsilon}}
\newcommand{\Lra}{\Leftrightarrow}
\newcommand{\Th}{\text{ þegar }}
\newcommand{\Ef}{\text{ ef }}
\newcommand{\Og}{\text{ og }}


\newcommand{\inner}[1]{\accentset{\circ}{#1}}
\newcommand{\eR}{\widetilde{\R}}

\newcommand{\com}[1]{\set{\text{#1}}}
\newcommand{\Com}[1]{\set{\text{Athsmd: \text{#1}}}}

\newcommand{\ub}[2]{\underbrace{#1}_{\text{#2}}}
\newcommand{\ubt}[2]{$\ub{\text{#1}}{#2}$}


\newenvironment{inum}{\begin{enumerate}[label=(\roman*).]}{\end{enumerate}}
\newenvironment{anum}{\begin{enumerate}[label=(\alph*).]}{\end{enumerate}}


\newcommand{\bcondef}{\left\{ \begin{array}{l l}}
\newcommand{\econdef}{\end{array} \right.}
\DeclarePairedDelimiter{\condef}{\bcondef}{\econdef}

\DeclarePairedDelimiter{\ceil}{\lceil}{\rceil}
\DeclarePairedDelimiter{\floor}{\lfloor}{\rfloor}
\DeclarePairedDelimiter{\set}{\{}{\}}
\DeclarePairedDelimiter{\braket}{\langle}{\rangle}


\newenvironment{lausn}{\begin{proof}[Lausn]}{\end{proof}}

\newcommand{\sep}{\;|\;}

\newcommand{\fig}[2]{
\begin{figure}[H]
  \centering
  \includegraphics{#1}
  \caption{#2}
  \label{fig:#1}
\end{figure}
}


\newtheorem*{setn}{Setning}
\newtheorem*{hsetn}{Hjálparsetning}
\lstset{  literate={á}{{\'a}}1
                  {ó}{{\'o}}1
                  {ú}{{\'u}}1
                  {ð}{{\dh}}1
                  {í}{{\'i}}1
                  {é}{{\'e}}1
                  {ö}{{\"o}}1
                  {þ}{{\th}}1
                  {æ}{{\ae}}1
                  {ý}{{\'y}}1
                  {Á}{{\'A}}1
                  {Ó}{{\'O}}1
                  {Ú}{{\'U}}1
                  {Ð}{{\DH}}1
                  {Í}{{\'I}}1
                  {É}{{\'E}}1
                  {Ö}{{\"O}}1
                  {Þ}{{\TH}}1
                  {Æ}{{\AE}}1
                  {Ý}{{\'Y}}1}


\theoremstyle{definition}
\newtheorem*{skgr}{Skilgreining}
\newtheorem*{daemi}{Dæmi}
\newtheorem*{frumsenda}{Frumsenda}

\theoremstyle{remark}
\newtheorem*{ath}{Athugasemd}


\newcommand{\cT}{\mathcal{T}}
\newcommand{\cL}{\mathcal{L}}
\newcommand{\cS}{\mathcal{S}}
\newcommand{\cP}{\mathcal{P}}
\newcommand{\cA}{\mathcal{A}}
\newcommand{\PA}{\mathcal{PA}}
\newcommand{\cN}{\mathcal{N}}
\newcommand{\cR}{\mathcal{R}}
\newcommand{\mb}[1]{\mathbf{#1}}
\newcommand{\mc}[1]{\mathcal{#1}}
\newcommand{\bA}{\mathbf{A}}
\newcommand{\ba}{\mathbf{a}}
\newcommand{\bB}{\mathbf{B}}
\newcommand{\bb}{\mathbf{b}}
\newcommand{\bC}{\mathbf{C}}
\newcommand{\bD}{\mathbf{D}}
\newcommand{\bE}{\mathbf{E}}
\newcommand{\bG}{\mathbf{G}}
\newcommand{\bR}{\mathbf{R}}
\newcommand{\bS}{\mathbf{S}}
\newcommand{\bT}{\mathbf{T}}
\newcommand{\bL}{\mathbf{L}}
\newcommand{\bN}{\mathbf{N}}
\newcommand{\bP}{\mathbf{P}}
\newcommand{\bc}{\mathbf{c}}
\newcommand{\bX}{\mathbf{X}}
\newcommand{\bx}{\mathbf{x}}
\newcommand{\bk}{\mathbf{k}}
\newcommand{\by}{\mathbf{y}}
\newcommand{\bz}{\mathbf{z}}
\newcommand{\bw}{\mathbf{w}}
\newcommand{\bu}{\mathbf{u}}
\newcommand{\bt}{\mathbf{t}}
\newcommand{\bv}{\mathbf{v}}
\newcommand{\xxn}{x_1, \dotsc, x_n}
\newcommand{\bxxn}{\bx_1, \dotsc, \bx_n}
\newcommand{\byyk}{\by_1, \dotsc, \by_k}
\newcommand{\aan}{a_1, \dotsc, a_n}
\newcommand{\baan}{\ba_1, \dotsc, \ba_n}
\newcommand{\bkaan}{\bk_{a_1}, \dotsc,\bk_{a_n}}
\newcommand{\dda}{\dot{-}}
\newcommand{\Thm}{{\mc{T}hm}}
\newcommand{\vT}{\vdash_{\cT}}
\newcommand{\vR}{\vdash_{\cR}}
\newcommand{\vN}{\vdash_{\cN}}
\newcommand{\vP}{\vdash_{\PA}}
\newcommand{\bTx}{\bT_{\bx}}
\newcommand{\thT}{Thm_{\cT}}
\newcommand{\emphs}[1]{\textbf{\emph{#1}}}
\newcommand{\demphs}[1]{\textbf{#1}}

\DeclarePairedDelimiter{\god}{\ulcorner}{\urcorner}
\DeclarePairedDelimiter{\pg}{\llcorner}{\lrcorner}

\title{Rökfræði}
\author{Matthías Páll Gissurarson}

\begin{document}
\maketitle


\chapter{Yrðingarökfræði}

Sjá hjá Reyni
\chapter{Umsagnarökfræði}

Sjá hjá Reyni

\chapter{Líkön}

\section{Mynztur og líkön}

\begin{proof}[Sönnun á Fullkomleikasetningu Gödels]
  G.r.f. að $\mathbf{A}$ sé ekki setning í $\mathcal{T}$ og látum $\mathbf{A}'$
  vera lokun $\mathbf{A}$. Þá er $\mathbf{A}'$ ekki heldur seting í
  $\mathcal{T}$. Þá er $\mathcal{T}[\lnot \mathbf{A}']$ samkvæm skv. HS og hefur
  þá líkan M skv. meginsetningu.

  En þá er M líkan fyrir $\mathcal{T}$ þ.a. $\lnot \mathbf{A}'$ sé sönn í M
  og þá er $\mathbf{A}'$ ósönn í M.
  
\end{proof}

\vdots

\begin{skgr}[Viðbót]
 Útvíkkun $\cT'$ kenningar $\cT$ er \emphs{einföld} ef kenningarnar hafa
sama mál. 
\end{skgr}


\begin{setn}[Hjálparsetn]
 Ef $\cT'$ er útvíkkun kenningar $\cT$ og $M'$ er líkan fyrir $\cT'$,
þá fæst líkan $M$ fyrir $\cT$ með því að sleppa þeim föllum og venzlum
sem ekki eru nefnd í $\cT$. Til að búa til líkan fyrir $\cT$, þá nægir að
búa til líkan fyrir útvíkkun á $\cT$. 
\end{setn}

\begin{skgr}
  \emphs{Henkin-kenning} er kenning $\cT$ þ.a. fyrir sérhverja lokaða yrðingu af
gerðinnni \( \exists \mathbf{x}\mathbf{A}\) á málinu $\cL(\cT)$, þá
  er til fasti \(\mathbf{c}\) í $\mathcal{T}$ þ.a. $\vdash  \exists \mathbf{x}\mathbf{A} \rightarrow  \mathbf{A}_{\mathbf{x}}[\mathbf{c}]$
\end{skgr}


Látum $\cL$ vera kenningu fytrstu stéttar á máli $\cL_0: \cL(\cT) = \cL$.
Útvíkkum $\cL_0$ í mál $\cL$ þannig: Fyrir sérhverja lokaða yrðingu
$\exists \bx \bA$ á $\cL$ bætum við við nýjum fasta '$\bc_{\exists \bx \bA}$'
við málið $\cL$ með því að skrifa bókstafinni '$c$' og setja '${\exists \bx \bA}$'
í lágvísi. Ef $\cL_n$ hefur verið skilgrein, $n \geq 1$, þá fáum við mál $\cL_{n+1}$
með því að bæta við fasta $c_{\exists \bx \bA}$ fyrir sérhverja lokaða yrðingu
$\exists \bx \bA$ í $\cL_n$ sem er ekki í $\cL_{n+1}$. Látum $\cL^* = \bigcup_{n \in \N} \cL_n$

vera málið með öllum föstunum úr $\cL_n$. Búum til útvíkkun $\cT^*$ á $\cT$ með
mál $\cL^*$ og nýjum frumsendum 
\[ \exists \bx \bA \to \bA_{\bx}[c_{\exists \bx \bA}]\]

Við $\cT$ fyrir allar lokaðar yrðingar $\exists \bx \bA$ á $\cL^*$. Köllum
þessar nýjar frumsendur sérstakar.

\begin{ath}
  Tungumálið $\cL^*$ hefur sömu fjöldatölu og $\cL$. Látum $\cL$ vera mál
fyrstu stéttar með stafróf $\cS$. Stafrófið er endanleg, því að við höfum teljanlega óendanlega
margar breytur, svo að $\#\cS \geq \aleph_0 := \# \N$. Athugum: Ef $A$ er 
óendanlegt mengi, þá er fjöldatala mengis allra endnlegra runa af stökum A jöfn $\# A$.

\[ \# A \leq \# \bigcup_{n \in \N} A^n \leq \sum_{n \in \N} ( \# A)^n = \sum_{n \in \N} \#A = \aleph_0 \cdot \# A = \# A.\]

því fyrir fjöldatölur $\alpha, \beta$ þ.a. önnur þeirra er óendanleg er $\alpha + \beta = \alpha \cdot \beta = \max \set{\alpha,\beta}$
\end{ath}

Fjöldi yrðinga í málinu $\cL$ er $\# \cS$; getum kallað $\# \cS$ fjöldatölu $\cL$.
Fjöldi fasta sem við bætum við $\cL_0$ til að fá $\cL$ er ekki stærri en fjöldi
yrðinga í $\cL_0$, svo að $\# \cS (\cL_1) = \# \cS (\cL_0)$ og
$\# \cL^* \leq \sum_{n \in \N} \# \cL_n \leq \aleph_0 \cdot \# \cL_0 = \# \cL_0$.

\begin{setn}
 Kenning $\cT^*$ er Henkin-kenning og íhaldsöm útvíkkun á $\cT$. 
\end{setn}

\begin{proof}
  Ljóst er að $\cT^*$ er Henkin-kenning: Látum $\cT'$ vera kenninguna með mál
$\cL^*$ en sömu frumsendur og $\cT$. Sk.v fastasetningu er $\cT'$ íhaldsöm útvíkkun
á $\cT$. Það nægir að sýna að yrðing á málinu $\cL$ sem er setning í $\cT^*$ er setning
í $\cT'$. Látumm $\bA$ vera slíka yrðingu. Skv. heimadæmi 28 eru til ólíkar sérstakar frumsendur
$\bB_1, \dots, \bB_m$ þ.a.
\[ \vdash_{\cT'} \bB_1 \wedge \dotsb \wedge \bB_m \to \bA, \text{ þ.e. } \vT \bB_1 \to \bB_2 \to \dotsb \to \bB_m \to \bA \]
Látum $\bc_i$ vera fastann í $\bB_j$, $n_j$ vera lægstu tölu þ.e. $\bc_j$ sé í $\cL_{n_j}$.
Getum raðað $\bB_1, \dots, \bB_n$ þ.a. $n_1 \geq n_2 \geq \dotsb \geq n_m$. Þá kemur $\bc_j$
ekki fyrior í $\bB_2, \dotsc, \bB_m$ og alls ekki í $\bA$. Skrifum $\bB_1$ sem $\exists \bx \bC \to \bC_{\bx}[\bc_j]$
og látum $\by$ vera breytu sem kemur ekki fyrir í $\bA$, $\bB_1, \dotsc, \bB_m$.

Skv. fastasetningu er $\vdash_{\cT'} (\exists \bx \bC \to \bC_{\bx}[\by]) \to \bB_2 \to \dotsb \to \bB_m \to \bA$
og skv. ($\exists \to$)-reglu er $\vdash_{\cT'} \exists \by ( \exists \bx \bC \to \bC_{\bx}[\by]) \to \bB_2 \to \dotsb \to \bB_m \to \bA$
en skv. tilbrigða setningu $\vdash_{\cT'} \exists \bx \bC \to \exists \by \bC_{\bx}[\by]$

og þá er $\vdash_{\cT'} \exists y ( \exists \bx \bC \to \bC_{\bx}[\by])$ skv. ``forskeytaaðgerð''.
En þá gefur \textbf{MP} að $\vdash_{\cT'} \bB_2 \to \dotsb \to \bB_m \to \bA$
og þrepun gefur $\vdash_{\cT'} \bA$.
\end{proof}

\begin{setn}[Fylgisetn]

Ef $\cT$ er samkvæm, þá er $\cT^*$ samkvæm.
  
\end{setn}

\begin{skgr}
  Kenning $\cT$ kallast fullkomin ef hún er samkvæm og fyrir sérhverja \emph{lokaða} yrðingu
$\bA$ gildir $\vT \bA$ eða $\vT \lnot \bA$.
\end{skgr}

\begin{setn}[Linenbaums]
 Sérhver samkvæm kenning hefur fullkomna einfalda útvíkkun. 
\end{setn}

\begin{proof}
  Lítum á kenninguna $\cT$ sem mengi setninga sinna. Látum $\cT$ vera samkvæma kenningu
  og $\mathbb{T}$ vera mengi allra samkvæmra kenninga $\cT'$ á málinu $\cL(\cT)$ þ.a.
  $\cT \subset \cT'$. Röðum $\mathbb{T}$ með hlutmengja venzlum. Látum $\mathbb{A}$
vera línulega raðað hlutmengi í $\mathbb{T}$. Þá er
\[ \widetilde{\cT} := \bigcup_{\cT' \in \mathbb{A}} \cT' \in \mathbb{T} \]
\end{proof}

Ef $\widetilde{\cT}$ er ekki samkvæm, þá er til lokuð yrðing $\bA$ þ.a.
$\bA$ og $\lnot \bA$ séu setningar í $\widetilde{\cT}$. Þar sem $\mathbb{A}$
er línulega rðað, þá er til $\cT'$ í $\mathbb{A}$ þ.a.
$\bA$ og $\lnot \bA$ séu stök í $\cT'$, sem er fráleitt. Svk. HS Zorns hefur $\mathbb{T}$
hástak $\cT^*$. Þá er $\cT^*$ fullkomin. Látum $\bA$ vera lokaða yrðingu á málinu $\cL$.
Þ.a. $\bA$ sé ekki setning í $\cT^*$. Þá er$\cT^*[\lnot \bA]$ skamkvæm,
$\cT^* \subset \cT^*[\lnot \bA]$ svo að $\cT^* = \cT^*[\lnot \bA]$, þ.e.
$\lnot \bA$ er í $\cT^*$.

\begin{setn}
 Fullkomin Hengin-kenning $\cT$ hefur líkan $M$ þ.a. að  fjöldatala
$|M|$ sé $\leq \lambda$, $\lambda = \# \cL(\cT)$
\end{setn}

Fáum styrkta meginsetningu:

\begin{setn}[Meginsetning]
 Látum $\cT$ vera samkvæma kenningu á máli $\cL$ með stafróf
$\cS$. Þá hefur $\cT$ llíkan $M$ þ.a. $\# |M| = \# \cS$. Ef
$\cT$ er samsemdarkenning, þá hefur hún samsemdarlíkan $M$ þ.a.
$\# |M| \leq \# |\cS|$.
\end{setn}




\begin{setn}
  Látum $\mathcal{T}$ vera fullkomna Henkin-kenningu
  með stafróf $\mathcal{S}$. Þá hefur $\mathcal{T}$ líkan M
  þ.a. $|M| = \#\mathcal{S}$;
 % ef hún er samsemdarkenning þá hefur hún samsemdalíkan $M$ þ.a. $|M| \leq \#\mathcal{S}$.

  \begin{proof}
    Aðeins er eftir að sanna fyrri fullyrðinguna.
    Látum $|M|$ vera mengi allra ``lokaðra heita'' (þ.e. heiti án breyta) á málinu
    $\mathcal{L}$. Fyrir sérhvert n-stætt fallatákn $\mathbf{f}$
    látum við $\mathbf{f}_M$ vera fallið $|M|^{\alpha} \rightarrow |M|$
    þ.a.

    \[\mathbf{f}_M ( \mathbf{a}_1, \dotsc,\mathbf{a}_n ) := \mathbf{f} \mathbf{a}, \dotsc, \mathbf{a}_n ,\]
    
    og fyrir n-stætt umsagnartákn $\mathbf{p}$ látum við $\mathbf{p}_M$ veera mengi allra
    n-unda $(\mathbf{a}_1, \dotsc, \\mathbf{a}_n)$ þ.a. 
    \(\mathbf{a}_1, \dotsc, \mathbf{a}_n \)
    þ.a. $\vdash_{\mathcal{T}} \mathbf{p} \mathbf{a}_1, \dotsc, \mathbf{a}_n$.
    Okkur nægir að sýna að fyrir lokaða yrðingu $\mathbf{A}$ gildir
    \[ \models_M \mathbf{A} \text{ þþaa } \vdash_{\mathcal{T}} \mathbf{A}; \]
    því að fyrir frumsendur $\mathbf{A}$ í $\mathcal{T}$. 
    Gildir $\vdash_{\mathcal{T}} \mathbf{A}'$; þar sem $\mathbf{A}'$ er lokun
    $\mathbf{A}$; en af því leiðir $\models_M \mathbf{A}'$ þá er líka $\models_M \mathbf{A}$.
    Svo að M er líkan fyrir $\mathcal{T}$. Þrepun yfir lengd yrðinga:
    \begin{enumerate}[(1)]
    \item Ef $\mathbf{A}$ er grunnyrðing, þá er þetta bein afleiðing af skilgr.
    \item $\mathbf{A}$ er $\lnot \mathbf{B}$. Ef $\models_M \mathbf{A}$, þá er ekki
      $\models_M \mathbf{B}$ og því ekki $\vdash_{\mathcal{T}} \mathbf{B}$ skv. þf.
      þar sem $\mathcal{T}$ er fullkomin og $\mathbf{B}$ lokað er
      $\vdash_{\mathcal{T}} \lnot \mathbf{B}$, þ.e. $\vdash_{\mathbf{A}}$. En ef
      ekki $\models_{M} \mathbf{A}$, þá er $\models_M \mathbf{B}$ og því
      $\vdash_{\mathcal{T}} \mathbf{B}$ skv. þf. En þar sem $\mathcal{T}$ er 
      samkvæm er ekki $\vdash_{\mathcal{T}} \lnot \mathbf{B}$ og því ekki
      $\vdash_{\mathcal{T}} A$.
    \item $\mathbf{A}$ er $\mathbf{B} \vee \mathbf{C}$: Ef $\models_M \mathbf{A}$,
      þá er annaðhvort $\models_M \mathbf{B}$ eða $\models_M \mathbf{C}$,
      svo að $\vdash_{\mathcal{T}} \mathbf{B}$ eða $\vdash_{\mathcal{T}} \mathbf{C}$ skv. þf.
      í báðum tilvikum er $\vdash_{\mathcal{T}} \mathbf{A}$ skv. sís.
      Ef ekki $\models_M \mathbf{A}$, þá er hvorki $\models_M \mathbf{B}$
      né $\models_M \mathbf{C}$ og því hvorki $\vdash_{\mathcal{T}}$ né 
      $\vdash_{\mathcal{T}} \mathbf{C}$.
      En $\mathcal{T}$ er fullkomin svo að $\vdash_{\mathcal{T}} \lnot \mathbf{B}$ 
      og $\vdash_{\mathcal{T}} \lnot \mathbf{C}$ og þá ekki $\vdash_{\mathcal{T}} \mathbf{A}$ skv. sís.

    \item $\mathbf{A}$ er $\forall \mathbf{x} \mathbf{B}$. Ef
      $\mathbf{B}$ er lokað, þá fæst $\models_M \mathbf{A}$ þþaa $\models_M \mathbf{B}$ sem skv.
      þf. gildi þþaa $\vdash_{\mathcal{T}} \mathbf{B}$ og það er jafngilt $\vdash_{\mathcal{T}} \mathbf{A}$A

      G.r.f. að $\mathbf{B}$ sé ekki lokað, þá er $\mathbf{x}$ eina breytan sem er frjáls í $\mathbf{B}$.A

      \begin{enumerate}[(a)]
      \item  G.r.f. að $\models_M \mathbf{A}$ en  ekki $\vdash_{\mathcal{T}} \mathbf{A}$.
        Skv. fullkomleika er $\vdash_{\mathcal{T}} \lnot \mathbf{A}$ og því
        $\vdash_{\mathcal{T}} \exists \mathbf{x} \lnot \mathbf{B}$. Þar sem
        $\mathcal{T}$ er hengin er til fasti $\mathbf{c}$ þa.a. 
        $\vdash_{\mathcal{T}} \lnot \mathbf{B}_{\mathbf{x}} [\mathbf{c}]$.
        En vegna $\models_M \forall \mathbf{x} \mathbf{B}$ svo að
        $\models_M \mathbf{B}_{\mathbf{x}} [\mathbf{c}]$ fyrir öll
        $\mathbf{c}$ í M.
        Skv. þf er $\vdash_{\mathcal{T}} \mathbf{B}_{\mathbf{x}} [\mathbf{c}]$. 
        Þetta er mótsögn, því að $\mathcal{T}$ er samkvæm.
      \item  G.rf. að $\vdash_{\mathcal{T}} \mathbf{A}$ en ekki $\models_M \mathbf{A}$.
        Vegna ekki $\models_M \forall \mathbf{x} \mathbf{B}$ er til úthlutun s
        þ.a. ekki $\models_M \forall \mathbf{x} \mathbf{B} (s)$ og því til $\mathbf{c}$ úr
        $|M|$ þ.a. ekki $\models_M \mathbf{B}_{\mathbf{x}} ( s_{\mathbf{x}}[\mathbf{c}])$;
        en það jafngildir $\models_M \mathbf{B}_{\mathbf{x}}[\mathbf{c}]$. En 
        $\vdash_{\mathcal{T}} \forall \mathbf{x} \mathbf{B}$ og því
        $\vdash_{\mathcal{T}} \mathbf{B}_{\mathbf{x}} [\mathbf{c}]$ og því
        $\models_M \mathbf{B}_{\mathbf{x}} [\mathbf{c}]$ skv. þf. 

        Þetta er mótsögn. Því sést:
        Ef $\vdash_{\mathcal{T}} \mathbf{A}$, þá $\vdash_{\mathcal{T}} \mathbf{A}$,
        þá $\models_M \mathbf{A}$.
      \end{enumerate}
    \end{enumerate}
    \end{proof}
\end{setn}

\begin{setn}[Löwenheim-Skolem-setningin]
Ef kenning fyrstu stéttar hefur teljanlegt stafróf,
þá hefur hún teljanlegt líkan.
\end{setn}

\begin{skgr}
  \begin{enumerate}[(1)]
  \item  Segjum að kenning $\mathcal{T}$ sé \emphs{hluti}
    af kenningu $\mathcal{T}'$ ef $\mathcal{T}$ og $\mathcal{T}'$ hafa
    sama mál og sérhver eiginleg frumsenda í $\mathcal{T}$ er frumsenda 
    í $\mathcal{T}'$
  \item Segjum að kenning $\mathcal{T}$ sé \emphs{endanleg frumsenduð}
    ef fjöldi eiginlegra frumsenda í $\mathcal{T}$ er endanleg.
  \end{enumerate}
\end{skgr}

\begin{ath}
 Hér teljum við \textbf{Eq 1} og \textbf{Eq 2} til rökfrumsenda fyrir
 samsemdarkenningar.
\end{ath}

\begin{setn}[Þjöppunarsetning]
  Yrðin í kenningu $\mathcal{T}$ er sönn í $\mathcal{T}$ ef (þ.e. sönn
  í sérhverju líkani fyrir $\mathcal{T}$ ) þþaa
  hún sé sönn í endanlega frumsenduðum hlut af $\mathcal{T}$.
  Sama gildir fyrir samsemdarkenningu
\end{setn}

\begin{proof}
  Höfum $\mathcal{T} \models \mathbf{A}$ þþaa $\vdash_{\mathcal{T}} \mathbf{A}$
  skv. fullkomleika setningu. En í sönnun $\mathbf{A}$ eru aðeins endalega
  margar eiginlegar frumsendur notaðar, svo að $\vdash_{\mathcal{T}} \mathbf{A}$
  jafngildir því að til sé endanlega frumsendaður hluti $\mathcal{T}_{1}$
  af $\mathcal{T}$ þ.a. $\vdash_{\mathcal{T}_1} \mathbf{A}$, en það jafngildir
  $\mathcal{T}_1 \models \mathbf{A}$
\end{proof}
\begin{setn}[Fylgisetn]
  Kenning $\mathcal{T}$ hefur líkan þþaa sérhver endanlega
  frumsendaður hluti af $\mathcal{T}$ hafi líkan.
  
  Samsemdarkenning hefur samsemdarlíkan þþaa sérhver endanlega frumsendaður
  hluti af $\mathcal{T}$ hafi samsemdarlíkan
\end{setn}

\begin{proof}
  Notum þjöppunarsetningu á yrðingu $\mathbf{A}$ af gerðinni
  $\mathbf{B} \wedge \lnot \mathbf{B}$. Kenning hefu líkan þþaa
  hún sé samkvæm þþaa ekki gildir $\vdash_{\mathcal{T}} \mathbf{B} \wedge \lnot \mathbf{B}$
  sérhver endanlega frumsendaður hluti af $\mathcal{T}$ hafi líkan.
\end{proof}

\begin{daemi}
  Notum þetta á kenningu $\mathcal{F}$ fyrir svið (stundum kallaðir kroppar);
  hún hefur svo fasta '$0$' og '$1$' og tvö tvístæð fallatákn $+$ og $\cdot$, er samsemdar
  kenning og hefur eiginlegar frumsendur.


  Látum $\mc{F}$ vera samsemdarkenning með föstum '$0$', '$1$',
 tvístæðum fallatáknum '$+$' og '$\cdot$' og eftirfarandi frumsendum:

 \begin{enumerate}[\textbf{F\arabic*}]
 \item  $' x + (y + z) = (x+y) + z'$
 \item '$x + 0 = x$'
 \item '$\forall x \exists y ( x+ y = 0)$'
 \item '$x+y = y +x$'
 \item '$(x \cdot y ) \cdot z = x \cdot (y \cdot z)$'
 \item '$x \cdot 1 = x$'
 \item '$x \neq 0 \to \exists y (x \cdot y = 1)$'
 \item '$x \cdot y = y \cdot x$'
 \item '$x \cdot (y + z) = x \cdot y + x \cdot z$'
 \item '$0 \neq 1$'
 \end{enumerate}

Samsemdarlíkan fyrir $\mc{F}$ eru \emphs{svið} (kroppar).

Látum $\bA_n$ vera yrðinguna '$1+1+ \dotsb + 1 = 0$', þar sem $1$ kemur fyrir
n-sinnum. Látum $\mc{F}$ vera kenninguna sem fæst með því að bæta við $\mc{F}$
frumsendunum
\[ \lnot \bA_2, \dotsc, \lnot \bA_{n -1} = \bA_n \]

fyrir $n \geq 2$. Samsemdar líkan fyrir $\mc{F}$ eru svið með kennitölu n.
Kennitala sviðs er alltaf frumtala, svo að $\mc{F}(n)$ er samkvæm \emph{þþaa}
n sé frumtala. Látum $\mc{F}(0)$ vera kenninguna sem fæst með að bæta við $\mc{F}$
öllum frumsendunum $\lnot \bA_n$ fyrir $n \geq 2$. \emph{Skv.} þjöppunarsetningunni
er $\bA$ sönn á sérhverjum endanlega frumsenduðum hluta $\cT_1$ í $\mc{F}(0)$.
Látum $n_0$ vera stærra en öll $n$ þ.a. $\lnot \bA_n$ sé frumsenda í $\cT_1$.

Þá er $\bA$ sönn í $\mc{F}(n)$ fyrir öll $n \geq n_0$,
þ.a. $\bA$ er sönn í öllum svið með kennitölu $\geq n_0$.


\end{daemi}

Afleiðing: Enginn endanlega frumsendaður hluti af $\mc{F}(0)$ er jafngildur $\mc{F}(0)$

Sýnum að ekki er til einföld útvíkkun $\mathcal{T}$ á $\mc{F}$ þ.a.
samsemdar líkön $\mc{T}$ séu nákvæmlega öll endanleg svið. G.r.f. að
slík kenning sé til. Látum $\mb{B}_n$ vera yrðingin sem segir að til séu
$n$ ólík stök, t.d. er $\mb{B}_{3}$ yrðingin
\[\exists x \exists y \exists z ( x \neq y \wedge x \neq z \wedge y \neq z)\]

ef við bætum öllum $\mb{B}_n$ við $\cT$ sem nýjum frumsendum, þá getur kenningin
$\cT'$ sem þannig fæst ekki haft líkan. Þar með er til endanlega frumsendaður hluti
$\cT''$ af $\cT'$ sem hefur ekkert líkan.

Látum $n_0$ vera stærra en öll $n$ þ.a. $\mb{B}_n$ sé eiginleg frumsenda í
$\cT''$ og K vera endanlegt svið þ.a. $\#K \geq n_0$. Þá er K líkan fyrir
$\cT''$ sem er mótsögn.

\begin{daemi}
  Látum $\cL$ vera mál með jafnaðarmerki, föstum '0', '1',
  falla táknum $'+'$ og $'\cdot'$ og tvístæðu umsagnartákni $<$.
  Þá er rauntalnasviðið $\R$ mynstur fyri $\cL$ með augljósum hætti.
  Látum $\cT$ vera mengi allra yrðinga á $\cL$ sem eru sannar í $\R$.
  Þá er $\cT$ kenning sem hefur teljanlegt stafróf og því teljanlegt líkan.

  Svo að $\cT$ gefur ekki fullkomna lýsingu á $\R$.

  Látum nú $\cL^{*}$ vera málið sem fæst með því að bæta við $\cL$ fasta 
  $\mb{r}_x$ fyrir sérhverja rauntölu $x$ og lítum á $\R$
  sem mynstur fyrir $\cL^*$  með því að 
  $(\mb{r}_x)_{\R} = x$. Látum $\cT^*$ vera mengi allra
  yrðinga í $\cL^*$ sem eru sannar í $\R$.
  Þá er $\R$ líkan fyrir $\cT^*$. Bætum fasta $\mb{c}$ við 
  $\cL^*$ og frumsendunum $\mb{r}_x < \mb{c}$ f. öll
  $x \in \R$; fáum þá kenningu $\cT^{**}$ sýnum að hún hefur líkan:
  Látum $x_1, \dotsc, x_n \in \R$ veljum $y > x_1, \dotsc, x_n$
  og setjum $\mb{c}_{\R} = y$


  Þá fæst líkan fyrir $\cT^*$ að viðbættum fumsendu
  $\mb{r}_{x_k} < r < \mb{c}$ fyrir $k = 1, \dotsc, n$; nefnilega 
  $\R$ sjálft!

  En þá hefur $\cT^{**}$ líkan ${}^* \R$ skv. þjöppunarsetningu. Þetta er 
  svið þ.a. allar setningar í $\cT^*$ eru sannar í ${}^* \R$; það inniheldur
  $\R$ sem hlutsvið (öllu heldur einsmóta eintak, nefninlega 
  $\set{ (\mb{r}_x)_{{}^* \R}: x \in \R}$ og til er tala 
  $y$ í ${}^* \R$, nefninlega $y := \mb{c}_{{}^* \R}$,
  þ.a. $x < y$ fyrir öll $x$ úr $\R$, því ${}^* \R$ er
  \emphs{óarkímedískt raðsvið} (og þá er $0 < \1{y} < x$ f. öll $x \in R$
\end{daemi}


\begin{setn}[Tarski]
  Látum $\cT$ vera samsemdarkenningu með stafrófi $\mc{S}$ 
  og $\lambda$  vera fjöldatölu þ.a. $\lambda \geq \# \mc{S}$.
  Ef $\cT$ hefur óendanlegt samsemdarlíkan, þá hefur það samsemdarlíkan með fjölda
  tölu $\lambda$.
\end{setn}

\begin{proof}
  Búum til nýja kenningu $\cT'$ með því að bæta mengi $C$ af nýjum
  föstum við $\cT$, þar sem $\# C = \lambda$, og hverja tvo ólíka
  fasta $\bc$ og $\bc'$ úr $C$ nýrri frumsendu $\bc \neq \bc'$.
  Sýnum að $\cT'$ hafi samsemdarlíkan.

  Skv. þjöppunarsetningu nægir að sýna að sérhver endanlega frumsendaður
  hluti af $\cT'$ hafi samsemdarlíkan. Látum $\bc_1, \dotsc, \bc_n$ vera
  fastana sem koma fyrir í nýjum
  frumsendum í $\cT_1'$ og $M$ vera óendanlegt samsemdarlíkan fyri $\cT$
  og $a_1, \dotsc, a_n$ 
vera ólík stök í $|M|$; gerum $M$ að mynztri fyrir $\cT_1'$ með því að láta
  $\bc_{k,M}$ vera $a_k$ og $\bc_M$ vera hvað sem er ef $\bc$ er í
  $C \setminus \set{\bc_1, \dotsc, \bc_n}$. Þá er $M$ samsemdarlíkan fyrir
  $\cT_1'$.

  Fjöldatala stafrófs $\cT'$ er í hæsta $\lambda + \lambda = \lambda$,
  svo að að getum valið $M$ með $\# |M| \leq \lambda$; en
  $\bc_M \neq \bc_M'$ f. öll $\bc, \bc'$ ílík í $C$. Svo að 
  $\# |M| \geq \lambda$ því er $\# |M| = \lambda$
  
  
\end{proof}

\begin{ath}
  Fyrir fjöldatölur $\lambda, \mu$A þ.a. $\mu \leq \lambda$, $\lambda$ óendanlegt
  $\mu \neq 0$, er 
  \[ \mu + \lambda = \mu X = X \]
\end{ath}


\chapter{IV}


\section{Rakin föll}

í þessum kafla er ``tala'' náttúruleg tala, ``fall'' er vörpun
$\N^{n} \to \N$, ``venzl'' er hlutmengi í $\N^{n}$. Leyfum okkur að nota
rökfræði tákn í yfirmálinu.

Kennifall n-stæðra vensla $R$ er 
$c_R: \N^n \to \N$,

\[ c_R (x_1, \dotsc, x_n) := \bcondef 0 & \Ef R(x_1, \dotsc,x_n) \\ 1 & \Ef \lnot R(x_1,\dotsc,x_n).\\ \econdef \]

(Skrifum $R(x_1,\dotsc, x_n)$ í stað $(x_1, \dotsc, x_n) \in R$.)

\begin{skgr}
  \emphs{Rakin föll} eru skilgrein með þrepun þannig:

  \begin{enumerate}[R\arabic*]
  \item  Eftirfarandi föll eru rakin:
    \begin{enumerate}[(a)]
    \item \emphs{Núllfallið} Z þ.a. $Z(x) = 0$
    \item \emphs{Eftirfarafallið} $N(x) = x+1$
    \item Ofanvörpun $I_i^n(x_1, \dotsc, x_n) := x_i$
    \end{enumerate}
  \item Ef $G,H_1, \dotsc, H_n$ eru rakin föll, þ'er fallið
    $F$ þ.a.
    \[ F(x_1,\dotsc,x_n) = G(H_1(x_1,\dotsc,x_n), \dotsc, H_n(x_1,\dotsc,x_n))\]
    rakið fall.
  \item Ef $G,H$ eru rakin föll og $F$ er skilgreint
    \begin{align*}
      F(x_1, \dotsc, x_n,0) & := G(x_1, \dotsc,x_n),\\
      F(x_1, \dotsc,x_n, yH) & := H(x_1, \dotsc, x_n, y, F(x_1, \dotsc, x_n, y))
    \end{align*}
    rakið fall.
  \item Ef $G$ er rakið fall, þ.a. $\forall x_1, \dotsc, \forall x_n  \exists y (G(x_1, \dotsc, x_n,y) = 0)$
    þá er fallið $F$ þ.a. 
    \[ F(x_1, \dotsc, x_n) := \mu y G(x_1, \dotsc, x_n,y) = 0)\]
    þar sem $\mu y(G(x_1,\dotsc,x_n,y) = 0)$ er minnsta $y$ þ.a.
    $G(x_1,\dotsc,x_n,y) = 0$ er rakið.
  \end{enumerate}

 Fall sem fæst með því að nota eingöngu $R_1, R_2, R_3$ 
 kallast \emphs{frumstætt rakið fall}.

 Við segjum að venzl $R$ séu rakin ef kennifallið 
 $c_R$ er rakið. Ef \[\forall x_1, \dotsc, \forall x_n  \exists y R(x_1, \dotsc, x_n,y)\]
 þá skrifum við
 \[\mu y R (x_1, \dotsc, x_n,y) := \mu y (G_R(x_1,\dotsc,x_n,y) = 0)\]
\end{skgr}


\begin{setn}
  \begin{enumerate}(1)
  \item Ef $G: \N^n \to \N$ er [frumstætt9 rakið fall,
    $i_1, \dotsc, i_k \in \set{1, \dotsc, n}$ og $F$ er gefið með
    \[ F(x_1, \dotsc,x_n) = G(x_{i_1}, \dotsc x_{i_k})\]
    þá er $F$ [frumstætt] rakið fall.
  \item Núllfallið $Z^n(x_1,\dotsc,x_n) = 0)$ er frumstætt rakið fall.
  \item Fyrir $k \in \N$ er fastafallið $K^n_k$ þ.a.
    $K^n_k ( \xxn) = k$ frumstætt rakið fall.
  \item í R4 má taka $n=0$ ef $H$ er [frumstætt] rakið fall,
    $k \in \N$ og $F: \N \to \N$ er skilgreint með
    \[ F(0) = k,\]
    \[ F(y+1) = H(y,F(y)) \]
    þá er $F$ [frumstætt] rakið fall.
  \end{enumerate}
\end{setn}


\begin{proof}
  \begin{enumerate}[(1)]
  \item Höfum $F(\xxn) = G(I^n_{i_1} (\xxn), \dotsc, I^n_{i_k}(\xxn))$
  \item $Z^n(\xxn) = Z(I^n_1 ( \xxn ))$.
  \item $K^n_0 (\xxn) = Z^(\xxn)$ og
    $K^n_{k+1} (\xxn) = N(K^n_k (\xxn))$.
  \item Skv. R3 er $G: N^2 \to \N$ þ.a.
    \[ G(x_1, 0) = K^1_k (x_1), \]
    \[ G(x_1, y +1) = H(I^2_2 (x_1, y), G(x_1,y))\]
  \end{enumerate}
  [frumstætt] rakið fall, og $F(y) = G(y,y)$
\end{proof}


\begin{setn}[og skilgreining]
  Eftirfarandi föll eru frumstæð rakin föll.
  \begin{enumerate}[(1)]
  \item $A(x,y) := x+y$.
  \item $M(x,y) := x \cdot y$.
  \item $V(x,y) := x ^ y$.
  \item $ \delta (x) := \bcondef x-1 & \Ef x > 0, \\ 0 & \Ef x = 0. \econdef$
  \item $ x-y := \bcondef x-y & \Ef x \geq y, \\ 0 & \Ef x < y. \econdef$
  \item $ |x-y| := \bcondef x-y & \Ef x \geq y, \\ y - x & \Ef x < y. \econdef$
  \item $ sg(x) := \bcondef 0 & \Ef x = 0 , \\ 1 & \Ef x > 0. \econdef$
  \item $ \overline{sg}(x) := \bcondef 1 & \Ef x = 0 , \\ 0 &  \Ef x > 0. \econdef$
  \item $x!$.
  \item $\min(\xxn)$.
  \item $\max(\xxn)$.
  \item $rm(x,y) = \text{ afgangur þegar x er deilt í y}$.
  \item $qf(x,y) = \text{ kvóti þegar x er deilt í y}$.
  \end{enumerate}
\end{setn}

\begin{proof}
  \begin{enumerate}[(1)]
  \item 
    \begin{gather*}
    A(x,0) = I^1_1(x),\\
    A(x,y+1) = N(A(x,y))
    \end{gather*}
  \item 
    \begin{gather*}
    M(x,0) = Z(x),\\
    M(x,y+1) = A(M(x,y),x)
    \end{gather*}
  \item 
    \begin{gather*}
    V(x,0) = K^1_1(x),\\
    V(x,y+1) = M(V(x,y),x)
    \end{gather*}
  \item 
    \begin{gather*}
    \delta(0) = 0,\\
    \delta(y+1) = y
    \end{gather*}
  \item 
    \begin{gather*}
    x - 0 = x,\\
    x - (y+1) = \delta(x-y)
    \end{gather*}

  \item 
    \begin{gather*}
    |x - y| = A(x-y,y-x),\\
    \end{gather*}

  \item 
    \begin{gather*}
    sg(0) = 0,\\
    sg(y+1) = K^1_1(y)
    \end{gather*}
  \item 
    \begin{gather*}
    \overline{sg}(x) = 1 - sg(x)
    \end{gather*}
  \item
    \begin{gather*}
      0! = 1 = K^1_1(x),\\
      (y+1)! = y! \cdot (y+1) = M(y!,y+1)
    \end{gather*}
  \item 
    \begin{gather*}
      \min(x,y) = x - (x-y), \\
      \min(\xxn, x_{n+1}) = \min( \min( \xxn), x_{n+1})
    \end{gather*}
  \item 
    \begin{gather*}
      \max(x,y) = y + (x-y), \\
      \max(\xxn, x_{n+1}) = \max( \max( \xxn), x_{n+1})
    \end{gather*}
  \item 
    \begin{gather*}
    rm(x,0) = 0,\\
    rm(x,y+1) = N(rm(x,y)) \cdot sg(|x-N(rm(x,y))|)
    \end{gather*}
  \item 
    \begin{gather*}
    qf(x,0) = 0,\\
    qf(x,y+1) = qf(x,y)) + \overline{sg}(|x-N(rm(x,y))|)
    \end{gather*}
  \end{enumerate}
\end{proof}

\begin{setn}
  Ef $F$ er [frumstætt] rakið fall, þá eru eftirfarandi föll [frumstætt] rakin:
  \begin{gather*}
    \sum_{y < z} F(\xxn,y), ( =: G(\xxn,z)) \\
    \sum_{y \leq z} F(\xxn,y),  \\
    \prod_{y < z} F(\xxn,y),  \\
    \prod_{y \leq z} F(\xxn,y),  \\
  \end{gather*}
\end{setn}

\begin{proof}
  \begin{gather*}
    G(\xxn,0) = 0 \\
    G(\xxn, z+1) = G(\xxn,z) + F(\xxn,z)\\
    \vdots
  \end{gather*}
  o.s.frv.
\end{proof}


\begin{daemi}
  Látum $t(0) = 1$ og $t(x)$ vera fjöldi deila
  tölunnar $x$ ef $x \geq 1$. Þá er
  $t$ frumstætt rakið, því að
  \[t(x) = \sum_{y \leq x} \overline{sg}(rm(y,x))\]
\end{daemi}

\begin{skgr}
  Setjum
 \[\mu y_{y< z} R(\xxn,y) := \bcondef \text{ minnsta $y$ þ.a. $y < z$ og $R(\xxn,y)$} & \Ef \text{ slíkt y er til} \\ z & \text{ annars}. \econdef\]
 \[ \forall y_{y<z} R(\xxn,y): \leftrightarrow \forall y ( y < z \rightarrow R(\xxn,y)),\]
 \[ \forall y_{y<z} R(\xxn,y): \leftrightarrow \exists y ( y < z \wedge R(\xxn,y));\]
 hliðstætt fyrir $\mu y_{y \leq z}, \forall y_{y\leq z}, \exists y _{y \leq z}$.
\end{skgr}

\begin{setn}
  Venzl sem fást úr [frumstætt] röknum venzlum með því að nota rökvirkjana
  '$\lnot$', '$\vee$' og takmarkaða magnara  eru [frumstætt] rakin.
  Föll sem fást úr [frumstæðum] röknum föllum með virkjunum
  $\mu y_{y < z}$ og $\mu y_{y \leq z}$ eru [frumstætt] rakin.
\end{setn}

\begin{proof}
  \begin{gather*}
    C_{\lnot R} (\xxn) = 1 - C_R(\xxn). \\
    C_{R_1 \vee R_2} (\xxn) = C_{R_1} (\xxn) \cdot C_{R_2}(\xxn)
  \end{gather*}
  Ef $\theta(\xxn,y): \leftrightarrow \exists y_{y < z} R(\xxn,y)$ þá er
  \[ C_R(\xxn) = \prod_{y<z} C_R(\xxn, y)\]
  $\exists y_{y\leq z}$ er jafngilt $\exists y_{y <  z+1}$, $\forall y_{y\ < z}$
  er jafngilt $\lnot \exists y_{y <  z}$ og
  \[\mu y_{y <  z} R(\xxn, y) = \sum_{y<z} (\prod_{u \leq y} C_R (\xxn, u))\]
  Því að $ \prod_{u < y}C_R(\xxn,u)$ er 1 f. öll $y$ þ.a. $R(\xxn,u)$ sé rangt.

  Fyrir öll $u$, en verður $0$ um leið og til er $u \leq y$ þ.a. $R(\xxn,u)$
  sé rétt. Svo að $\sum_{y<z} \prod_{u \leq y} C_R(\xxn,u)$ er földi allra staka
  frá $0$ upp í $y-1$
  þar sem $y < z$ er fyrsta $y$ þ.a. $R(\xxn, y)$ sé rétt, ef slíkt $y$ er til 
  en jafnt $z$ ef slíkst $y$ er ekki til.
\end{proof}


\begin{setn}
  Venzlin $x = y$, $x < y$, $x \leq y$ , $x > y$, $x \geq y$, $x | y$, $x \equiv y \text{ (mod } z)$.
  og $Pr(x)$ eru frumstæði rakin, þar sem $Pr(x)$ þyðir að $x$ sé frumtala.
  \begin{proof}
    \begin{gather*}
      C_=(x,y) = sg(|x-y|),\\
      C_<(x,y) = \overline{sg}(y \dda x),\\
      x \leq y \leftrightarrow x < y \vee x = y\\
      C_|(x,y) = sg(rm(x,y)),\\
      x \equiv y (mod z) \leftrightarrow z | |x-y|\\
      C_{Pr} (x) = sg(|t(x)-2|)\\
    \end{gather*}
    því að $p$ er prímtala þþaa fjöldi talna sem gengur upp í $p$ sé $2$.
  \end{proof}
\end{setn}

\begin{daemi}
  \begin{enumerate}[(1)]
  \item Látup $p_x$ vera $(x+1)$-stu frumtöluna í vaxandi röð. Þá
    er $p_x$ frumstætt rakið fall, því að
    \begin{gather*}
      p_0 = 2,\\
      p_{x+1} = \mu y_{y \leq (p_x)!+1} (p_x < y \wedge Pr(y)).
    \end{gather*}
    sérhverja nátt. tölu $x\geq 1$ má skrifa með
    nákv. einum hætti sem margfeldi
    \[ x = p_0^{v_0(x)} p_1^{v_1(x)} p_2^{v_2(x)} \dotsb \]
    þar sem $v_k(x) \in \N$ o(g $v_k(x) = 0$ f. öll nógu stór k);
    setjum til þæginda
    $v_k(x) := 0$. Sérhvert fall $v_k(x)$ er frumstætt rakið, því að
    \[ v_j(x) = \mu y_{y < x}(p_j^y | x \wedge \lnot p_j^{y+1} | x).\]
  \end{enumerate}
\end{daemi}

\begin{setn}
 Látum $G_1, \dotsc, G_k$ vera [frumstætt] rakin föll og $R_1, \dotsc, R_k$ vera
 [frumstæð] rakin þ.a. fyrir sérhvert $(\xxn)$ sé nákvæmlega eitt
 af $R_1(\xxn), \dotsc, R_k(\xxn)$ satt. Þá er fallið $F$ þ.a.
 \[ F(\xxn) := \bcondef G_1(\xxn) & \Ef R_1(\xxn)\\ \vdots & \vdots \\ G_k(\xxn) & \Ef R_k(\xxn) \econdef \]
 er frumstætt rakið fall.
 \begin{proof}
   \[F = G_1 \cdot C_{\lnot R_1} + \dotsb + G_k \cdot C_{\lnot R_k} \]
 \end{proof}
\end{setn}

Oft eru föll $f$ skilgreind með þrepun þ.a.
í skgr. á $f(y+1)$ eru öll gildi $f(0), \dotsc, f(y)$
notuð. Setjum
\[ f^{\#}(\xxn,y) = \prod_{u < y}p_n^{f(\xxn,u)} \]
Getum reiknað $f$ útfrá $f^{\#}$ með
\[ f(\xxn,y) = v_y (f^{\#}(\xxn,y+1)) \]

\begin{setn}
  Ef $H$ er [frumstætt] rakið fall og f fullnægi
  \[ f(\xxn, y) = H(\xxn,y, f^{\#}(\xxn,y))\]
  þá er f [frumstætt] rakið fall.
\end{setn}

\begin{proof}
  $f^{\#}$ er [frumstætt] rakið, því að
  \begin{align*}
    f^{\#}(\xxn,0) = 1& ,\\
    f^{\#}(\xxn,y+1) &= f^{\#}(\xxn,y) \cdot p_y^{f(\xxn,y)}\\
    &= f^{\#}(\xxn,y) \cdot p_y^{H(\xxn,y,f^{\#}(\xxn,y)))}\\
  \end{align*}
  þar með er $f(\xxn,y) = v_y(f^{\#}(\xxn,y+1))$ það líka.
\end{proof}

\begin{daemi}
  Fibonaci-runan er skilgreind með 
  $f(0) = 1, f(1) = 1$ og \[f(k+2) = f(k)+ f(k+1)\].
  Höfum:
  %\[f(k) = \overline{sg}(k) + \overline{sg}(|k-1|) + (v_{k-1}(f^{\#}(k))+v_{k-2}(f^{\#}(k))) \cdot sg(k \dda 1)  = H(k, f^{\#} (k)) \] % svona hjá tandra
  \[f(k) = \overline{sg}(k) + \overline{sg}(|k-1|) + (v_{k-1}(f^{\#}(k))+v_{k-2}(f^{\#}(k))) \cdot sg(k \dda 2)  = H(k, f^{\#} (k)) \]

  þar sem  
  \[ H(y,z) =  \overline{sg}(y) + \overline{sg}(|y-1|) + (v_{y-1}(z) + v_{y-2}(z))\cdot sg(y \dda 1)\]
  er frumstætt rakið fall, svo að $f$ er frumstætt.
\end{daemi}

\begin{setn}[Hjálparsetning]
  Fallið $Q: \N^2 \to \N, Q(x,y) = (x+y)^2 +x + 1$ er eintækt.
  \begin{proof}
    G.r.f. að $Q(x,y) = Q(s,t)$. Ef $x+y < s+t$, þá er
     \[ Q(x,y) \leq (x+y +1)^2 \leq (s+t)^2 < Q(s,t),\]
     sem er mótsögn, eins ef $s+t < x+y$. Því er $x+y = s+t$,
     en þá fæst að $x = s$ og þá $y = t$.
  \end{proof}
\end{setn}

\begin{setn}
  Til er ákveðið frumstætt rakið fall $\beta: \N^2 \to \N$
  þ.a. $\beta (a,i) \leq a \dda 1$ og þ.a. fyrir öll $a_0, \dotsc, a_{n-1}$ er til
  tala $a$ þ.a.
  \[\beta(a,i) = a_i \]
  fyrir öll $i$ þ.a. $0 \leq i < n$
  \begin{proof}
    Setjum $Q(x,y) := (x+y)^2 + x +1$ og
    \[ \beta(a,i) = \mu x_{x \leq a - 1} \exists y_{y<a} \exists z_{z<a} (a = Q(y,z) \wedge 1 + (Q(x,i)+1) \cdot z | y ) \]
    Látum $a_0, \dotsc, a_{n-1}$ vera fegin. Fáum
    \[ c:= \max (Q(a_0,0) + 1, \dotsc, Q(a_{n-1},n-1) +1 ) \]
    
    Setjum $ z := c!$. Fyrir $j < l < c$ eru tölurnar
    $1 +jz$ og $1 + lz$ ósamþátta, því að sameiginlegur frumþáttur
    $p$ gengur upp í $(1+lz) - (1+jz) = (l-j)z$, en $l-j < c$, svo að
    $l-j | z$ og því $p|z$, sem er fráleitt (þá fengist $p | 1$). Skv.
    kínversku leifasetningunni er til tala $y$ þ.a. fyrir $j = =1, \dotsc, c-1$
    gildi $1+jz|y$ þþaa $j$ sé ein af tölum $Q(a_i,i)+1, i \in \set{0,\dotsc,n-1}$.
    Setjum $a := Q(y,z)$. Þá er $a_i < y < a$ og $ z < a$.

    Skv. HS fæst: Til að sýna að $\beta(a,i) = a_i$ fyrir $i = 0, \dotsc, n-1$
    nægir að sýna að $a_i$ sé minnsta tala $x$ þ.a.
    $1+(Q(x,i)+1)\cdot z$ gangi upp í $y$. En ef $x < a_i$,
    þá er $Q(x,i) < Q(a_i,1) < c$ og $Q(x,i)$ er
    ekki ein af tölum $Q(a_j,i)$, þ.a. $1 + (Q(x,i)+1)\cdot z$
    gengur ekki upp í $y$.
  \end{proof}
\end{setn}

\begin{skgr}
  Köllum $\beta$ Gödel-fallið.

\end{skgr}

\begin{ath}
  
  Vegna $\beta(a,i) \leq a - 1$ er 
  \[\beta(0,i) = 0\]
  og fyrir $a \neq 0$ er $\beta(a,i) < a$.
\end{ath}

\begin{ath}
  Reynir notar í þessum kafla að ofan táknið $\dot{-}$,
  en ég hef táknað það með $-$. Það ætti þó að laga þetta.
\end{ath}


\section{Framsetning í $\mathcal{N}$}

Rifjum upp að $\cN$ er samsemdarkenningin með fasta '$0$', einstætt
fallatákn '$S$', tvö tvístæð fallatákn '$+$' og '$\cdot$' og
eitt tvístætt umsagnartákn '$<$' (auk '$=$') og eftirfarandi eiginlegum
frumsendum:

\begin{enumerate}[label=\textbf{N\arabic*}]
\item  \( Sx \neq 0\)
\item  $ Sx = Sy \rightarrow x = y$.
\item  $ x + 0 = x$.
\item  $ x + Sy = S(x+y)$.
\item  $ x \cdot 0 = 0$.
\item  $ x \cdot Sy = (x \cdot y) + x$.
\item  $ \lnot (x < 0)$.
\item  $ x < Sy \leftrightarrow x < y \vee x = y$.
\item  $ x < y \vee x = y \vee x > y$.
\end{enumerate}

Heitin '$0$', '$S0$', '$SS0$', $\dotsc$ o.s.frv. kallast
tölutákn látum $\mb{k}_n$ vera tölutáknið sem hefur nkvl. $n$ '$S$'.

\begin{skgr}
  Látum $\cT$ vera kenningu með sama máli og $\cN$. Við segjum
  að yrðingin $\bA$ ásamt ólíkum breytum 
  $\bx_1, \dotsc, \bx_n ,\mathbf{y}$ sé framsetning á n-stæðu fallinu
 $F$ í $\cT$ ef fyrir öll $a_1, \dotsc, a_n$ í $\N$ gildir
 \[ \vdash_{\cT} \bA_{\bx_1, \dotsc, \bx_n} [ \bk_{a_1}, \dotsc, \bk_{a_n}] \leftrightarrow y = \bk_b \]
 þar sem $ b = F( a_1, \dotsc, a_n)$. Segjum að $F$ sé framsetjanlegt í
 $\cT$ ef slík framsetning er til.

 Segjum að yrðing $\bA$ ásamt ólíkum breytum $\bxxn$ sé framsetning í $\cT$
 á n-stæðum venzlum $P$ ef f. öll $\aan$ úr $\N$ gildi
 \begin{enumerate}[(i)]
 \item  ef $P(\aan)$, þá $\vdash_{\cT} \bA_{\bxxn}[\bk_{a_1}, \dotsc, \bk_{a_n}]$;
 \item  ef $\lnot P(\aan)$, þá $\vdash_{\cT} \lnot \bA_{\bxxn}[\bk_{a_1}, \dotsc, \bk_{a_n}]$
 \end{enumerate}
 segum þá að $P$ sé \emphs{framsetjanlegt} í $\cT$.

 Segjum að heiti $\ba$ ásamt ólíkum breytum $\bxxn$ sé framsetning á 
 n-stæðu falli $F$ í $\cT$ ef f. öll $\aan$ í $\N$
 gildir:
 \[ \vdash_{\cT} \ba_{\bxxn}[\bk_{a_1}, \dotsc, \bk_{a_n}] = \bk_b \]
 þar sem $b = F(\aan)$
\end{skgr}

\begin{ath}
  Ef heitið $\ba$ ásamt $\bxxn$ er framsetning á $F$ í $\cT$ og
  $\by$ er ný breyta, þá er yrðingin $\by = \ba$ ásamt
  $\bxxn, \by$ framsetning á $F$ í $\cT$.
\end{ath}

\begin{daemi}
  \begin{enumerate}[(1)]
  \item  Yrðingin '$ x = y$' ásamt '$x$', '$y$' er framsetning á 
    venzlum $=$ í $\cN$. Til að sanna það þarf að sýna:
    \begin{enumerate}[(i)]
    \item  ef $m = n$, þá er $\bk_m = \bk_n$
    \item  ef $m \neq n$, þá er $\bk_m \neq \bk_n$
    \end{enumerate}

    Staðhæfing (i) fæst úr $\mathbf{EQ}_1$ með innsetningu.
    Til að sýna (ii) má gera ráð fyrir að $m > n$.
    Notum þrepun yfir $n$. Ef $n = 0$, fæst þetta úr \textbf{N1}.
    Látum þá $n > 0$. Með innsetningu í \textbf{N2} fæst
    \[ \vdash_{\cN} \bk_m = \bk_n \rightarrow \bk_{m-1} = \bk_{n-1}.\]
    Skv. þf. er $\vdash_{\cN} \bk_{m-1} \neq \bk_{n-1} $ og þá
    $\vdash_{\cN} \bk_m \neq k_n$ skv. sís.
  \item Heitið '$0$' ásamt '$x$' er framsetning á núllfallinu
    $Z: \N \to \N$, því að $\vdash_{\cN} 0 = 0$ og
    $0_x [\bk_a]$ er '$0$'.
  \item Heitið '$Sx$' ásamt '$x$' er framsetning á eftrifara fallinu
    $N(x) = x + 1$ í $\cN$; það þýðir að
    \[ \vdash_{\cN} S \bk_n = \bk_{n+1}.\]
    En $S\bk_n = \bk_{n+1}$, svo að þetta fæst úr
    $\mathbf{EQ}_1$ með innsetningu.
  \item Heitið '$x_i$' ásamt '$x_1$',$\dotsc$,'$x_n$' er
    framsetning á ofanvarpinu $I^n_i$. Ef
    $\bA$ er $'x_i'$, þá er $\bA_{\bxxn}[\bk_{a_1}, \dotsc, \bk_{a_n}]$
    heitið $\bk_{a_i}$ og 
    \[ \vdash_{\cN} \bk_{a_i} = \bk_{a_i} \]
    skv. \textbf{EQ1}
  \item Heitið $'x+y'$ ásamt $'x', 'y'$ er framsetning í $\cN$ á
    samlagningunni $\N^2 \to \N$,
    $(x,y) \mapsto x+y$. Til að sanna það þarf að sýna að
    \[ \vdash_{\cN} \bk_m + \bk_n = \bk_{m+n} \]
    Þrepum yfir $n$. Fyrir $ n =0$ er þetta
    \[ \vdash_{\cN} \bk_m + 0 = \bk_{m} ,\]
    sem fæst úr \textbf{N3} með innsetningu.
    G.r.f. að $\vdash_{\cN} \bk_m + \bk_n = \bk_{m+n}$
    Fyrir fast tiltekið $n$ (og öll $m$). Skv.
    setningu um samsendarkenningar
    \[ \vdash_{\cN} S(\bk_m + \bk_n) = S \bk_{m+n}\]
    þ.e.

    \[ \vdash_{\cN} S(\bk_m + \bk_n) = \bk_{m+n+1}\]
    skv. N4 er
    \[ \vdash_{\cN} S(\bk_m + \bk_n) = \bk_{m} + \bk_{n+1}\]
    svo að
    \[ \vdash_{\cN} \bk_m + \bk_{n+1} = \bk_{m+n+1}.\]
    Eins sést að tilteki '$x\cdot y$' ásamt $'x'$, '$y$' er
    framsetning  á margfölduninni $\N^2 \to \N$ í $\cN$.
  \item  Yrðingin $'x < y'$ ásamt '$x$' og '$y$' er
    framsetning í $\cN$ á venzlunum $<$ í $\N^2$. Sýna þarf:
    \begin{enumerate}[(i)]
    \item  Ef $m < n$ ,þá er $\vdash_{\cN} \bk_m < \bk_n$
    \item  Ef $m \geq n$ ,þá er $\vdash_{\cN} \lnot (\bk_m < \bk_n)$ 
    \end{enumerate}
    Fyrir $n = 0$ gildir (i) sjálfkrafa og (ii) fæst úr \textbf{N7}.
    G.r.f að (i) og (ii) gildi fyrir tiltekið $n$. skv. \textbf{N8}
    fæst
    \[ \vdash_{\cN} \bk_m < \bk_{n+1} \leftrightarrow (\bk_m < \bk_n \vee \bk_m = \bk_n) (^*) \]
    Gerum fyrst ráð fyrir að $m < n+1$. Ef $m<n$ þá er $\vdash_{\cN} \bk_m < \bk_n$
    skv. þf. en ef $ m = n$, þá er $\vdash_{\cN} \bk_m = \bk_n$.

    Í báðum tilvikum gefur $(^*)$ okkur að $\vdash_{\cN} \bk_m < \bk_{n+1}$.
    Gerum næst ráð fyrir að $m \geq n+1$. Þá er $m \geq n$
    því $\vdash_{\cN} \lnot (\bk_m < \bk_n)$ skv. þf. og 
    $m \neq n$ og því $\vdash_{\cN} \lnot (\bk_m < \bk_{n+1})$
    skv. $(^*)$ og \emph{sís}.
  \end{enumerate}
\end{daemi}

\begin{setn}
  Venzl P eru framsetjanleg í $\cN$ þþaa kennifallið $C_P$ sé framsetjanlegt í $\cN$
\end{setn}

\begin{proof}
  Gerum fyrst r.f. að $\bA$ ásamt $\bxxn$ sé framsetning á $P$ í $\cN$
  látum $\bB$ vera yrðinguna
  \[ ( \bA \wedge \by = \bk_0) \vee (\lnot \bA \wedge \by = \bk_1)\]
  Þar sem $\by$ er ný breyta. sýnum að $\bB$
  ásamt $\bxxn, \by$ er framsetning á $C_P$.
  G.r.f. að $C_P (\aan) = 0$. Þá er $P(\aan)$
  og því $\vdash_{\cN} \bA_{\bxxn} [\bk_{a_1}, \dotsc,\bk_{a_n}]$; af því og \emph{sís}
  fæst
  \[ \vdash_{\cN} \bB_{\bxxn} [ \bk_{a_1}, \dotsc,\bk_{a_n} ] \leftrightarrow \by = \bk_0 \]
  Samskonar rök ef $C_P (\aan) = 1$.

  Gerum næst r.f. að $\bA$ ásamt $\bxxn, y$ sé framsetning á $C_P$. Sýnum að
  $\bA_{\by}[0]$ ásamt $\bxxn$ er framsetning á $P$.
  \begin{enumerate}[(i)]
  \item Ef $P(\aan)$, þá er $C_P(\aan) = 0$ og því
    \[\vdash_{\cN} \bA_{\bxxn}[\bk_{a_1}, \dotsc,\bk_{a_n}] \leftrightarrow \by = \bk_0 \]
    Setjum $\bk_0$ inn fyrir $\by$. Vegna $\vdash_{\cN} \bk_0 = \bk_0$
    fæst \[\vdash_{\cN} ( \bA_{\by}[0] )_{\bxxn}[\bk_{a_1}, \dotsc,\bk_{a_n}]\]
  \item Ef $\lnot P(\aan)$ þá er $C_P ( \aan) = 1$
    og \[\vdash_{\cN} \bA_{\bxxn} [ \bk_{a_1}, \dotsc,\bk_{a_n}] \leftrightarrow \by = \bk_1.\]
    Setjum $\bk_0$ inn fyrir $\by$, vegna $\vdash_{\cN} k_0 = k_1$, fæst
    \[ \vdash_{\cN} \lnot (\bA_{\by}[0])_{\bxxn} [\bk_{a_1}, \dotsc,\bk_{a_n}]\]
  \end{enumerate}
\end{proof}



\[\vdots\]

Viljum sanna:

\begin{setn}[Framsetningarsetning]
Öll rakin föll eru framsetjanleg í $\cN$.
\end{setn}

Höfum séð að $N, Z, I_k^n$ eru framsetjanleg. Þurfum að sýna að 
föll sem eru búin til úr framsetjanlegum föllum með reglunum 
R2, R3 og R4 séu framsetjanleg. Gerum það í röð af hjálpar setningum.

\begin{setn}[HS1]
 Ef $G, H_1, \dotsc, H_k$ eru framsetjanleg og F er skilgrein með
 \[ F(\aan) = G(H_1(\aan), \dotsc, H_k(\aan)) \]
 þá er F framsetjanlegt.
\end{setn}
\begin{proof}
  Látum $\bxxn, \byyk, \bz$ vera ólíkar breytur, $\bA_i$ vera
þ.a. $\bA_i$ ásamt $\bxxn, \by_i$ sé framsetning á $H_i$ og
$\bB$ þa $\bB$ ásamt $\byyk, \bz$ sé framsetning á $\bG$. Látum $\bC$
vera 
\[ \exists \by_1, \dotsc, \exists \by_k ( \bA_1 \wedge \dotsb \wedge \bA_k \wedge \bB )\]

Þá er $\bC$ ásamt $\bxxn, \bz$ framsetning á $F$.

Látum $F(\aan) = C$. Setjum $b_i := H_i (\aan)$.
Þá er $G(b_1, \dotsc,b_k) = c$. Látum $\bA'$ vera $(\bA_i)_{\bxxn}[\bkaan]$

og $\bC'$ vera $\bC_{\bxxn}[\bkaan]$. Þá er \[ \vN \bA_i' \leftrightarrow \by_i = \bk_{b_i} \]

Skv. jafngildissetningu fæst
\[\vN \bC' \leftrightarrow  \exists \by_1, \dotsc, \exists \by_k ( \bA_1 \wedge \dotsb \wedge \bA_k \wedge \bB )\]
en skv. setningu um samsemdarkenningar er

\[ \vN \bC' \leftrightarrow \bB_{\byyk} [\bk_{b_1}, \dotsc \bk_{b_k}] \]

en
\[ \vN \bB_{\byyk} [\bk_{b_1}, \dotsc \bk_{b_k}] \leftrightarrow \bz = \bk_c \]
svo að \[ \vN \bC' \leftrightarrow z = \bk_c\]

\end{proof}


\begin{setn}[HS2]
 \[ \vN \bA_{\bx}[\bk_0] \wedge \dotsb \wedge \bA_{\bx}[\bk_{n-1}] \wedge \bx < \bk_n \to \bA \]

 \begin{proof}
   Þrepun yfir n...
 \end{proof}
\end{setn}

\begin{setn}[HS3]
  Ef $\vN \lnot \bA_{\bx}[\bk_i]$ fyrir öll $i < n$ og $\cN \bA_{\bx}[\bk_n]$,
  þá \[ \vN \bA \wedge \forall \by ( \by < \bx  \to \lnot \bA_{\bx}[\by]) \leftrightarrow \bx = \bk_n \]

 \begin{proof}
...
 \end{proof}
\end{setn}

\begin{setn}[HS4]
 Ef $G$ er framsetjanlegt í $\cN$ og fullnægir $\forall x_1 \dotsc \forall x_n \exists y ( G(\xxn,y) = 0)$
 og $F(\aan) = \mu \times (G(\aan,x) = 0)$, þá er $F$ framsetjanlegt í $\cN$.
 \begin{proof}
   Látum $\bA$ ásamt $\bxxn, \by, \bz$ vera framsetningu á $G$ í $\cN$. 
   Látum $\bw$ vera nýja breytu og $\bB$ vera 
   \[ \bA_{\bz}[0] \wedge \forall \bw ( \bw < \by \to \lnot \bA_{\by, \bz}[\bw,0])\]

   Sýnum að $\bB$ ásamt $\bxxn, \by$ er framsetning á $F$ í $\cN$.
   Látum $F(\aan)  = b$ og setjum $c_i = G(\aan, i)$.
   Látum $\bA'$ og $\bB'$ fástum úr $\bA$ og $\bB$ með því að setja $\bkaan$ inn
   fyrir $\bxxn$. $\vN \bA_{\by}' [\bk_i] \leftrightarrow \bz = \bk_{c_i}$
   Ef $i < b$, þá er $c_i \neq 0$ og þá $\vN \bk_0 \neq \bk_{c_i}$ og þar með
   
   (1) $\vN \lnot \bA_{\by,\bz}'[\bk_i,0]$ ef $i < b$.

   Vegna $c_b = 0$ er $\vN \bk_0 = \bk_{c_b}$ og því

   (2) $\vN \lnot \bA_{\by,\bz}'[\bk_b,0]$.

   Af (1) og (2) og \emph{HS3} fæst
   $\vN \bB' \leftrightarrow y = \bk_b$.
   sem sýna átti.
 \end{proof}
\end{setn}

(Höfum nú afgreitt R2 og R4)
\begin{setn}[HS5]
  Ef n-stæður venzlin $R_1, R_2$ eru framsetjanleg, þá eru venzlin $R_1 \wedge R_2$ (þ.e. $R_1 \cap R_2$ ) framsetjanleg.
  \begin{proof}
    Látum $\bA$ ásamt $\bxxn$ vera framsetningu á $R_i$ , $i = 1,2$. 

    Ef $(R_1 \wedge R_2)(\aan)$, þá gildir $R_1(\aan)$ og $R_2(\aan)$. og því
   \[ \vN (\bA_i)_{\bxxn}[\bkaan], \text{ fyrir } i = 1,2\]
   og því
   \[ \vN (\bA_1 \wedge \bA_2)_{\bxxn}[\bkaan]\]
   Ef $\lnot (R_1 \wedge R_2)(\aan)$, þá er annað hvort $\lnot R_1(\aan)$ eða $\lnot R_2(\aan)$. Í fyrra
   tilvikinu er $\vN \lnot (\bA_1)_{\bxxn}[\bkaan]$ og því $\vN \lnot (\bA_1 \wedge \bA_2)_{\bxxn}[\bkaan]$.
  \end{proof}
\end{setn}




\begin{setn}[HS6]
  Látum venzlin $\cR(\xxn,y)$ vera framsetjanleg i $\cN$ og
  \[\cS(\xxn,y): \leftrightarrow \exists y_{y < z} R(\xxn, y)\]
  þá eru venslin $\cS$ framsetjanleg í $\cN$.
\end{setn}


\begin{proof}
Látum $\bA$ ásamt $\bxxn, \by$ vera framsetningu á $\cR$ í$\cN$ og $\mb{w}$
vera nýja breytu og $\bB$ vera 
\[ \exists \by (\by < \mb{w} \wedge \bA) \]

Sýnum að $\bB$ ásamt $\bxxn, \by$ er framsetning á $\cS$ í $\cN$.
G.r.f að $\bB(\aan, c)$. Þá er ti b þ.a $b < c$ og $\cR(\aan, b)$;
þá er $\vN \bk_b < \bk_c$ og $\vN \bA_{\bxxn, \by} [\bkaan, \bk_b] $
því
  \[\vdash_{\cN} \bk_b < \bk_c \wedge \bA_{\bxxn} [ \bk_{a_1},
  \dotsc,\bk_{a_n},\bk_b] .\] Af innsetningarsetningu leiðir að
  \[\vdash_{\cN} \exists y ( y < \bk_b \wedge \bA_{\bxxn} [ \bk_{a_1},
  \dotsc,\bk_{a_n}])\] þ.e.
 
  \[\vdash_{\cN} \bB_{\bxxn,\mb{w}} [ \bk_{a_1},
  \dotsc,\bk_{a_n},\bk_b] .\]

  G.r.f. að $\lnot S(\aan,c)$. Þá gildir $\lnot R(\aan, i)$ fyrir öll
  $i = 1, \dotsc, c-1$.

  Og því að
  \[\vdash_{\cN} \lnot \bA_{\bxxn,\by} [ \bk_{a_1},
  \dotsc,\bk_{a_n},\bk_i]\] fyrir öll $i = 1, \dotsc, c-1$.  Ef $\bC$
  er \(\vdash_{\cN}\bA_{\bxxn} [ \bk_{a_1}, \dotsc,\bk_{a_n}]\) má
  skrifa þetta
  \[\vdash_{\cN} \lnot \bC_{y} [ \bk_{i}] , i = 0, \dotsc, c-1 \]

  skv. HS2 er

  \[\vdash_{\cN} \lnot (y < \bk_c \wedge \bC)\]

  Skv. \textbf{Ath} fæst

  \[ \vdash_{\cN} \forall y \lnot (y < \bk_c \wedge \bC) \] þ.e.

  \[ \vdash_{\cN} \lnot \exists y (y < \bk_c \wedge \bC) \]

  og það þýðir:

  \[ \vdash_{\cN} \lnot \bB_{\bxxn} [\bkaan, \bk_c] \]
\end{proof}


\begin{setn}[HS7]
  Gödelfallið $\beta$ er framsetjanlegt í $\cN$.
\end{setn}
\begin{proof}
  Munum að
  \[ \beta(a,i) = \mu x_{\bx \leq \dot{-} 1} \exists y_{y<a} \exists z_{z < a}\]

\[\vdots\]

Ljóst er að $Q$.
\[\vdots\]

eru framsetjanleg í $\cN$ og þá er $\beta(a,i) = \mu x R(x,i)$
framsetjanlegt í $\cN$.
\end{proof}

\begin{setn}[HS8]

Látum $G,H$ vera framsetjanlegt í $\cN$, og $F$ vera skilgreint með
\begin{gather*}
  F(\xxn,0) = G( \xxn)\\
  F(\xxn, y+1) = H(\xxn,F(\xxn,y))
\end{gather*}
þá er $F$ framsetjanlegt í $\cN$.
\end{setn}

\begin{proof}
  Látum $\bA$ ásamt $\bxxn, \by$ vera framsetningu á $G$ í $\cN$
  og $\bC$ ásamt $\bxxn,\by,\bz,\bv$ vera framsetningu á Gödelfallinnu $\beta$
  í $\cN$. Látum $\bD$ vera

  \begin{gather*}
    \exists \mb{u} [ \exists \mb{w} ( \bB_{\mb{u},\mb{y},\mb{z}} [
    \mb{u}, 0, \mb{w}] \wedge \bA_y [w]) \wedge
    \bB_{\mb{u},\mb{y},\mb{z}}[\mb{u},\mb{y},\mb{z}]\\ \wedge
    \exists \mb{w} (\mb{w} < \by \rightarrow \exists \mb{t} \exists \mb{s} ( \bB_{\mb{y},\mb{z}}[\mb{w};\mb{t}] \wedge \bB_{\mb{y},\mb{z}}[S\mb{w}, \mb{s}] \wedge C_{\mb{u},\mb{y},\mb{z}}[\mb{w}, \mb{t}, \mb{s}]]]]
  \end{gather*}

  Viljum sýna að $\bD$ ásamt $\bxxn, \by, \bz $ sé framsetning á $F$ í $\cN$
  þ.e.: Ef $F(\aan,b) = c$ þá er
  \[\vdash_{\cN} \bD_{\bxxn,\by} [\bkaan, \bk_b] \leftrightarrow \bz = \bk_c\]
  Við gefum óformlega sönnun og látum lesanda eftir að þýða yfir í $\cN$.
  Með því að nota $\bA, \bB$ og $\bC$ sem yrðingar um Föllin
  $G, \beta, H$ og skrifa $'z'$ fyrir $\bz$ sést að 
  yrðingin $\bD_{\bxxn,y}[\bkaan,\bk_b]$ þýðir að eftirfarandi skilyrðum sé fullnægt.
  \begin{enumerate}[(i)]
  \item Til er w þ.a. 
    \[ \beta(u,0) = w \]
    og \[G(\aan) = w\]
  \item $\beta(u,b) = z$

  \item fyrir öll $w < b$ eru til $t$ og $s$ þ.a.

    \begin{gather*}
      t = \beta(u,w), s = \beta(u,w+1)\\
      s = H(\aan,w,t)
    \end{gather*}
  \end{enumerate}

  Með þí að setja $d_i = \beta(u,i)$ sést að þetta segir eftirfarandi:

  \[ (*) \bcondef d_0 = C(\aan) \\ d_{i+1} = H(\aan,i,d_i) \text{ fyrir } i < 0
  \\ d_b = z \econdef \]
  En af því sést að $d_i = F(\aan,i)$ og þá $z = F(\aan,b) = c$.
  Öfugt ef $z := F(\aan,b)$ og skilgreinum $d_i := F(\aan,i)$,
  þá fullnægja $k_i$-in $(*)$. Látum u vera
  $\beta(u,i) = d_i$ fyrir $i = 0, \dotsc,b$, þá er skilyrði (i)-(iii) fullnægt.

\end{proof}
Þar með er aðal setningin sönnuð!

\chapter{Ófullkomleikasetningar}

\section{Gödel-tölusetning}

Látum $\beta$ vera Gödel-fallið. Fyrir sérhverja 
n-un ($\aan$) táknum við með
\[< \aan > \]

minnstu tölu $a$ þ.a. $\beta(a,0) = n$ og 
$\beta(a,i) = a_i$ fyrir $i = 1, \dotsc, n$. Köllum
$<\aan>$ \emphs{runutölu} $n$-undarinnar
$(\aan)$. Leyfum $n=0$ og höfum $<> = 0$.

Fyrir fast $n$ er 
\[ F(\aan) := < \aan>\]
rakið fallið því að
\[< \aan> = \mu x(\beta(x,0) = n \wedge \beta(x,1) = a_1 \wedge \dotsb \wedge \beta(x,n) = a_n )\]

Setjum líka

\begin{gather*}
  ln(a) = \beta(a,0) \text{   (líka skrifaðl lh)} \\
  (a)_i = \beta(a,i+1)
\end{gather*}
(Vegna $\beta(a,i) \leq a \dda -1$ sést:
Ef $a \neq < >$, þá er $ln(a) < a$ og $(a)_i < a$.)
Ef $a = < a_0 \dotsc, a_{n-1}$, þá er
\begin{gather*}
  n = ln(a)\\
  (a)_i = a_i \text{ fyrir } i = 0, \dotsc, n-1
\end{gather*}

Látum Seq vera mengi allra rauntalna. Það er rakin einstæð venzl vegna
\[Seq(a) \leftrightarrow \forall x_{x < a} ( ln(x) \neq ln(a) \vee \exists i_{i < ln(a)} ((x)_i = (a)_i))\].

Látum nú $\cT$ vera kenningu fyrstu stéttar með mál $\cL$.

Komum okkur saman um eftirfarandi:

\begin{enumerate}[(1)]
\item Breytunum í $\cL$ hefur verið raðað í ákveðna röð
  \[ \bz_0, \bz_1, \bz_2, \dotsc\]
  Úthlutum breytunni $\bz_i$ tölunni $2i$; köllum $2i$
  \emphs{tákntölu} breytunnar $\bz_i$ og skrifum
 
  \[\tau(\bz_i) = 2i\]
\item Sérhverju öðru táknu $\bv$ í $\cL$ hefur verið úthlutað
  tiltekinni oddatölu $\tau(\bv)$ þ.a. ólík tákn fái ólíkar tölur.
  Köllum $\tau(\bv)$ tákntölu táknsins $\bv$.
\item Eftirfarandi föll eru rakin:
  \begin{itemize}
  \item \[\phi(x,n): \leftrightarrow x \text{ er tákntala n-stæðs fallatákns  í } \cL\]
  \item \[\omega(x,n): \leftrightarrow x \text{ er tákntala n-stæðs umsagnartákns í } \cL\]
  \end{itemize}

  \begin{ath}
   T.d. má setja 
   \[ \tau{\lnot} := 3, \tau(v) := 5, \tau(\forall) := 7 \]
   Ef við getum raðað fyrir hvert n n-stæða fallatáknunum í röði
   \[\mb{f}_0^n, \mb{f}_1^n, \dotsc \]
   (endanlega eða óendanlega), þá má setja
   \[ \tau(\mb{f}_k^n) := 1 + 8\cdot 2^n 3^k \]
   Eins ef raða má n-stæðu umsagnartáknum í röð
   \[\mb{p}_0^n, \mb{p}_1^n, \dotsc \]
   þá má setja
   \[ \tau(\mb{p}_k^n) :=3 + 8\cdot 2^n 3^k \]

   Næst úthlutum við sérhverju heiti eða
   yrðingu $\bu$ úr $\cL$ náttúrulegri tölu
   sem við skrifum:
   \[ \god{\bu}, \]
   sem við köllum \emphs{Gödel-tölu} $\bu$
   með þrepun.
  \end{ath}
\end{enumerate}


\begin{setn}[Gödel-tölusetning]
  \begin{enumerate}[(1)]
  \item  Ef $\bx$ er breyta, sem við lítum á sem heiti, þ.e.
    runu af lengf 1, þá er Gödel-talan
    \[ \god{\bx} =  \braket{\tau(\bx)} \]
  \item Ef $\mb{f}$ er n-stætt fallatákn og
    $\baan$ eru heiti, þá er
    \[\god{\mb{f} \baan} = \braket{\tau(\mb{f}), \god{\ba_1}, \dotsc, \god{\ba_n}}\]
  \item Ef $\mb{p}$ er n-stætt umsagnartákn, og $\baan$ eru heiti, þá er
    \[\god{\mb{p} \baan} = \braket{\tau(\mb{p}), \god{\ba_1}, \dotsc, \god{\ba_n}}\]
  \item 
    Ef $\bA$ er yrðing, þá er
    \[\god{\lnot \bA} = \braket{\tau(\lnot),\god{\bA}}\]
  \item Ef $\bA$ og $\bB$ eru yrðingar, þá er
    \[\god{\vee \bA \bB} = \braket{\tau(\vee),\god{\bA},\god{\bB}}\]
  \item 
    Ef $\bA$ er yrðing, þá er
    \[\god{\forall \bx \bA} = \braket{\tau(\forall),\god{\bx},\god{\bA}}\]
  \end{enumerate}
  Þarsem föllin $\braket{\aan}$ eru reiknanleg getum við reiknað
  $\god{\mb{n}}$ ef tákntölur allra tákna eru gefnar. Getum ákvarðað
  hvort gefin tala $a$ er Gödel-talal heitis eða yrðingar. Þetta má sjá með
  þrepun yfir $a$.
  
  Athugum fyrst hvort a er runu tala $\neq \braket{}$; það getum við, því að
  $Seq$ er rakið og því reiknanlegt.

  Ef a er ekki runutala, þá er a ekki Gödel-tala heitis né yrðingar.  Ef a er
  runutala, þá
  finnum við tölurnar $a_0, \aan$ þ.a. $a = \braket{a_0, \aan}$
  sem við getum af því að $lh$ og $ a \vdash (a)$; [hérna gæti staðið 
  $a \to (a)$, sé það ekki.] eru reiknanleg.
  Athugum hvort eitt af eftirfarandi gildir:
  \begin{enumerate}[(1)]
  \item  $\ba_0$ er jöfn tala.
  \item $\ba_0$ er tákntala n-stæðs fallatákns og $\aan$
    Gödel-tölur heita; þetta er unnt skv. þf, því að
    $a_T < a$ og fallið $\phi(x,n)$ er reiknanlegt.
  \item $a_0$ er tákntala n-stæðs umsagnar tákns og
    $\aan$ eru Gödel-tölur heita
  \item $a_0 = \tau(\lnot)$, $n = 1$ og $a_n$ er Gödel-tala yrðingar.
  \item $a_0 = \tau(\vee)$, $n=2$ og $a_1, a_2$ eru Gödel-tölur yrðinga.

  \item $a_0 = \tau(\forall)$, $n = 2$, $a_1$ er jöfn tala og 
    $a_2$ er Gödel-tala yrðingar.

    
  \end{enumerate}
    Ef einu af þessum skilyrðum er fullnægt, þá er a Gödel-tala heitis eða
    yrðingar, annars ekki.
\end{setn}

\begin{skgr}
 
  Látum $\cT$ vera kenningu fyrstu stéttar á máli $\cL$. Táknum með
 \[\Thm_{\cT}\]
 megi allra Gödel-talna allara setninga í $\cT$.
 Segjum að $\cT$ sé \emphs{ákvarðanleg} ef $\Thm_{\cT}$
 er rakið.
\end{skgr}

Skilgreinum nokkur venzl og föll; af skilgreiningunum sést
að þau eru rakin.
\begin{enumerate}[(1)]
\item \[Var(a) \leftrightarrow a = \braket{(a)_0} \wedge \exists y_{y \leq a} ((a)_0 = 2y)\]
  \emph{Ath}. $Var(a)$ þýðir að $a = \god{\bx}$ fyrir einhverja
  \emph{breytu} $\bx$
\item \[Term(a) \leftrightarrow a = Var(a) \vee [Seq(a) \wedge \phi((a)_0, lh(a) \dda 1) \wedge \forall u_{u < lh(a) \dda 1} Term((a)_{n+1}) ]\]
  \emph{Ath}. $Term(a)$ þýðir annaðhvort að  $a = \god{\bx}$ fyrir einhverja
  \emph{breytu} $\bx$, eða að $a$ sé runu tala $a = \braket{x_0, \xxn}$
  þar sem $x_0$ er tákntala n-stæðs fallatákns og $\xxn$ eru Göde-tölur heita;
  m.ö.o. $Term(a)$ þýðir að til sé $\ba$ þ.a. $a = \god{\ba}$. Term er rakið,
  því að í skilgr. koma aðeins fyrir $Term(b)$ með $b<a$.
\item  \[Afor(a) \leftrightarrow Seq(a) \wedge \omega((a)_0, lh(a) \dda 1) \wedge \forall u_{u < lh(a) \dda 1} Term((a)_{n+1}) \]
  \emph{Ath}. $Afor(a)$ þýðir að $a$ er Gödel-tala grunnyrðingar.

\item 
  \begin{gather*}
    For(a) \leftrightarrow Afor(a) \vee [Seq(a) \wedge lh(a) = 2 \wedge (a)_0 = \tau(\lnot) \wedge For((a)_1) ]\\
    \vee [Seq(a) \wedge lh(a) = 3 \wedge (a)_0 = \tau(\vee) \wedge For((a)_1) \wedge For((a)_2) ]\\
    \vee [Seq(a) \wedge lh(a) = 3 \wedge (a)_0 = \tau(\forall) \wedge Var((a)_1) \wedge For((a)_2) ]\\
  \end{gather*}
  \emph{Ath}. $For(a)$ þýðir að að sé gödel tala yrðingar.
\item 

\[ Sub(a,b,c) \bcondef
c & \Ef Var(a) = b\\
\braket{(a)_0, Sub((a)_1,b,c)} & \Ef a = \braket{(a)_0,(a)_1}\\
\braket{(a)_0, Sub((a)_1,b,c),Sub((a)_2,b,c)} & \Ef a = \braket{(a)_0,(a)_1,(a)_2} \Og (a)_0 \neq \tau(\forall)\\
\braket{(a)_0,(a)_1,Sub((a)_2,b,c)} & \Ef a = \braket{\tau(\forall),(a)_1,(a)_2} \Og (a)_1 \neq b\\
a & \text{annars}
 \econdef \]

\emph{Ath}. 
$Sub(\god{\ba},\god{\bx},\god{\bb}) = \god{\ba_{\bX} [\bb]}$
og
$Sub(\god{\bA},\god{\bx},\god{\ba}) = \god{\bA_{\bX} [\ba]}$
$\bx$ er bretya, $\ba, \bb$ eru heiti og $\bA$ yrðing.

\item 
\[Fr(a,b) \leftrightarrow \bcondef a = b & \Ef Var(a),\\
Fr((a)_1,b) & \Ef a = \braket{(a)_0, (a)_1},\\
Fr((a)_1,b) \vee Fr((a)_2,b) & \Ef a = \braket{(a)_0, (a)_1,(a)_2} \wedge (a)_0 \neq \tau(\forall),\\
Fr((a)_2,b) \wedge (a)_1 = b & \text{annars}
\econdef
\]

\emph{Ath.} $Fr(\god{\bA},\god{\bx})$ þýðir að $\bx$ komi fyrir frjálst í $\bA$.

\item 

\[
Subtl(a,b,c) \leftrightarrow \bcondef
Subtl((a)_1,b,c) & \Ef a = \braket{(a)_0, (a)_1}\\
Subtl((a)_1,b,c) \wedge Subtl((a)_2,b,c) & \Ef a = \braket{(a)_0, (a)_1, (a)_2} \\
& \wedge (a)_0 \neq \tau(\forall)\\
Subtl((a)_2,b,c) \wedge  \lnot Fr((a)_2,b) \vee \lnot Fr(c,(a)_1) & \Ef a = \braket{\tau(\forall), (a)_1, (a)_2}\\
& \wedge (a)_1 \neq b,\\
0 = 0 & \text{annars}
\econdef
\]

\emph{Ath}. $Subtl(\god{\bA}, \god{\bx}, \god{\ba})$ þýði að
$\bA$ sé innsetjanlegt fyrir $\bx$ í $\bA$.

\item 

\[
Ax_1(a) \leftrightarrow \exists x_{x < a} (For(x) \wedge a = \braket{\tau(\vee), \braket{\tau(\lnot),\braket{\tau(\vee),x,x}},x})
\]

\emph{Ath}. $Ax_1(a)$ þýðir að $a$ sé Gödel-tala frumsendu af gerðinni
\[ \bA \vee \bA \to \bA\] í pólskum rithætti. $\vee \lnot \vee \bA \bA \bA$

\item 
\[
Ax_2(a) \leftrightarrow \exists x_{x < a} \exists y_{y < a} (For(x) \wedge For(y) \wedge
 a = \braket{\tau(\vee), \braket{\tau(\lnot),x} \braket{\tau(\vee),y,x}})
\]

\item 
  \begin{gather*}
    Ax_3(a) \leftrightarrow \exists x_{x < a} \exists y_{y < a} \exists z_{z < a} (For(x) \wedge For(y) \wedge For(z)\\
    \wedge a = \braket{\tau(\vee), \braket{\tau(\lnot),\braket{\tau(\vee), \braket{\tau(\lnot),x},y}},\\
      \braket{\tau(\vee), \braket{\tau(\lnot), \braket{\tau(\vee),z,x}}, \braket{\tau(\vee),y,z}}})
\end{gather*}

\begin{ath}
  Frumsenda af gerð \textbf{F3} er svona í Pólskum rithætti:
   \[ \vee \lnot \vee \lnot \bA \bB \vee \lnot \vee \bC \bA \vee \bB \bC \]
\end{ath}


\item 
  \begin{gather*}
    Ax_4(a) \leftrightarrow \exists x_{x < a} \exists y_{y < a} \exists z_{z < a}\\
    ( Var(x) \wedge For(y) \wedge Term(z) \wedge Subtl(y,x,z) \wedge\\
    a = \braket{\tau(\vee),\braket{\tau(\lnot), \braket{ \tau(\forall),x,y}}, Sub(y,x,z)}
    )
\end{gather*}

\item 
  \begin{gather*}
    Ax_5(a) \leftrightarrow \exists x_{x < a} \exists y_{y < a} \exists z_{z < a}\\
    ( Var(x) \wedge For(y) \wedge For(z) \wedge \lnot Fr(y,x) 
    \wedge a = \braket{\tau(\vee), \braket{\tau(\lnot),\\
        \braket{\tau(\forall),x, \braket{\tau(\vee),y,z}}}, \braket{\tau(\vee),y, \braket{\tau(\forall),x,z}}}).
\end{gather*}

\item  
\[ Eq_1(a) \leftrightarrow \exists_{x < a} (Var(x) \wedge a = \braket{\tau(=),x,x} \]

\item 
  \begin{gather*}
    Eq_5(a) \leftrightarrow \exists x_{x < a} \exists y_{y < a} \exists z_{z < a} \exists z_{u < a}\\
    (Var(x) \wedge Var(y) \wedge Var(u) \wedge Var(z) \wedge Afor(u) \\
    \wedge a = \braket{\tau(\vee),\braket{\tau(\lnot),\braket{\tau(=),x,y}},\\
      \braket{\tau(\vee), \braket{\tau(\lnot),Sub(u,z,x),Sub(u,z,y)}}}).
  \end{gather*}



\item 

\[LAx(a) \leftrightarrow Ax_1(a) \vee Ax_2(a)  Ax_3(a) \vee Ax_4(a) \vee Ax_5(a) \vee Eq_1(a) \vee Eq_2(a) \]

\begin{ath}
  $LAx(a)$ þýðir að a er Gödel-tala rökfrumsendu eða samsendarfrumsendu.

\end{ath}



\begin{skgr}
  Táknum með $PAx_{\cT}$ mengi allra Gödel-talna allra
  \emphs{eiginlegra} frumsenda kenningarinnar $\cT$.
  Við segjum að $\cT$ sé \emphs{frumsenduð (á rakinn hátt)}
  ef $PAx_{\cT}$ er rakið hlutmengi í $\N$
\end{skgr}

\begin{ath}
  Endanlega frumsenduð kenning er frumsenduð. Sér í 
  lagi er $\cN$ frumsenduð.
\end{ath}


\item 

\[Ax(a) \leftrightarrow LAx(a) \vee PAx(a) \]

\item 
\[Mp(a,b,c) \leftrightarrow (b)_0 = \tau(\vee) \wedge ((b)_1)_0 = \tau(\lnot) \wedge ((b)_1)_1 = a \wedge (b)_2 = c\]

[Athugum að hérna erum við í grunninn að decode-a runu röktákna
yfir í fall, þ.e. Mp(a,b,c) er runa tákna í pólskum þ.e.
$(b)_0 = \vee$, etc., held ég.]
\begin{ath}

$Mp(\god{\bA}, \god{\bB},\god{\bC})$ þýðir að $\bC$ er afleiðing af
$\bA$ og $\bB$ skv. \textbf{MP}

\end{ath}

\item 

\[Gen(a,b) \leftrightarrow \exists x_{x < b} (Var(x) \wedge (b)_0 = \tau(\forall) \wedge (b)_1 = x \wedge (b)_2 = a \]

\begin{ath}
  $Gen(\god{\bA},\god{\bB})$ þýðir að $\bB$ er afleiðing af $\bA$
skv. \textbf{ALM}.

úthlutum runu $\bA_1, \dotsc,\bA_n$ af yrðingum runutölunni
$\braket{\god{\bA_1}, \dotsc,\god{\bA_n}}$.

\end{ath}

\item 
  \begin{gather*}
    Prf_{\cT} (a) \leftrightarrow Seq(a) \wedge lh(a) \neq 0 \\
    \wedge \forall i_{i < lh(a)} [ (Ax(a)_i) \vee \exists j_{j < i} \exists k_{k < i}\\
    (Mp((a)_j,(a)_k,(a_i)) \vee Gen((a)_j,(a)_i)) \wedge For((a_i))]
  \end{gather*}

\item  
\[\cP r_{\cT} (a,b) \leftrightarrow Prf_{\cT}(b) \wedge a = (b)_{lh \dda 1} \]
 
\begin{ath}
  $\cP r_{\cT}(\god{\bA},b)$ þýðir að $b$ sé Gödel-tala sönnunar á $\bA$ í $\cT$.
\end{ath}

Höfum þá:

\begin{setn}
  \[Thm_{\cT} (a) \leftrightarrow \exists x Pr_{\cT} (a,x) \]
\end{setn}

\begin{skgr}
  Venzl $P$ kallast \emphs{uppteljanleg á rakinn hátt} ef til eru
  talna venzl Q þ.a.
  \[ P(\xxn) \leftrightarrow \exists y Q(\xxn) \]
\end{skgr}

\begin{setn}[Fylgisetning]
  Ef $\cT$ er frumsenduð kenning, þá er mengið
  $Thm_{\cT}$ uppteljanlegt á rakinn hátt.
\end{setn}

Fyrir kenningu sem er útvíkunn á 
$\cN$ skilgreinum við sérstaklega:

\item
  \begin{gather*}
    Num(0) = \braket{\tau(0)}\\
    Num(a+1) = \braket{\tau(s), Num(a)}
  \end{gather*}

  \begin{ath}
    \[Num(a) = \god{\bk_a} \]
  \end{ath}

\end{enumerate}


\section{Ófullkomleikasetningar}


\begin{setn}[Kyrrapunktssetning]
  Látum $\cT$ vera útvíkkun á
  $\cN$ og $\bB$ vera yrðingu sem hefur nákvæmlega
  eina frjálsa breytu $\bx$. Þá er til lokuð yrðing
  $\bC$ þannig að

  \[\vdash_{\cT} \bC \leftrightarrow \bB_{\bx} [\bk_{\god{\bC}}] \]

  Setningin segir að við getum fundið
  $\bC$ þannig að $\bB_{\bx}[\bk_{\bC}] = \bC$ s.s.
  Kyrrapunkt.
  
\end{setn}


\begin{proof}
  Setjum
  \[D(u) := Sub(u, \god{\bx},Num(u)). \]
  Ef $\bA$ er yrðing, þá er
  \[D(\bA) := \god{\bA_{\bx}[\bk_{\god{\bA}}]}. \]

  Nú er $D$ rakið fall og hefur framsetningu $\bD$
  ásamt $\bx,\by$ í $\cN$ og
  þá líka $\cT$. Látum $\mb{E}$ vera
  \[ \forall \by (\bD \rightarrow \bB_{\bx}[\by]). \]

  Setjum $m:= [\mb{E}]$ og $\bC$ vera $\mb{E}_{\bx}[\bk_m]$. Setjum $q:= \god{\bC}$.
  Þá er
  \[ D(m) = D(\god{\mb{E}}) = \god{\bE_x[\bk_{\god{\bE}}]} = \god{\bC} = q .\]
  og
  \[ \vdash_{\cT} \bD_{\bx}[\bk_m] \leftrightarrow \by = \bk_q \]

  Viljum sýna að $\vdash_{\cT} \bC \leftrightarrow \bB_{\bx} [\bk_{q}] $
  
  Ef $\vdash_{\cT}$, þá

  \[\vdash_{\cT} \forall \by ( \by = \bk_q \rightarrow \bB_{\bx}[\by])\]

  og því $\vdash_{\cT} \bB_{\bx}[\bk_q]$ skv. afleiðslusetningu.

  Ef $\vdash_{\cT} \bB_{\bx}[\bk_q]$ þá gildir
  
  \[\vdash_{\cT} \forall \by ( \by = \bk_q \rightarrow \bB_{\bx}[\by])\]
  og því
  \[ \forall \by ( \bD_{x}[\bk_m] \rightarrow \bB_x[y]) \]
  þ.e. $\vdash_{\cT} \bC$ skv. jafngildissetningu.

\end{proof}


\begin{skgr}
  Útvíkkun $\cT$ kenningarinnar $\cN$
  kallast $\omega$-samkvæm ef fyrir sérhverja yrðingu 
  $\bA$ gildir:
  
  Ef $\vdash_{\cT} \lnot \bA_{\bx}[\bk_n]$ fyrir öll $n$ úr $\N$
  þá gildir ekki $\vdash_{\cT} \exists \bx \bA$.
\end{skgr}

\begin{ath}
  \begin{enumerate}[(1)]
  \item Ef $\N$ ($\cN$) er útvíkkun fyrir $\cT$, þá er $\cT$ samkvæm.

  \item Ef $\cT$ er $\omega$-samkvæm, þá er hún samkvæm.
  \end{enumerate}
\end{ath}

Ef $\cT$ er útvíkkun á $\cN$, $\cT$ frumsendanleg, þá eru venzlin
$Pr_{\cT}$ rakin og því framsetjanleg.
Látum $\mb{P}$ asamt $\bx, \by$ vera framsetningu á 
$Pr_{\cT}$ og látum $\bB$ vera $\forall y \lnot \mb{P}$.
Skv. kyrrapunktssetningu er til lokuð yrðing $\mb{G}$ þ.a.
\[\vdash_{\cT} \mb{G} \leftrightarrow \forall \by \lnot \mb{P}_{\bx}[\bk_{\god{\mb{G}}}]\]
Yrðingin $\mb{G}$ segir að ekki er til tala y sem er
Gödel-tala sönnunarinnar í $\cT$ á yrðingunni $\mb{G}$.
Þ.e. $\mb{G}$ er í $\cT$  jafngild þeirri fullyrðingu að $\mb{G}$
sé ekki sannanleg í $\cT$. Köllum $\bG$ Gödelyrðingu $\cT$.

\begin{setn}[Ófullkomleikasetning Gödels.]
  .
  \begin{enumerate}
  \item Ef $\cT$ er samkvæm, þá er ekki $\vdash_{\cT} \bG$.
  \item Ef $\cT$ er $\omega$-samkvæm, þá er ekki $\vdash_{\cT} \lnot \bG$.
  \end{enumerate}
\end{setn}

\begin{proof}
  Setjum $q := \god{\bG}$.
  \begin{enumerate}[(1)]
  \item G.r.f. $\vdash_{\cT} \bG$. Látum $r$ vera Gödel-tölu sönnunar á
    $\bG$ í $\cT$. Þá er $Pr_{\cT}(q,r)$ og því
    $\vdash_{\cT} \mb{P} [\bk_q, \bk_r]$ og því
    $\vdash_{\cT} \exists y \mb{P}_x [\bk_q]$ þ.e.
    $\vdash_{\cT} \lnot \forall y \lnot \mb{P}_x[\bk_q]$ Þar með er
    $\vdash_{\cT} \bG$ í mótsögn við að $\cT$ er samkvæm.


  \item G.r.f að $\cT$ sé $\omega$-samkvæm og  $\vdash_{\cT} \lnot \bG$. Þar sem.
    $\vdash_{\cT} \bG \leftrightarrow \forall \by \lnot \mb{P}_{\bx}[\bk_q]$ fæst
    $\vdash_{\cT} \exists \by \mb{P}_x[\bk_q]$.
    En $\cT$ er samkvæm svo að ekki gildir $\vdash_{\cT} \bG$, svo að
    $\lnot Pr_{\cT}(q,n)$ f. öll $n$ og því
    $\vdash_{\cT} \lnot \mb{P}_{\bx,\by}[\bk_q,\bk_n]$ f. öll $n$.
    Vegna $\omega$-samkvæmi er $\vdash \exists \by \mb{P}_{\bx}[\bk_q]$, sem er mótsögn.

  \end{enumerate}
\end{proof}

Þetta er lygara þversögnin. Hún segir að hún se ekki sannanleg.
Ef hún er ekki sannanleg, þá er hún ekki sannanleg og því sannanleg, en
ef hún er sannanleg þá er hún ekki sannanleg etc.


%2014-11-20


\begin{setn}[Fylgisetning]
Frumsenduð $\omega$-samkvæm útvíkkun á $\cN$ er ekki fullkomin.
\end{setn}

\begin{skgr}
  Köllum kenninguna $\cR$ með mál $\cL(\cN)$ þ.a. setningarnar í $\cR$
  séu nákvæmlega þær yrðingar í $\cL(\cN)$ sem er sannar í $\N$
  \emphs{sannan reikning}.
\end{skgr}

\begin{skgr}
  Segjum að kenning sé \emphs{frumsendanleg} ef til er frumsenduð kenning sem
  er jafngild henni.

\end{skgr}

\begin{setn}[Fylgisetning]
 Sannur reikningur er ekki frumsendanlegur. 
\end{setn}

\begin{setn}[Hjálparsetning]
Ef $\cT$ er samkvæm útvíkkun á $\cN$, þá er mengið $Thm_{\cT}$ (mengi allra Gödel-talna allra setninga í $\cT$)
ekki framsetjanlegt í $\cT$.
\end{setn}

\begin{proof}
  G.r.f. að $\bT$ ásamt $\bx$ sé framsetning á $Thm_{\cT}$ í $\cT$.
  Látum $\bC$ vera yrðingu þ.a.
  \[\vdash_{\cT} \bC \leftrightarrow \lnot \bT_{\bx}[\bk_{\god{\bC}}]. \]
  Setjum $q:= \god{\bC}$. Nú er $\vdash_{\cT}$ annars væri
  $q \not\in Thm_{\cT}$ og því $\vT \lnot \bT_{\bx}[\bk_q]$ af því að
  $\bT$ ásamt $\bx$ er framsetning á $\thT$ í $\cT$,
  og þá fæst $\vT \bC$. Þetta er í mótsögn við að
  $\cT$ er samkvæm.

\end{proof}
Sér í lagi sést að $\thT$ er ekki rakið mengi.

\begin{setn}[Church] Ef $\cT$ er samkvæm útvíkkun kenningarinnar $\cN$,
  þá er $\cT$ ekki úrskurðanleg.
\end{setn}
\begin{skgr}
  Kenning er fullkomin ef við getum sannað annað hvort
 $A$ eða $\lnot A$.
\end{skgr}

\begin{skgr}
  Að kenning $\cT$ sé \emphs{úrskurðanleg} þýðir að
  Fyrir sérhverja yrðingu getum við gengið úr skugga um hvort
  hún sé setning eða ekki í endanlega mörgum skrefum.
  
  Þ.e. $\Thm_{\cT}$ er rakið mengi.

  Að kenning sé \emphs{Frumsendanleg} segir að við getum gengið úr skugga um í endanlega
  mörgurm skrefum hvort e-ð sé frumsenda eða ekki.

\end{skgr}
\begin{setn}
  Sannur reikningur er ekki úrskurðanlegur.
\end{setn}

Önnur afleiðing af HS:

\begin{setn}[Óskilgreinanleikasetning Tarsky]
  $Thm_{\cR}$ er ekki framsetjanlegt í $\cR$.
  Þetta þýðir að ekki er til yrðing $\bA$ þ.a.
  fyrir sérhverja yrðingu $\bB$ gildi:
  \[ \models_{\cN} \bB \text{ þþaa } \vR \bA [\bk_{\god{\bB}}] \]
\end{setn}

þ.e. ``sannleika í reikningi er ekki unnt að skilgreina í reikningi.''

Skyld setning, stundum kölluð Gödel-Tarsky setning:
\begin{setn}[Gödel-Tarsky]
 Í samkvæmri útvíkkun $\cT$ á $\cN$ er ekki til yrðing $\bT$
 með nákvæmlega einni frjálsri breytu $\bx$ þ.a. fyrir allar
 lokaðar yrðingar $\bA$ gildi:
 \[ \vT \bA \leftrightarrow \bT_{\bx}[\bk_{\god{\bA}}]\]


 því að skv. kyrrapunktssetningunni er til lokuð yrðing $\bA$ þ.a.
 \[\vT \bA \leftrightarrow \lnot \bT_{\bx}[\bk_{\god{\bA}}]\]
 og þá fengist $\bA \leftrightarrow \lnot \bA$
 
\end{setn}

\begin{innsk}
  
 [$\bT_{\bx}[\bk_{\god{\bA}}]$ þýðir að $\bT$ gildi um Gödel-tölu $\bA$.]

 [$\leftrightarrow$ er lesið ``jafngilt'' ]

 [Í $\cN$ $\bk_n$ er skammstöfun fyri nafnið á tölunni  $SS..SS0$, þar sem $S$ kemur $n$ sinnum fyrir.
 Þetta er s.s. bara nafn fyrir náttúrulegu töluna n.]

 [$\cN$ er kenning um náttúrulegar tölur. Mjög einföld, notum ekki Peano-reikning.]

 [$\bT_{\bx}[a]$ segir að við skiptum út $a$ fyrir $\bx$ í yrðingunni $\bx$.]

 Gödel yrðingin, $\bP$ er yrðing sem segir hvort e-ð sé setning:
 \[ \vT \bG \leftrightarrow \forall \by \lnot \bP_{\bx} [\bk_{\god{\bG}}] \]
 Segir að ekki er til $y$ þ.a. $x$ sé sönnun á yrðingunni $\bG$.
 S.s. setning sem segir að ekki sé til sönnun á henni.

 Gödel-talan er reiknanleg.

 %$\bP_{\bx}[\bk_n]$ segir að $n$ sé Gödel-tala yrðingar $\by$

 $Pr_{\cT}(x,y)$ segir að $y$ sé gödel tala sönnunar á $x$.
\end{innsk}


Af Church-setningu leiðir að  hrein rökfræði er ekki úrskurðanleg.

\begin{setn}
  Látum $\cT$ vera  vera hreina rökfræði (bara rökfrumsendur)
  á máli $\cL(\cN)$ þ.e.
  $\cT$ er kenning fyrstu stétta á málinu $\cL(\cN)$ sem hefur engar
  eiginlegar frumsendur. Þá er $\cT$ ekki úrskurðanleg.
\end{setn}

\begin{innsk}
  Getum það þó í Yrðingarökfræði. Við getum úrskurðað hvort e-ð sé
  sísanna.
  Getum það ekki þegar við bætum rökfræðinni við.
  Ef við höfum yrðingu sem við vitum ekki hvort að sé setning eða
  ekki.  Ef við getum sannað hana, þá vitum við að hún er setning, en
  ef við getum ekki sannað hana, þá vitum við það ekki.
\end{innsk}

\begin{proof}
  Látum $\mb{N}_k$ vera lokun frumsendunnar
  $Nk$ í $\cN$, $k = 1, \dotsc, g$ og $\bC$ vera
  $\N_1 \wedge \dotsb \wedge N_g$.
  Þá hefur $\cN$ sömu setningar og $T[\bC]$.


  Yrðinng $\bA$ er skv. afleiðslusetningu setning í
  $T[\bC]$ þþaa $\vT \bC \to \bA$.
  Setjum $q := \god{\bC}$ og skilgreinum rakið fall $f$ með:

  \[ f(a) := \braket{\tau{\vee},\braket{\tau{\lnot},q},a} \]

  þá er 
  \[ Thm_{T[\bC]} (a) \leftrightarrow For(a) \wedge \thT(f(a)) \]
  Ef $\thT$ væri rakið, þá væri $Thm_{T[\bC]}$ rakið og því
  $\cN$ úrskurðanlegt í mótsögn við Church.

\end{proof}

\begin{skgr}
  Skgr. á \emphs{úrskurðanleg} frá Reyni.
  $Thm_{\cT}$ er mengi allra Gödel-talna allra setninga í $\cT$
  $\cT$ er \emphs{úrskurðanleg} þþaa mengið $Thm_{\cT}$ sé rakið.
  Getum s.s. úrskurðað hvort gefin yrðing sé setning eða ekki.
\end{skgr}


\begin{setn}
  Venzl $P$ eru rakin þþaa  $P$ og $\lnot P$ séu uppteljanleg á
  rakinn hátt.
\end{setn}

\begin{proof}
  Ef venzlin $P$ eru rakin, þá  eru venzlin $P$ og $\lnot P$ bæði rakin
  og þá eru þau uppteljanleg á rakinn hátt. Gerum þá ráð fyrir að
  $P$ og $\lnot P$ séu bæði uppteljanleg. Þá eru til rakin venzl $Q$ og $R$ þ.a.
  \begin{gather*}
    P(\aan) \leftrightarrow \exists Q(\aan,x)\\
    \lnot P(\aan) \leftrightarrow \exists R(\aan,x)
  \end{gather*}
  Fyrir $(\aan)$ er $P(\aan) \vee \lnot P(\aan)$
  svo til er $x$ þ.a. $Q(\aan,x)\vee R(\aan,x)$

  Þá má skilgreina rakið fall $F$ með
  \begin{gather*}
    F(\aan) := \mu x(Q(\aan,x)\vee R(\aan,x))
  \end{gather*}
  en þá er
  \[ P(\aan) \leftrightarrow Q(\aan,F(\aan))\]

  Ef $ Q(\aan,F(\aan))$ þá $\exists Q(\aan,x)$
  
  og þá $P(\aan)$; en ef $\lnot Q(\aan,F(\aan))$

  þá $R(\aan,F(\aan))$ svo að $\exists x R(\aan,x)$

  og þá $\lnot P(\aan)$
\end{proof}

\begin{setn}
  Ef $\cT$ er frumsenduð og fullkomin útvíkkun á $\cN$, þá er $\cT$ úrskurðanleg.
\end{setn}

\begin{proof}
  Skilgreinum rakin föll $F$ og $K$ með
  \begin{gather*}
    F(0,a) = a\\
    F(n+1,a) = \braket{\tau(\forall),\god{\bz_n},F(n,a)}\\
    K(a) = F(a+1, a)
  \end{gather*}

  Ef $a = \god{\bA}$, þá er
  \[K(a) = \god{\forall \bz_a \forall\bz_{a-1} \dotsb \forall \bz_0 A }\]
  Ef $\bz_i$ kemur fyrir í $\bA$, þá er
  \[ i < \god{\bz_i} < \god{A} = a \]
  svo að $\forall \bz_a \forall\bz_{a-1} \dotsb \forall \bz_0 A $ er 
  lokuð yrðing, Skv. alhæfingu er hún setning þþaa $\bA$ sé setning.
  Nú er $\cT$ fullkomin, svo að $\bA$ er \emph{ekki} setning
  þþaa $ \vT \lnot \forall \bz_a \forall\bz_{a-1} \dotsb \forall \bz_0 A $
  þess vegna gildir:
  \[ \lnot Thm_{\cT}(a) \leftrightarrow \lnot For(a) \vee Thm_{\cT} (\braket{ \tau(\lnot), K(a)}) \]
  þ.e 
  \[ \lnot Thm_{\cT}(a) \leftrightarrow \exists y( \lnot For(a) \vee Pr_{\cT} (\braket{ \tau(\lnot), K(a)},y)). \]
  svo að $\lnot Thm_{\cT} (a)$ er uppteljanlegt á rakinn hátt og
  $Thm_{\cT} (a)$ er það líka; skv. setn. er $Thm_{\cT}$ rakið.
\end{proof}

\begin{setn}[Gödel-Rosser]
Ef $\cT$ er frumsendanleg og samkvæm útvíkkun á $\cN$, þá er $\cT$ ekki fullkomin.
\end{setn}

\begin{proof}
  Ef hún væri fullkomin, þá væri hún skv. síðustu setningu úrskurðanleg í mótsögn
 við setningu Church.
\end{proof}

\begin{ath}
  Höfum ekki beint á eina setningu sem er hvort sannanleg né afsannanleg.
  En það má gera: Látum $\bE$ vera
  \[\forall \by (\bP \rightarrow \bz (\mb{N} \rightarrow \exists \bu (\bu < \by \wedge \bP_{\bx,\by}[\bz,\bu])))\]
  þar sem $\bP$ ásamt $\bx,\by$ er framsetning í $\cN$ á $Pr_{\cT}$

  [$Pr_{\cT}(a,n)$ segir að $a$ sé gödel tala sönnunar á yrðingunni $n$.]

  og $\mb{N}$  ásamt $\bx, \by$ er framsetning fallsins
  \[ Neg(a) = \braket{\tau(\lnot),a}. \]

  Skv. kyrrapunktssetningu er til lokuð yrðing $\bR$ þ.a.
  \[ \vT \bR \leftrightarrow \bE_x [\bk_{\god{\bR}}] \]
  sem við köllum \emphs{Rosser-yrðingu} fyrir $\cT$.
  Í venjulegri túlkun segir hún:
  
  Ef $\bR$  hefur sönnun með Gödel-tölu $y$, þá hefur 
  $\lnot \bR$  sönnun með Gödel-tölu minni en $y$. 
  Sýna má að hvorki er unnt að sanna eða afsanna $\bR$ í $\cT$.
\end{ath}

[Ath $\cP\cA$ stendur fyrir Peano reikning.]

Látum $\bP$ ásamt $\bx,\by$ vera framsetningu á 
$Pr_{\cP\cA}$ í $\cN$ og þá líka á $\PA$.
 og látum $\bT$ vera
 \[ \exists \by \bP \]

Við höfum $\vdash_{\PA} \bP_{\bx,\by} [\bk_a, \bk_b]$ ef $\cP r_{\PA}(a,b)$ og
$\vdash_{\PA} \lnot \bP_{\bx,\by} [\bk_a, \bk_b]$ ef $\cP r_{\PA}(a,b)$.
Skrifum til þæginda $\pg{\bA}$ í stað $\bk_{\god{\bA}}$

[$\pg{\bA}$ er Gödeltala $\bA$ skrifuð  í $\PA$] 

\begin{setn}[Hilbert-Bernays-Löb-afleiðsluskilyrði]
  Höfum:
\begin{enumerate}[label=\textbf{HBL\arabic*}]
  \item Ef $\vdash_{\PA} \bA$, þá $\vP \bT_{x} [\pg{\bA}]$.
  \item $\vP \bT_x [\pg{\bA \to \bB}] \to (\bT_x[\pg{\bA}] \to \bT_x[\pg{\bB}])$.
  
  \item $\vP \bT_{\bx}[\pg{\bA}] \to \bT_{\bx}[ \pg{\bT_x[\pg{\bA}]}]$.
  \end{enumerate}
\end{setn}

\begin{ath}
  Flestar bækur sanna ekki þessa setningu, en sleppa sönnuninni. Síðan hefur
  reynst erfitt að finna sönnun á þessari setningu í neinni bók, og ein bók
 segir ``erfitt er að finna sönnun á þessu, fremur en upprunalegu sönnunina''.
\end{ath}

\begin{proof}
  Sönnum \textbf{HBL1}, \textbf{HBL2}, en ekki \textbf{HBL3}

  Fyrir \textbf{HBL1} athugum við:  Ef $\vP \bA$, þá er
  til sönnun á $\bA$ í $\PA$. Látum $m$ vera Gödel-tölu
  þeirrar sönnunar. Þá $\cP r_{\PA}(\god{\bA},m)$ og því
  $\vP \bP_{\bx,\by} [\bk_{\god{\bA}},\bk_m]$
  En þá fæst $\vP \exists \by \bP_{\bx}[\bk_{\god{A}}]$, en þetta er akkúrat
  $\bT[\pg{\bA}]$

 [T af gödel-tölunni $\bA$, skrifaðri í $\PA$]
 [HBL2 er e-rskonar modus ponens]

 Fyrir \textbf{HBL2}. Það má skrifa
 \[\vP \bT_x[\pg{\bA \to \bB}] \wedge \bT_{\bx}[\pg{\bA}] \to \bT_{\bx}[\pg{\bB}]\]


 Gefum okkur að $\vP \bT_x[\pg{\bA \to \bB}]$ og
 $\vP \bT_x[\pg{\bA}]$. Það þýðir að til eru sannanir á 
 $\bA \to \bB$ og $\bA$ í $\PA$;
 Látum $c,a$ vera Gödel-tölu þeirra sannana;
 $c$ er Gödel-tala sönnunar $\bC_1, \dotsc, \bC_n$ á
 $\bA \to \bB$ og $a$ er Gödel-tala sönnunar
 $\bA_1, \dotsc, \bA_m$ á $\bA$. Þá er
 $\bC_1, \dotsc, \bC_n,\bA_1, \dotsc, \bA_m, \bB$ sönnun á $\bB$ í $\PA$
 svo að $c * a * \braket{\god{\bB}}$ er Gödel-tala sönnunar á $\bB$ í $\PA$,
 Þar sem 
 \begin{gather*}
   a*b := \mu x (lh(x) = lh(a) + lh(b) \wedge\\
   \forall i_{i < lh(a)} ((x_i) = (a_i)) \wedge i_{j< lh(b)} ((x)_{lh(a)+j} = (b)_i))
 \end{gather*}

þannig að 

\[\braket{\aan} *  \braket{b_1, \dotsc, b_n} = \braket{\aan, b_1, \dotsc,b_n}.\]

Svo að $(a,b) \mapsto a*b$ er rakið. Höfum
\[ \cP r_{\PA} (\god{\bB}, c * a * \braket{\god{\bB}})\]
svo að $\vP \exists \by \bP_{\bx}[\bk_{\god{\bB}}]$, þ.e. $\vP \bT_{\bx} [\pg{\bB}]$.
\end{proof}


Höfum aftur:

Látum $\bP$ ásamt $\bx,\by$ vera framsetningu á 
$Pr_{\cP\cA}$ í $\cN$ og þá líka á $\PA$.
 og látum $\bT$ vera
 \[ \exists \by \bP \]

Skrifum til þæginda $\pg{\bA}$ í stað $\bk_{\god{\bA}}$

\begin{setn}[Hilbert-Bernays-Löb-afleiðsluskilyrði]
  Höfum:
\begin{enumerate}[label=\textbf{HBL\arabic*}]
  \item Ef $\vdash_{\PA} \bA$, þá $\vP \bT_{x} [\pg{\bA}]$.
  \item $\vP \bT_x [\pg{\bA \to \bB}] \to (\bT_x[\pg{\bA}] \to \bT_x[\pg{\bB}])$.
  
  \item $\vP \bT_{\bx}[\pg{\bA}] \to \bT_{\bx}[ \pg{\bT_x[\pg{\bA}]}]$.
  \end{enumerate}
\end{setn}

\begin{setn}[Löb]
  Ef $ \vP \bT_{\bx}[\pg{\bA}] \to \bA$, þá $\vP \bA$
  
(þ.e. ef $\bA$ er afleiðing af því að hægt sé að sanna $\bA$, þá gildir $\bA$).
\end{setn}

\begin{proof}
Skv. kyrrapunktssetningu er til lokuð yrðing $\bL$ þ.a.

\begin{enumerate}[(1)]
\item  $\vP \bL \leftrightarrow (\bT_{\bx} [\pg{\bL}] \to \bA).$

þá er
\item $\vP \bL \to (\bT_{\bx}[\pg{\bL} \to \bA)$.

Skv. \emph{HBL1} er þá
\item $\vP \bT_{\bx}[\pg{\bL \to (\bT_{\bx}[\pg{\bL}]\to \bA)}]$

og skv. \emph{HBL2} er
\item $\vP \bT_{\bx}[\bL] \to \bT_{\bx}[\pg{\bT_{\bx} [\pg{\bL}] \to \bA}]$

Með \emph{HBL2} er
\item $\vP \bT_{\bx}[\pg{\bT_{\bx}[\pg{\bL}] \to \bA}] \to (\bT_{\bx}[\pg{\bTx[\pg{\bL}]}] \to \bTx[\pg{\bA}])$

af (4) og (5) fæst með gegnvirkni að

\item  $\vP \bTx[\pg{\bL}] \to (\bTx[\pg{\bTx[\pg{\bL}]}] \to \bTx[\pg{\bA}])$

Skv. \emph{HBL3}
\item 
 $\vP \bTx [\pg{\bL}] \to \bTx [\pg{\bTx[\pg{\bL}]}]$

af (6) og (7) leiðir

\item $\vP \bTx[\pg{\bL}] \to \bTx[\pg{\bA}]$

Forsenda setningarinnar er

\item $\vP \bTx[\pg{\bA}] \to \bA$

af (8) og (9) fæst
\item $\vP \bTx [\pg{\bL}] \to \bA$

Af (1) og (10) leiðir
\item $\vP \bL$

með \emph{HBL1} fæst
\item 
 $\vP \bTx[\pg{\bL}]$
\end{enumerate}
Af (10) og (12) fæst:
$\vP \bA$ með \textbf{MP}
\end{proof}

\begin{ath}
  Þetta minnir á eftirfarandi röksemdarfærslu:
  Löb-setningin segir eitthvað á þá leið:
  ``Ef $\bA$ er sannanlegt þá A'' leiðir til $\bA$.

  Látum $\bS$ vera

  ``Ef þessi setning er sönn, þá er jólasveinninn til''.

  m.ö.o er S

  ``Ef S er sönn, þá er jólasveinninn til''.

  Gefum okkur nú sem forsendu:

  \begin{enumerate}[(1)]
  \item  S er sönn

    Þetta þýðir 


  \item ``Ef S er sönn, þá er jólasveinninn til'' er sönn.

    Af (2) leiðir:

  \item Ef S er sönn, þá er jólasveininn til.

    Af (1) og (2) leiðir
  \item  Jólasveinninn er til

   Hef sýnt að  (1) leiðir til (4), þ.e.

 \item Ef S er sönn, þá er jólasveinninn til.

   Úr því það sannað, þá er það satt, svo að
 \item  ``Ef S er sönn, þá er jólasveinninn til'' er sönn

   Því gildir
 \item S er sönn.

 Af (5) og (7) fæst með \textbf{MP}:
 \item  Jólasveininn er til.
  \end{enumerate}
\end{ath}



\section{Seinni ófullkomleikasetning Gödels}

[$\vP$ þýðir eiginlega að hægt sé að sanna í $\PA$]


\begin{setn}[Seinni ófullkomleikasetning Gödels]
  Að því gefnu að $\PA$ sé samkvæm, þá er ekki

  \[\vP \lnot \bTx[\pg{0 = S0}]\]
\end{setn}
\begin{proof}
G.r.f. að við getum sannað það, þ.e.
$\vP \lnot \bTx[\pg{0 = S0}]$

Þá fæst $\vP \bTx[\pg{0 = S0}] \to 0 = S0$. [Því að þá er $\bTx[\pg{0 = S0}]$,
og af rangri yrðing leiðir hvað sem er].

Skv. setningu Löbs er þá $\vP 0 = S0$. En 
nú er $\vP 0 \neq S0$, svo að þetta er í mótsögn
við að $\PA$ sé samkvæm.
  
\end{proof}

\begin{ath}
  Líta má svo á að yrðingin $\lnot \bTx[\pg{0 = S0}]$ segir (í $\PA$) að $\PA$ sé samkvæm.
  [Þetta er þýðing yfir í $\PA$ að $\PA$ sé samkvæm]. Seinni Gödel-setningin
 segir þá að ef $\PA$ er samkvæm, þá getum við ekki sannað í $\PA$ að
 $\PA$ sé samkvæm.
\end{ath}

Getum búið til yrðinguna \textbf{Con}

\[ \forall \bx \forall \by \forall \bz \forall \bt \lnot (\bP \wedge \bP_{\bx,\by}[\bz,\bt] \wedge \bN_{\bx,\by}[\by,\bt]), \]

$\by$ sö á $\bx$, $\bt$ sö á $\bz$
%[Þetta þýðir að $\bP$ þýðir að $\by$ er gödel-tala sönnunar á $\bx$]

$\bN$ ásamt $\bx,\by$ er framsetning á  
\[ Neg(a) = \braket{\tau(\lnot),a} \]

og $P$ er framsetning á Pr.

þetta segir að ekki sé til sönnun í $\PA$ á yrðingu og neitun hennar. Þá er ljóst að
\[ \vP \mathbf{Con} \to \lnot \bTx[\pg{0 = S0}] \]

Svo af því leiðir að ekki gildir $\vP \mathbf{Con}$.






\end{document}
