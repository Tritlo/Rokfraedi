\documentclass[12pt]{book}

% not needed with polyglossia
\usepackage[utf8]{inputenc}
\usepackage[T1]{fontenc}

%\usepackage{polyglossia}
%\setdefaultlanguage{icelandic}


\usepackage{graphics,amsmath,amsfonts,amsbsy,amssymb,amsthm}
\usepackage{fancyvrb}
\usepackage[a4paper]{geometry}
\usepackage{graphicx}
\usepackage{hyperref}
\usepackage{datatool}
\usepackage{float}
\usepackage{mdframed}
\usepackage{listingsutf8}
\usepackage{enumerate}
\usepackage{comment}
\usepackage{epstopdf}
\usepackage{caption}
\usepackage{subcaption}
\usepackage{tikz}
\usepackage[shortlabels]{enumitem}
\usepackage{mathtools}
\usepackage{tabu}

\usepackage{accents}

\setlength{\parskip}{8pt plus 1pt minus 1pt}
%Verdur ad vera her, sumir pakkar dependa a thetta.
\usepackage[icelandic]{babel}

%viljum ekki númeraða kafla á dæmum

\newcommand{\nonums}{\setcounter{secnumdepth}{-1}}

%flýtiskipanir
\newcommand{\e}{\textbf}
\newcommand{\R}{\mathbb{R}}

\newcommand{\X}{\mathbb{X}}
\newcommand{\Y}{\mathbb{Y}}


%\newcommand{\R}{\Real}
%\newcommand{\C}{\Complex}
%\newcommand{\Z}{\Integer}
%\newcommand{\N}{\Natural}
%\newcommand{\Q}{\Rational}

\newcommand{\K}{\mathbb{K}}
\newcommand{\C}{\mathbb{C}}
\newcommand{\Con}{\mathcal{C}}
\newcommand{\Z}{\mathbb{Z}}
\newcommand{\N}{\mathbb{N}}
\newcommand{\Q}{\mathbb{Q}}
\newcommand{\f}{\frac}
\newcommand{\1}{\frac{1}}
\newcommand{\eps}{\f{\epsilon}}
\newcommand{\Lra}{\Leftrightarrow}
\newcommand{\Th}{\text{ þegar }}
\newcommand{\Ef}{\text{ ef }}
\newcommand{\Og}{\text{ og }}


\newcommand{\inner}[1]{\accentset{\circ}{#1}}
\newcommand{\eR}{\widetilde{\R}}


\newcommand{\com}[1]{\set{\text{#1}}}
\newcommand{\Com}[1]{\set{\text{Athsmd: \text{#1}}}}

\newcommand{\ub}[2]{\underbrace{#1}_{\text{#2}}}
\newcommand{\ubt}[2]{$\ub{\text{#1}}{#2}$}


\newenvironment{inum}{\begin{enumerate}[label=(\roman*).]}{\end{enumerate}}
\newenvironment{anum}{\begin{enumerate}[label=(\alph*).]}{\end{enumerate}}


\newcommand{\bcondef}{\left\{ \begin{array}{l l}}
\newcommand{\econdef}{\end{array} \right.}
\DeclarePairedDelimiter{\condef}{\bcondef}{\econdef}

\DeclarePairedDelimiter{\ceil}{\lceil}{\rceil}
\DeclarePairedDelimiter{\floor}{\lfloor}{\rfloor}
\DeclarePairedDelimiter{\set}{\{}{\}}
\DeclarePairedDelimiter{\braket}{\langle}{\rangle}


\newenvironment{lausn}{\begin{proof}[Lausn]}{\end{proof}}

\newcommand{\sep}{\;|\;}

\newcommand{\fig}[2]{
\begin{figure}[H]
  \centering
  \includegraphics{#1}
  \caption{#2}
  \label{fig:#1}
\end{figure}
}


\newtheorem*{setn}{Setning}
\newtheorem*{hsetn}{Hjálparsetning}
\lstset{  literate={á}{{\'a}}1
                  {ó}{{\'o}}1
                  {ú}{{\'u}}1
                  {ð}{{\dh}}1
                  {í}{{\'i}}1
                  {é}{{\'e}}1
                  {ö}{{\"o}}1
                  {þ}{{\th}}1
                  {æ}{{\ae}}1
                  {ý}{{\'y}}1
                  {Á}{{\'A}}1
                  {Ó}{{\'O}}1
                  {Ú}{{\'U}}1
                  {Ð}{{\DH}}1
                  {Í}{{\'I}}1
                  {É}{{\'E}}1
                  {Ö}{{\"O}}1
                  {Þ}{{\TH}}1
                  {Æ}{{\AE}}1
                  {Ý}{{\'Y}}1}


\theoremstyle{definition}
\newtheorem*{skgr}{Skilgreining}
\newtheorem*{daemi}{Dæmi}
\newtheorem*{frumsenda}{Frumsenda}

\theoremstyle{remark}
\newtheorem*{ath}{Athugasemd}
\newtheorem*{innsk}{Innskot}


\newcommand{\cT}{\mathcal{T}}
\newcommand{\cL}{\mathcal{L}}
\newcommand{\cN}{\mathcal{N}}
\newcommand{\mb}[1]{\mathbf{#1}}
\newcommand{\mc}[1]{\mathcal{#1}}
\newcommand{\bA}{\mathbf{A}}
\newcommand{\ba}{\mathbf{a}}
\newcommand{\bB}{\mathbf{B}}
\newcommand{\bS}{\mathbf{S}}
\newcommand{\bb}{\mathbf{b}}
\newcommand{\bC}{\mathbf{C}}
\newcommand{\bG}{\mathbf{G}}
\newcommand{\vT}{\vdash_{\cT}}
\newcommand{\bE}{\mathbf{E}}
\newcommand{\bF}{\mathbf{F}}
\newcommand{\bP}{\mathbf{P}}
\newcommand{\bT}{\mathbf{T}}
\newcommand{\bD}{\mathbf{D}}
\newcommand{\bH}{\mathbf{H}}
\newcommand{\bc}{\mathbf{c}}
\newcommand{\bX}{\mathbf{X}}
\newcommand{\bx}{\mathbf{x}}
\newcommand{\bz}{\mathbf{z}}
\newcommand{\bk}{\mathbf{k}}
\newcommand{\by}{\mathbf{y}}
\newcommand{\xxn}{x_1, \dotsc, x_n}
\newcommand{\bxxn}{\bx_1, \dotsc, \bx_n}
\newcommand{\aan}{a_1, \dotsc, a_n}
\newcommand{\baan}{\ba_1, \dotsc, \ba_n}
\DeclarePairedDelimiter{\god}{\ulcorner}{\urcorner}
\nonums
\title{Rökfræði}
\author{Matthías Páll Gissurarson}

\begin{document}
\maketitle


\chapter{2014-11-05}

\section{37}

\begin{daemi}
  Sýnið að mengi allra rakinna falla er teljanlegt, en að upptalningin
  $f_0,f_1,f_2, \dotsc$ á einstæðum röknum föllum er rekki rakin.

  \begin{lausn}
    Látum $R_0$ vera mengi þeirra rakinna falla sem fást með $R1$.
    Þ.e.as.
    \[ R_0 = \set{I_i^n | n \in \N^{*} \Og \i \in \set{1, \dotsc, n}} \cup \set{Z,N} \]
    Þetta mengi er teljanlegt.

    Látum $k \geq 1$ vera náttúrulega tölu.
    \begin{enumerate}[(i)]
    \item  Látum $R_{3k-2}$ vera mengið af þeim röktu föllum sem fást með því að 
      beita $R2$ á föllin í $R_{3k-3}$, ásamt öllum föllum í
      $R_{3k-3}$.

      \emph{R2} Segir: Ef $G,H_1, \dotsc, H_n$ eru rakin föll,
      á er $G(H_1(\xxn), \dotsc, H_n(\xxn))$ rakið fall.

      Fyrir sérhvert endanlegt hlutmengi í $R_{3k-3}$
      fáum við endanlega mörg ný föll í $R_{3k-2}$.
      Ef $R_{3k-3}$ er teljanlegt eru þessi hlutmengi teljanlega mörg,
      svo $R_{3k-2}$ er líka teljanlegt.
    \item Skgr. $R_{3k-1}$ á sama hátt með $R_3$. Það verður líka
      teljanlegt ef $R_{3k-2}$ er teljanlegt.
    \item Skgr. $R_{3k}$ á sama hátt með $R4$....
    \end{enumerate}
    Það er því ljóst að $R_n$ er teljanlegt f. öll $n \in \N$
    og þar með er mengi allra rakinna falla
    \[ R = \bigcup_{n \in \N} R_n \] 
    einnig teljanlegt.


    Látum nú $f_0, f_1, \dotsc$ vera upptaliningu á öllum einstæðum röktum föllum.
    og $F: \N^2 \to \N$ vera fallið þ.a.
    \[ F(x,y) = f_x(y) \] fyrir öll $x,y \in \N$

    Fallið $f: \N \to \N $,
    \[ f(x) = F(x,x) + 1 \]
    er rakið ef $F$ er rakið.

    En fyrir $x \in \N$ er
    \[ f_x(x) < f_x(x)+1 = F(x,x) +1 = f(x) \]
    svo $F$ er ekkert fallana $f_0, f_1, \dotsc$
    og það með ekki rakið.
  \end{lausn}
\end{daemi}

\section{38}
Sýnið að eftirfarandi föll og venzl eru rakin:

\begin{enumerate}[(a)]
\item 
  \begin{gather*}
    F(n) := \lfloor \sqrt{n} \rfloor \Og G(n) := \lfloor ne \rfloor\\
    F(n) := \mu y_{y < n+1} (sg (n+1 - y \cdot y) = 0) -1
  \end{gather*}
  Sannað með upptalningu:
  $F(N)$ er $F(0), \dotsc, F(9) = 0,1,1,1,2,2,2,3$
  $H(N)$ er $H(0), \dotsc, H(9) = 0,1,1,1,2,2,2,3$

\item $\pi(n)$ = fjöldi prímtala $\leq n$
\[\pi(n) = \sum_{y \leq n} C_{\lnot Pr}(y)\]
\end{enumerate}

\section{39}

\begin{daemi}
  \begin{gather*}
    F(\xxn,y+1) = N(\xxn,y,F(\xxn,y),G(\xxn,y)\\
    G(\xxn,y+1) = P(\xxn,y,F(\xxn,y),G(\xxn,y)\\
  \end{gather*}
  \begin{lausn}
  Skrifum $x$ í stað $(\xxn)$ og getum þá skrifað skilgreiningarnar á $F,G$
  þannig:
  \begin{gather*}
    F(x,0) = L(x),\\
    G(x,0) = M(x),\\
    F(x,y+1) = N(x,y,F(x,y),G(x,y))\\
    G(x,y+1) = P(x,y,F(x,y),G(x,y))
  \end{gather*}
  Viljum sjá að $F,G$ rakin ef $L,M,N,P$ eru rakin.
  Skilgreinum $H$ með:
  \[H(x,y) = 2^{F(x,y)} 3^{G(x,y)} \]
  þá er 
  \[ H(x,y+1) = 2^{N(x,y,v_0(H(x,y)),v_1(H(x,y))} 3^{P(x,y,v_0(H(x,y)),v_1(H(x,y))} \]
  svo að $H$ er [frumstætt] rakið og $F(x,y) = v_0(H(x,y)), G(x,y) = v_1(H(x,y))$;
  $v_k$ rakni skv. fyrirlestri.
  \end{lausn}

\end{daemi}

\section{40}




\chapter{2014-11-19 Dæmablað 12}

\section{45}

\begin{daemi}
  Skilgreinum tvístætt fall $f: \N^2 \to \N$ með því að láta $f(x,y)$
  ver töluna í menginu $\set{0,\dotsc,9}$ sem er táknuð með þeim
  tölustaf sem er y sætum til vinstri frá aftasta staf þegar við
  skrifum $x$ í tugakerfi þannig er t.d.
  \begin{gather*}
    f(6382,0) = 2,\\
    f(6382,1) = 8,\\
    f(6382,3) = 3,\\
    \vdots
  \end{gather*}
\end{daemi}

\begin{lausn}[Áslaug]
  Sýnum fyrst að tvístæða fallið $g: \N^2 \to \N$
  $(x,y) \mapsto x^y$ sé frumstætt rakið:
  \[g(x,0) = 1, g(x,y+1) = g(x,y)x.\]
  þá er $f(x,y) = rm(qt(x,10^y),10)$
  frumstætt rakið.
\end{lausn}

\section{46}




\section{47}

\begin{daemi}
  Látum $\cL$ vera mál kenningarinnar $\cN$ og
látum $\cT(\N)$ sem hefur sem seningar nkvl. þær yrðingar sem eru sannar
í $\N$. Látum $Q$ vera mengi af frumtölum. Sýnið að til er líkan $M$ f.
kenninguna $\cT(\N)$ ásamt staki $a$ úr $M$ þ.a. frumtala $p$
gangi upp í $a$ í líkaninu þþaa $p \in Q$
\end{daemi}

\begin{lausn}[Tandri]
  Látum $\cL^*$ vera málið sem fæst með því að bæta við málið
  $\cL$ fasta $\ba$ Fyrir náttúrulega tölu $n$ látum við
  $\bA_n$ vera 
  \[ \exists x ( x \cdot S \dotsc S 0= \ba) \]
  Þar sem $S$ kemur fyrir $n$ sinnum.

  Látum $\cT^+$ vera kenninguna á málinu $\cL^*$ sem fæst með því að bæta við
  $\cT(\N)$ frumsenum $\bA_p$ fyrir sérhvert $p$ úr Q og frumsendunni
  $\lnot \bA_p$ fyrir sérhverja frumtölu $p \not \in Q$. 
  Ef $\cT^*$ er endanlega frumsendaður hluti kenningarinnar $\cT^*$,
  þá hefu hann endanlega margar frumsendur af gerðinni $\bA_p$.
  Segjum
  $\bA_{p_1}, \dotsc, \bA_{p_n}$ en þá er $\N$ líkan fyrir
  $\cT^*$ með $(\ba)_{\N} = \prod_{j=1}^k p_j$.

  Skv. þjöppunarsetn hefur $\cT^*$ því líkan $M$, sem er þá sér í lagi líkan
  fyrir $\cT(\N)$. Það er þá ljóst að frumtala $p$ gengur upp í $ a := (\ba)_n$
  þþaa $p \in Q$.
\end{lausn}




\section{48}

\begin{daemi}
  Látum $\cL$ vera málið sem hefur jafnaðarmerki, fastana $'0'$ og $'1'$,
  tvístæðu falltáknin $'+'$ og $'-'$ og tvístæða umsagnartáknið $'<'$;
  og bætum auk þess við einstæðu falltákni $'f'$. Við lítum á rauntölurnar $\R$ 
 sem mynztur fyrir þetta mál með eðlilegum hætti, þar sem túlkun falltáknsins 
 $'f'$ er gefin með einhverju gefnu falli $f:\R \to  \R$
 sem er þannig að $f(0) = 0$. Bætum nú við málið föstum fyrir allar rauntölur og látum
 $\cT$ vera mengi allra yrðinga á nýja málinu sem eru sannar 
 í líkaninu. Bætum við nýjum fasta og frumsendum sem segja að nýi fastinn sé 
 stærri en allar rauntölur og fáum þannig líkt og í fyrirlestrum líkan fyrir nýju
 kenninguna sem er óarkimedísktsvið ${}^*\R$; látum ${}^*f$
 vera túlkun falltáknsins $'f'$ í þessu nýja líkani. Segjum að stak $x \in \R$
 sé \emph{óendanlega lítið} ef $-y < x < y$fyrir sérhvert jákvætt stak í $\R$.

 Sýnið: Fallið $f:\R \to \R$ er samfellt í punktinum $0$ 
 þá og því aðeins að stærðin ${}^*f(x)$ sé óendanlega lítil fyrir
 sérhverja óendanlega litla stærð $x$.
\end{daemi}

\begin{lausn}[Áslaug]
G.r.f. að $f$ sé samfellt í $0$. G.r.f að til
sé óendanlega lítil stærð $x$ þ.a. ${}^*f(x)$ sé ekki óendanlega líti. Þá er til

tala $\epsilon > 0, \epsilon \in \R$ þ.a. $|{}^*f(x)| > \epsilon$. Þá er ekki


til $\delta > 0, \delta \in \R$ þ.a. $|{}^*f(y)| < \epsilon$ fyrir öll
$|y| < \delta$, því $|x| < \delta$. Því er $f$ ekki samfellt.
  
\end{lausn}

\chapter{Dæmablað 13}

\section{52}

[Reynir hefur ekki leyst það almennilega ennþá]

\begin{daemi}
  \begin{enumerate}[(a)]
  \item 
    \begin{gather*}
      \text{Fullyrðingin hér að neðan er ósönn}\\
      \text{Fullyrðingin hér að ofan er sönn}
    \end{gather*}
    Hvað er hæft í þessu?
  \item Látum $\cT$ vera útvíkkun á $\cN$, $\bB$ og $\bC$ vera tvær yrðingar á
    máli $\cT$. Sem hafa hvor um sig nákvæmlega  eina frjálsa breytu $\bx$.
    Til eru lokaðar yrðingar $\bE$ og $\bT$ þ.a.
    \[ \vT \bE \leftrightarrow \bB_{\bx}[\bk_{\god{\bF}}] \text{ og } \vT \bF \leftrightarrow \bC_{\bx}[\bk_{\god{\bE}}]\]
  \end{enumerate}
\end{daemi}

\begin{proof}
  Setjum 
  \[ D(u,v) = sub(sub(v,\god{\bx},Num(v)), \god{\bz},Num(u)) \]
  Þar sem $\bz$ er ný breyta sem kemur hvergi fyrir í $\bB$ og $\bC$.
  Ef $\bS$ og $\bT$ eru yrðingar er
  \[ D(\god{\bS},\god{\bT}) = sub(\bT_{\bx}[\bk_{\god{\bT}}], \god{\bz},\bk_{\god{\bS}}) = \god{\bT_{\bx,\bz}[\bk_{\god{\bT}}, \bk_{\god{\bS}}] } .\]
 Þar sem $D$ er rakið fall, hefur það framsetningu $\bD$ ásamt $\bx, \bz$ og $\by$ í $\cT$.
 Það þýðir að fyrir $n,m \in \N$ er
 \[ \vT \bD_{\bx,\by}[\bk_n, \bk_m] \leftrightarrow \by = \bk_l \]
 þar sem $l = D(n,m)$.
 
 Látum $\bG$ vera yrðinguna $\forall \by (\bD \to \bB_{\bx}[\by])$,
 og $\bH$ vera $\forall \by ( \bD \to \bC_{\bx}[\by] ) $
 og látum $m = \god{\bG}$ og $n = \god{\bH}$.

 Látum einnig $\bE$ vera $\bG_{\bx, \bz}[\bk_m, \bk_n]$ og $\bF$ vera
 $\bH_{\bx,\bz}[\bk_n, \bk_m]$. Setjum $p:= \god{\bE}$ og $q:= \god{\bF}$.

 Þá fæst:
 \[ D(m,n) = D(\god{\bG}, \god{\bH}) = \god{\bH_{\bx,\bz} [\bk_n,\bk_m]} = \god{\bF} = q \]
 og $D(n,m) = p$,
 svo

 \begin{gather*}
   \vT \bD_{\bx,\bz}[\bk_m,\bk_n] \leftrightarrow \by = \bk_q \text{ og } \\
   \vT \bD_{\bx,\bz}[\bk_n,\bk_m] \leftrightarrow \by = \bk_p
 \end{gather*}

 Fáum nú
 \begin{align*}
   & \vT \bE & \\
 \text{ \textbf{þþaa} } & \bG_{\bx,\bz}[\bk_m,\bk_n] & \\
\text{ \textbf{þþaa} } & \vT \forall \by (\bD_{\bx,\bz}[\bk_m, \bk_n] \to \bB_{\bx}[\by]) & \\
\text{ \textbf{þþaa} } & \vT \forall (\by = \bk_q \to \bB_{\bx}[\by]  & \\
\text{ \textbf{þþaa} } & \vT \bB_{\bx}[\bk_q]  & \text{ skv. afleiðslusetn. } \\
\text{ \textbf{þþaa} } & \vT \bB_{\bx}[\bk_{\god{\bF}}] & \\
 \end{align*}
svo $\vT \bE \leftrightarrow \bB_{\bx}[\bk_{\god{\bF}}]$.
Á sama hátt fæst
$\vT \bF \leftrightarrow \bC_{\bx}[\bk_{\god{\bE}}]$.

(c) Látum $\cT$ vera frumsendanlega útvíkkun á $\cN$, þá eru venslin $Pr_{\cT}$ rakin
og því framsetjanleg. ($Pr_{\cT}(a,b)$ þþaa  $b$ sé Gödel-tala sönnunar á yrðingu sem hefur Gödel-tölu $a$).

Látum $\bP$  ásamt $\bx,\by$ vera framsetningu á $Pr_{\cT}$,
látum $\bB$ $\forall \by \lnot \bP$ og $\bC$ vera $\exists \by \bP$.

Skv. (b)-lið eru til lokaðar yrðingar $\bE$ og $\bF$ þ.a.
\[ \vT \bE \leftrightarrow \forall \by \lnot \bP_{\bx}[\bk_{\god{\bF}}]\]
\[ \vT \bF \leftrightarrow \exists \by \bP_{\bx}[\bk_{\god{\bE}}] \]

\begin{ath}
  Að $\bE$ er jafngilt þeirri fullyrðingu að $\bF$ sé ekki sannanleg
  í $\cT$ og $\bF$ jafngilt fulllyrðingunni að $\bE$ sé sannanlegt í $\cT$
  
\end{ath}


Við getum sýnt:

\begin{enumerate}[(1)]
\item  Ef $\cT$ er samvkæm, þá er $\vT \bE$
\item  Ef $\cT$ er samvkæm, þá er hvorki $\vT \lnot \bE$, né $\vT \bF$.
\end{enumerate}

Gerum það í fjórum skrefum.

Látum fyrst 
$p := \god{\bE}, q := \god{\bF}$

\begin{enumerate}[(i)]
\item  (Ef $\vT \bE$, þá $\vT \bF$).
G.r.f. að $\vT \bE$ og látum r vera Gödel-tölu sönnunar á $\bE$ í $\bT$.
Þá er $Pr_{\cT}(p,r,)$, svo
$\vT \bP_{\bx,\by}[\bk_p, \bk_r]$ og því $\vT \exists \bP_{\bx}[\bk_p]$
og þá $\vT \bF$.

\item  (Ef $\vT \bF$, þá $\vT \lnot \bE$).

G.r.f. að $\vT \bF$, og látum s vera Gödel-tölu sönnunar á $\bF$ í
$\cT$. Þá er $Pr_{\cT}(q,s)$, svo $\vT \bP_{\bx,\by}[\bk_q,\bk_s$, og því
$\vT \exists \by \bP_{\bx}[\bk_q]$ þ.e.a.s
$\vT \lnot \forall \by \lnot \bP_{\bx} [\bk_q]$, svo $\vT \lnot \bE$.

Af (i) og (ii) leiðir (1).

\item (Ef $\cT$ $\omega$-skv. og $\vT \lnot \bE$, þá ekki $\vT  \bF$)
G.r.f. að $\cT$ sé $\omega$-skv. og $\vT \lnot \bE$.
Þá gildir ekki $\vT \bE$, því $\cT$ er sér í lagi samkvæm.

En þá er $\lnot Pr_{\cT}(p,n)$ fyrir öll $n$ úr $\N$ og því
$\vT \lnot \bP_{\bx,\by}[\bk_p,\bk_n]$ fyrir öll $n$ úr $\N$.
Því fæst að ekki gildir $\vT \exists \by P_{\bx}[\bk_p]$,
þar sem $\cT$ er $\omega$-samvkvæm.

En þá gildir ekki $\vT \bF$.
\item (Ef $\cT$ er $\omega$-samkvæm. og ekki $\vT \bF$, þá er ekki $\vT \lnot \bE$)
G.r.f. að $\cT$ sé $\omega$-samvkæm og ekki gildi $\vT \bF$.
Þá er $\lnot Pr_{\cT}(q,n)$ fyrir öll $n$ úr $\N$,
og því $\vT \lnot \bP_{\bx,\by}[\bk_q,\bk_y]$ fyrir öll $n$ úr $\N$.

Því fæst með $\omega$-samvkæmni að ekki gildi
\[ \vT \exists \by \bP_{\bx}[\bk_q] \]
þ.e.a.s að ekki gildi 
\[\vT  \lnot \forall \by \lnot \bP_{\bx}[\bk_q], \]

en þá gildir ekki $\vT \lnot \bE$
Fáum nú (2):

G.r.f að $\cT$ sé $\omega$-samkvæm:
\begin{itemize}
\item  Ef $\vT \lnot \bE$, þá er ekki $\vT \bF$ skv. (iii) 
  og þá ekki $\vT \lnot \bE$ skv. (iv). Mótsögn.
\item Ef $\vT \bF$, þá er $\vT \lnot \bE$ skv. (iii) og þá
ekki $\vT \bF$ skv. (iii). Mótsögn.
\end{itemize}
\end{enumerate}

\emph{Niðurstaða}: $\omega$-samkvæm, frumsendanleg útvíkkun á $\cN$ er ekki fullkomin.

\end{proof}



\end{document}