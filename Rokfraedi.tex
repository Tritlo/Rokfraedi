\documentclass[12pt]{book}

% not needed with polyglossia
\usepackage[utf8]{inputenc}
\usepackage[T1]{fontenc}

%\usepackage{polyglossia}
%\setdefaultlanguage{icelandic}


\usepackage{graphics,amsmath,amsfonts,amsbsy,amssymb,amsthm}
\usepackage{fancyvrb}
\usepackage[a4paper]{geometry}
\usepackage{graphicx}
\usepackage{hyperref}
\usepackage{datatool}
\usepackage{float}
\usepackage{mdframed}
\usepackage{listingsutf8}
\usepackage{enumerate}
\usepackage{comment}
\usepackage{epstopdf}
\usepackage{caption}
\usepackage{subcaption}
\usepackage{tikz}
\usepackage{enumitem}
\usepackage{mathtools}
\usepackage{tabu}

\usepackage{accents}

\setlength{\parskip}{8pt plus 1pt minus 1pt}
%Verdur ad vera her, sumir pakkar dependa a thetta.
\usepackage[icelandic]{babel}

%viljum ekki númeraða kafla á dæmum

\newcommand{\nonums}{\setcounter{secnumdepth}{-1}}

%flýtiskipanir
\newcommand{\e}{\textbf}
\newcommand{\R}{\mathbb{R}}

\newcommand{\X}{\mathbb{X}}
\newcommand{\Y}{\mathbb{Y}}


%\newcommand{\R}{\Real}
%\newcommand{\C}{\Complex}
%\newcommand{\Z}{\Integer}
%\newcommand{\N}{\Natural}
%\newcommand{\Q}{\Rational}

\newcommand{\K}{\mathbb{K}}
\newcommand{\C}{\mathbb{C}}
\newcommand{\Con}{\mathcal{C}}
\newcommand{\Z}{\mathbb{Z}}
\newcommand{\N}{\mathbb{N}}
\newcommand{\Q}{\mathbb{Q}}
\newcommand{\f}{\frac}
\newcommand{\1}{\frac{1}}
\newcommand{\eps}{\f{\epsilon}}
\newcommand{\Lra}{\Leftrightarrow}
\newcommand{\Th}{\text{ þegar }}
\newcommand{\Ef}{\text{ ef }}
\newcommand{\Og}{\text{ og }}


\newcommand{\inner}[1]{\accentset{\circ}{#1}}
\newcommand{\eR}{\widetilde{\R}}

\newcommand{\com}[1]{\set{\text{#1}}}
\newcommand{\Com}[1]{\set{\text{Athsmd: \text{#1}}}}

\newcommand{\ub}[2]{\underbrace{#1}_{\text{#2}}}
\newcommand{\ubt}[2]{$\ub{\text{#1}}{#2}$}


\newenvironment{inum}{\begin{enumerate}[label=(\roman*).]}{\end{enumerate}}
\newenvironment{anum}{\begin{enumerate}[label=(\alph*).]}{\end{enumerate}}


\newcommand{\bcondef}{\left\{ \begin{array}{l l}}
\newcommand{\econdef}{\end{array} \right.}
\DeclarePairedDelimiter{\condef}{\bcondef}{\econdef}

\DeclarePairedDelimiter{\ceil}{\lceil}{\rceil}
\DeclarePairedDelimiter{\floor}{\lfloor}{\rfloor}
\DeclarePairedDelimiter{\set}{\{}{\}}
\DeclarePairedDelimiter{\braket}{\langle}{\rangle}


\newenvironment{lausn}{\begin{proof}[Lausn]}{\end{proof}}

\newcommand{\sep}{\;|\;}

\newcommand{\fig}[2]{
\begin{figure}[H]
  \centering
  \includegraphics{#1}
  \caption{#2}
  \label{fig:#1}
\end{figure}
}


\newtheorem*{setn}{Setning}
\newtheorem*{hsetn}{Hjálparsetning}
\lstset{  literate={á}{{\'a}}1
                  {ó}{{\'o}}1
                  {ú}{{\'u}}1
                  {ð}{{\dh}}1
                  {í}{{\'i}}1
                  {é}{{\'e}}1
                  {ö}{{\"o}}1
                  {þ}{{\th}}1
                  {æ}{{\ae}}1
                  {ý}{{\'y}}1
                  {Á}{{\'A}}1
                  {Ó}{{\'O}}1
                  {Ú}{{\'U}}1
                  {Ð}{{\DH}}1
                  {Í}{{\'I}}1
                  {É}{{\'E}}1
                  {Ö}{{\"O}}1
                  {Þ}{{\TH}}1
                  {Æ}{{\AE}}1
                  {Ý}{{\'Y}}1}


\theoremstyle{definition}
\newtheorem*{skgr}{Skilgreining}
\newtheorem*{daemi}{Dæmi}
\newtheorem*{frumsenda}{Frumsenda}

\theoremstyle{remark}
\newtheorem*{ath}{Athugasemd}


\newcommand{\cT}{\mathcal{T}}
\newcommand{\cL}{\mathcal{L}}
\newcommand{\cN}{\mathcal{N}}
\newcommand{\mb}[1]{\mathbf{#1}}
\newcommand{\mc}[1]{\mathcal{#1}}
\newcommand{\bA}{\mathbf{A}}
\newcommand{\ba}{\mathbf{a}}
\newcommand{\bB}{\mathbf{B}}
\newcommand{\bb}{\mathbf{b}}
\newcommand{\bC}{\mathbf{C}}
\newcommand{\bD}{\mathbf{D}}
\newcommand{\bc}{\mathbf{c}}
\newcommand{\bX}{\mathbf{X}}
\newcommand{\bx}{\mathbf{x}}
\newcommand{\bk}{\mathbf{k}}
\newcommand{\by}{\mathbf{y}}
\newcommand{\bz}{\mathbf{z}}
\newcommand{\bu}{\mathbf{u}}
\newcommand{\bv}{\mathbf{v}}
\newcommand{\xxn}{x_1, \dotsc, x_n}
\newcommand{\bxxn}{\bx_1, \dotsc, \bx_n}
\newcommand{\aan}{a_1, \dotsc, a_n}
\newcommand{\baan}{\ba_1, \dotsc, \ba_n}
\newcommand{\bkaan}{\bk_{a_1}, \dotsc,\bk_{a_n}}
\newcommand{\dda}{\dot{-}}
\newcommand{\Thm}{{\mc{T}hm}}

\DeclarePairedDelimiter{\god}{\ulcorner}{\urcorner}

\title{Rökfræði}
\author{Matthías Páll Gissurarson}

\begin{document}
\maketitle


\chapter{1}

Sjá Rokfraedi.md fyrir byrjun.

\chapter{2}
\begin{proof}
  $\hdots$ \\
  $ \textbf{P}_{ij}$ vera yrðingabreytuna $\textbf{x}_j$ ef talan $1$ stendur í
  j\-ta dálki i\-tu línu, en látum $\textbf{P}_{ij}$ vera $\lnot x_j$ ef
  þar stendur 0.  Látum $A_{i}$ vera

  \[ \textbf{P}_{i,1} \wedge \textbf{P}_{i,2} \wedge \dotsb \wedge
  \textbf{P}_{i,n} \] fyrir $i = 1, \dotsc, 2^n$ og $A$ vera
  \[ \textbf{A}_{i_1} \vee \textbf{A}_{i_2} \vee \dotsb \vee \textbf{A}_{i_n} \] þar
  sem $i_1, \dotsc, i_r$ eru númerin á i.\\

  $\hdots$ \\
  Ef engar slíkar línur eru til, látum þá $\textbf{A}$ vera
  \[ \textbf{x}_1, \wedge \dotsb \wedge \textbf{x}_n \wedge \lnot \textbf{x}_1 \wedge
  \dotsb \wedge \lnot \textbf{x}_n \]
\end{proof}

töluðum um mál $L^{*}$ með 16 táknum fyrir öll tvístæð yrðingasnið, þar á meðal
$\vee$, $\wedge$, $\rightarrow$, $\leftrightarrow$, og líka einstæða táknið $\lnot$,



$\hdots$\\


vegna þess að $'p \vee q'$ er jafngilt $'\lnot ( \lnot p \wedge  \lnot q)'$
er $\set{'\lnot', '\wedge'}$. Mengið $\set{'\lnot'}$ getur ekki verið nægjanlegt.
En $'p \vee q'$ er jafngilt $'((\lnot p) \rightarrow q'$ svo að mengið
$\set{'\lnot', '\rightarrow'}$ er nægjanlegt. Atugum að $'\lnot p'$ er jafngilt
\[p \downarrow p\]
og
$'p \wedge q'$ er jafngilt $'(( p \downarrow p) \downarrow (q \downarrow q))'$
svo að $\set{'\downarrow'}$ er nægjanlegt. Líka er
$'\lnot p'$ jafngilt $'p | p'$.
og $'p \vee q'$ er jafngilt $'(p | p) | (q|q)'$ svo að $\set{'|'}$ er nægjanlegt.

segð $'(p \rightarrow)'$ verðr jafngild

\['(((p \downarrow  p) \downarrow (( q \downarrow q) \downarrow (q \downarrow q))) \downarrow (( p \downarrow p) \downarrow ((q \downarrow q) \downarrow (q \downarrow q))))' \]

sem í pólskum rithætti verður $ \downarrow \downarrow \downarrow  p p  \downarrow \downarrow q q  \downarrow q q  \downarrow \downarrow p p  \downarrow \downarrow q q  \downarrow q q'$

\begin{setn}
Ef \textbf{s} er tvístætt tákn í $L^{*}$ og $\set{\textbf{s}}$ er
nægjanlegt, þá er $\textbf{s}$ annaðhvort $'\downarrow'$ eða $'|'$.

\end{setn}


\section{Frumsendur fyrir yrðingarökfræði}

\begin{skgr}
  \emph{Formleg kenning $\mathcal{T}$} er gefin með:
  \begin{anum}
  \item Gefið er formlegt mál $\mathcal{L(T)}$, \emph{mál kenningarinnar}.
  \item  Gefið er tiltekið mengi $\mathcal{A}$ af segðum á $\mathcal{L(T)}$
    sem kallast \emph{frumsendur} kenningarinnar.
  \item Gefnar eru endanlega margar \emph{rökreglur}. Hver rökregla segir
    að segðir af tiltekini gerð hafi einhverja segð af tiltekkinni gerð sem \emph{afleiðingu}.
  \end{anum}
    Venjulega er krafizt að við getum gengið úr skugga um í endanlega mörgum skrefum
    hvort segð er frumsenda eða ekki. Regla sem leyfir okkur að búa til frumsendur í
    $\mathcal{T}$ kallast \emph{frumsendugrip}.

    \emph{Sönnun} í $\mathcal{T}$ er endanleg runa $\textbf{A}_1, \dotsc, \textbf{A}_n$
    af segðum á $\mathcal{L(T)}$ þ.a. fyrir sérhvert $k = 1, \dotsc, n$ sé $\textbf{A}_k$
    annaðhvort frumsenda eða afleiðing af einhverjum af segðunum
    $\textbf{A}_1, \dotsc, \textbf{A}_{k-1}$ skv. rökreglum kenningarinnar.

    Segð $\textbf{A}$ er \emph{setning} í $\mathcal{T}$ ef hún er síðasta setningin
    í sönnun; skrifum þá
    \[ \vdash_{\mathcal{T}} \textbf{A} \text{ eða } \vdash \textbf{A}\]
    skrifum
    \[ \mathcal{H} \vdash_{\mathcal{T}} \textbf{A} \]
    þar sem $\mathcal{H}$ er mengi af segðum ef
    $\textbf{A}$ er setning í $\mathcal{T[H]}$ sem fæst með því að bæta
    öllum segðunum í $\mathcal{H}$ við $\mathcal{T}$ sem nýjum frumsendum.
    Ef $\mathcal{H}$ hefur bara endanlega margar segðir,
    $\textbf{H}_1, \dotsc, \textbf{H}_n$ , þá skrifum við
    \[\textbf{H}_1, \dotsc, \textbf{H}_n \vdash_{\mathcal{T}} \textbf{A}\]
    í stað $\mathcal{H} \vdash_{\mathcal{T}} A$.
\end{skgr}

\begin{skgr}
  Skilgreinum formlega kenningu $\mathcal{P}$ fyrir yrðingarökfræði þannig:
  \begin{anum}
  \item Mál kenningarinnar er $\mathcal{L(P))}$
  \item Frumsendur kenningarinnar eru gefnar með eftirfarandi frumsendugripum:
    \begin{enumerate}[label=\textbf{F}\arabic*]
    \item \[ \textbf{A} \vee \textbf{A} \rightarrow \textbf{A} \]
    \item \[ \textbf{A} \rightarrow \textbf{B} \vee \textbf{A} \]
    \item \[ ( \textbf{A} \rightarrow \textbf{B}) \rightarrow (\textbf{C} \vee \textbf{A} \rightarrow \textbf{B} \vee \textbf{C})\]
    \end{enumerate}
  \item Kenningin hefur aðein eina rökreglu: \emph{modus ponens}, og er þannig:
    \[ \textbf{MP}: \text{ Af } \textbf{A} \rightarrow \textbf{B} \text{ og } \textbf{A} \text{ leiðir } \textbf{B}\]
    Fáum strax \emph{afleidda rökreglu}; \emph{innsetningarreglu}:

    \textbf{Inn}: Látum \textbf{A} vera yrðingarsnið, $ \textbf{x}_1, \dotsc, \textbf{x}_n$
    vera ólíkar yrðingabreytur og $ \textbf{B}_1, \dotsc, \textbf{B}_n$ vera
    yrðingasnið. Ef $\vdash \textbf{A}$, þá er $\textbf{A}_{\textbf{x}_1, \dotsc, \textbf{x}_n}[\textbf{B}_1, \dotsc, \textbf{B}_n]$
    \begin{proof}
      Ef $\textbf{A}_1, \dotsc, \textbf{A}_m$ er sönnun á $A$, þá er
      \[\textbf{A}_{ 1 \textbf{x}_1, \dotsc, \textbf{x}_n}[\textbf{B}_1, \dotsc, \textbf{B}_n], \dotsc, \textbf{A}_{m \textbf{x}_1, \dotsc, \textbf{x}_n}[\textbf{B}_1, \dotsc, \textbf{B}_n]\]
      sönnun á $\textbf{A}_{\textbf{x}_1, \dotsc, \textbf{x}_n}[\textbf{B}_1, \dotsc, \textbf{B}_n]$

    \end{proof}

  \end{anum}
\end{skgr}

\emph{Afleidd rökreglar $\textbf{R1}$}. Látum $\mathcal{H}$ vera mengi af yrðingasniðum
og $\textbf{A,B,C}$ vera yrðingasnið.

Ef $\mathcal{H} \vdash \textbf{A} \rightarrow \textbf{B}$ og $\mathcal{H} \vdash \textbf{C} \vee \textbf{A}$
þá $\mathcal{H} \vdash \textbf{B} \vee \textbf{C}$æ

\begin{proof}
  Notum $\textbf{MP}$ tvisvar á $\textbf{F3}$
\end{proof}

$\vdots$\\


\begin{setn}[Afleidd rökregla \textbf{R6}]
  Ef $\mathcal{H} \vdash \textbf{A} \rightarrow \textbf{C}$ og
  $\mathcal{H} \vdash \lnot \textbf{A} \rightarrow \textbf{C}$,
  þá $\mathcal{H} \vdash \textbf{C}$.
\end{setn}

\begin{setn}[Afleidd rökregla \textbf{R7}]
    Ef $\mathcal{H} \vdash \textbf{A} \rightarrow \textbf{B}$,
    þá $ \mathcal{H} \vdash \textbf{A} \rightarrow \textbf{C} \vee \textbf{B}$.
\end{setn}


$\vdots$\\

\begin{setn}[Fylgisetning]
  Við höfum
  \[ \mathbf{A} \vdash \mathbf{B} \text{ þþaa } \vdash \mathbf{A} \rightarrow \mathbf{B} \]

\end{setn}

\begin{setn}
  Sérhver setning i $\mathcal{P}$ er sísanna.
  \begin{proof}
    '$p \vee p \rightarrow p$, '$p \rightarrow q \vee p$' og
    '$(p \rightarrow q) \rightarrow (r \vee p \rightarrow q \vee r)$'
    eru sísönnur, svo að frumsendur er sísönnur og \textbf{MP} varðveitir sísönnur.
  \end{proof}
\end{setn}
\section{Fullkomleikasetning fyrir yrðingarökfræði}

Sérhver sísanna er setning í formlegu kenningunni $\mathcal{P}$
\begin{skgr}
  Kenning $\mathcal{T}$ er \emph{samkvæm} ef til er segð á máli $\mathcal{T}$
  sem er ekki setning í $\mathcal{T}$.
\end{skgr}

\begin{setn}
  $\mathcal{P}$ er samkvæm.
    \begin{proof}
      Til eru yrðingarsnið sem eru ekki sísönnur.
    \end{proof}
\end{setn}


$\vdots$\: \: \:hér vantar nokkra daga.


Athugum að í '$\exists y (2 \cdot y = x )$', sem er yrðing í málinu $\mathcal{N}$,
þar sem $2$ er skammstöfun fyrir $SS0$, er $y$ buyndin breyta þar sem hún kemur fyrir, en $x$
ekki; þetta er fullyrðing um $x$, en ekki um $y$. Gætum eins skrifað
'$\exists t (2 \cdot t = x )$', $\exists z (2 \cdot z = x )$, og
allar þessar fullyrðingar segja ``x er jöfn tala''. Viljum nú setja heiti inn fyrir
breytur í fullyrðingum en bara þar sem þær eru frjálsar; viljum að fullyrðingin segi
hið sama um hlutinn með heitið eins og hún segir um það sem breytan stendur
fyrir; t.d. segir '$\exists t (2 \cdot t = 2 )$' að $2$ sé jöfn tala, sem er rétt,
og '$\exists t (2 \cdot t = 3 )$', sem er rangt. En setjum nú $y+1$ inn fyrir $x$
í '$\exists y (2 \cdot y = y+1 )$' sem segir alls ekki að $y+1$
sé jöfn tala, heldur að jafnan '$2y = y+1$ hafi lausn (í $\N$), sem er rétt.

Þetta er af því að heitið inniheldur breytuna '$y$', sem verður bundin þegar
að við setjum hana inn. Þetta verður að banna!

\begin{ath}
 \textbf{x} er bundin á tilteknum stað í yrðingu \textbf{A} ef
staðurinn er í hlutfullyrðingunni í A af gerðinni
$\forall \mathbf{x} \mathbf{B}$ (eða $\exists x B$).
\end{ath}

\begin{skgr}
  \begin{enumerate}[(1)]
  \item Við segjum að heiti $\mathbf{a}$ á máli fyrstu stéttar
    $\mathcal{L}$ sé \emph{innsetjanlegt} í yrðingu \textbf{A} á
    $\mathcal{L}$ ef eftirfarandi skilyrði sé fullnægt. Fyrir breytu
    \textbf{y} aðra en \textbf{x} sem kemur fyrir í \textbf{a}
    inniheldur yrðingin \textbf{A} enga hlutyrðingu af gerðinni
    $\forall \mathbf{y} \mathbf{B}$ ( eða $\exists \mathbf{y}
    \mathbf{B}$) þannig að breytan $\mathbf{x}$ komi fyrir frjáls í
      $\mathbf{B}$.
    \item Látum \textbf{A} vera yrðingu á $\mathcal{L}$,
      $\mathbf{x}_1, \dotsc, \mathbf{x}_n$ ólíkar breytur og
      $\mathbf{a}_1, \dotsc, \mathbf{a}_n$ heiti þ.a. fyrir
      hvert $k = 1, \dotsc, n$ sé $\mathbf{a}_k$ innsetjanlegt
      fyrir $\mathbf{x}_k$ í $\mathbf{A}$.
      Táknum með $\mathbf{A}_{\mathbf{x}_1, \dotsc, \mathbf{x}_n}[\mathbf{a}_1, \dotsc, \mathbf{a}_n]$
      eða $I^{\mathbf{a}_1, \dotsc, \mathbf{a}_n}_{\mathbf{x}_1, \dotsc, \mathbf{x}_n} \mathbf{A}$
      yrðinguna sem fæst með því að setja inn $\mathbf{a}_k$ inn fyrir $\mathbf{x}_k$ í
      $\mathbf{A}$ fþar sem $\mathbf{x}_k$ kemur frjálst fyrir í $\mathbf{A}$ fyrir
      öll $k = 1, \dotsc, n$ \emph{samtímis}.

    \item Ef $\mathbf{b}$ er heiti, þá er
      \[ \mathbf{b}_{\mathbf{x}_1, \dotsc, \mathbf{x}_n} [ \mathbf{a}_1, \dotsc, \mathbf{a}_n] \]
      eða
      \[I^{\mathbf{a}_1, \dotsc, \mathbf{a}_n}_{\mathbf{x}_1, \dotsc, \mathbf{x}_n} \mathbf{b}\]
      heitið sem fæst með því að setja $\mathbf{a}_k$ inn fyrir $\mathbf{x}_k$ í $\mathbf{b}$
      fyrir $k= 1, \dotsc, n$ \emph{samtímis}.
    \end{enumerate}
\end{skgr}

\begin{ath}
  Yrðingarnar $I^{\mathbf{a}, \mathbf{b}}_{\mathbf{x}, \mathbf{y}} \mathbf{A}$,
  $I^{\mathbf{a}}_{\mathbf{x}} I^{\mathbf{b}}_{\mathbf{y}} \mathbf{A}$ og
  $I^{\mathbf{b}}_{\mathbf{y}} I^{\mathbf{a}}_{\mathbf{x}} \mathbf{A}$
  get allar verið ólíkar, þar sem við erum ekki að setja allt inn á sama tíma.
\end{ath}

Við erum alltaf að gera ráð fyrir að við séum með innsetjanlegt dót.


\begin{skgr}[\emph{Formleg kenning fyrstu stéttar}] $\mathcal{T}$ er gefin með
  eftirfarandi hættir:
  \begin{enumerate}[A.]
  \item Gefið er formlegt mál $\mathcal{L(T)}$ fyrstu stéttar sem við köllum
    \emph{mál kenningarinnar}.
  \item Frumsendur kenningarinnar skiptast í tvo flokka;
    \emph{eiginlegar frumsendur}, sem geta verið hvaða yrðingar sem er á málinu
    $\mathcal{L(T)}$, og \emph{rökfrumsendur} sem myndaðar eins fyrir allar svona
    kenningar og eru búnar til með fimm frumsendum:
    \begin{enumerate}[\textbf{F\arabic*}]
    \item  $\mathbf{A} \vee \mathbf{A} \rightarrow \mathbf{A} $.
    \item  $ \mathbf{A} \rightarrow \mathbf{B} \vee \mathbf{A}$.
    \item  $ ( \mathbf{A} \rightarrow \mathbf{B}) \rightarrow (\mathbf{C} \vee \mathbf{A} \rightarrow \mathbf{B} \vee \mathbf{C})$.
    \item  $ \forall \mathbf{x} \mathbf{A} \rightarrow \mathbf{A}_{\mathbf{x}} [\mathbf{a}]$, ef $\mathbf{a}$
      fyrir $\mathbf{x}$  í $\mathbf{A}$.
    \item $\forall \mathbf{x} ( \mathbf{A} \vee \mathbf{B} ) \rightarrow \mathbf{A} \vee \forall \mathbf{x} \mathbf{B}$ ef $\mathbf{x}$ kemur hvergi fyrir frjáls í $\mathbf{A}$.
    \end{enumerate}

    Hér getur $\mathbf{A}, \mathbf{B}, \mathbf{C}$ verið hvaða yrðing á $\mathcal{L(T)}$
    sem vera skal.
  \item Kenningin hefur tvær rökreglur, \emph{modus ponens} og \emph{alhæfingu}:
    \begin{itemize}
    \item \textbf{MP}. Af $\mathbf{A} \rightarrow \mathbf{B}$ og $\mathbf{A}$ leiðir $\mathbf{B}$
    \item \textbf{Alh.} Af $\mathbf{A}$ leiðir $\forall \mathbf{x} \mathbf{A}$.
    \end{itemize}
  \end{enumerate}
\end{skgr}

\begin{ath}
  Ef ég segi: ``Ef x og y eru rauntölur, þá er $x + y = y+x$'', þá meina ég
  ``Fyrir öll x og y gildir: Ef x og y eru rauntölur þá er $x  + y = y + x$''.


  \textbf{Alh}. þýður ekki að $\mathbf{A} \rightarrow \forall \mathbf{x} \mathbf{A}$ gildi !!!
\end{ath}

\begin{skgr}[\emph{Samsendarkenningin}] er formleg kenning fyrstu stéttar sem hefur
  samsemdarmerkið ``$=$'' sem tví stætt umsagnartákn (hugsanlega skilgreint) og
  þ.a.


  \textbf{Eq1}. $\vdash_{\mathcal{T}}$ '$x = x$'

  og þ.a. fyrir sérhverja grunnyrðingu $\mathbf{A}$ sé


 \textbf{Eq2}. $\vdash_{\mathcal{T}}$ '$x = y$' $\rightarrow ( \mathbf{A}_z[\mathbf{x}] \rightarrow \mathbf{A}_z[\mathbf{y}])$.

  (að því gefnu að $\mathbf{x}, \mathbf{y}$ sé innsetjanleg fyrir $\mathbf{z}$ í $\mathbf{A}$).

  Við teljum samsemdarkenninguna með rökfrumsendum, en ekki eiginlegum frumsendum.
\end{skgr}

\begin{daemi}
  \begin{enumerate}[(1)]
  \item Skilgreinum kenningu $\mathcal{O}_1$ umm röðuðu mengi
    sem samsemdarkenningu á málinu $\mathcal{L}$ $(\mathcal{O}_{1})$ sem hefur aðeins
    tvístæða umsagnartáknið '$\leq$' og þrjár eiginlegar frumsendur


    '$x \leq x$', ' $x \leq y \wedge y \leq x \rightarrow x = y$', '$x \leq y \wedge y \leq z \rightarrow x \leq z$'

    Skilgreinum kenningum $\mathcal{O}_2$ um stantar raðir sem samsendar kenningu
    með aðeins tvístæða umsagnar táknið ``$<$'' og tvær eiginlegar frumsendur


    '$\lnot (x < x)$' og $' x < y \wedge y < z \rightarrow x < z$'

  \item
    Skilgreinum tvær kenningar $\mathcal{G}_1$ og $\mathcal{G}_2$ um grúpur,
    báðar sem samsendarkenningar.

    Sú fyrri hefur mál með einu tvístæðu fallatákni,
    '$\cdot$' og tveimur frumsendum


    '$x \cdot (y \cdot z) = (x \cdot y) \cdot z$'

    og

    '$\exists y (\forall x (( x \cdot y = x) \wedge (y \cdot x) = x) \wedge \forall x \exists z ((x \cdot z = y) \wedge (z \cdot x = y)))$'

    seinnni kenningin hefur mál með tvístæðu fallatákni '$\cdot$', einum
    fasta $e$ og einu einstæðu fallatákni $J$; skrifum $x^{-1}$sem skammstöfum fyrir '$Jx$'; höfum þrjár frumsendur

    '$x \cdot (y \cdot z) = (x \cdot y) \cdot z$', '$(x \cdot e = x) \wedge ( e \cdot x = x)$' og '$(x\cdot x^{-1} = e) \wedge (x^{-1} \cdot x = e)$'
  \end{enumerate}


\end{daemi}

\begin{skgr}
  \begin{enumerate}[(1)]
  \item  Kenningin $\mathcal{N}$ fyrir náttúrulegar tölur er skilgreind sem
    samsemdarkenning á málinu $\mathcal{L}(\mathcal{N})$
    ,með fasta '$0$', tvístæðum fallatáknum '$+$' og '$\cdot$', tvístæðu umsagnartákni '$<$' og einstæðu fallatákni '$S$' og eftirfarandi
frumsendum:
    \begin{enumerate}[\textbf{N\arabic*}]
    \item $Sx \neq 0$
    \item $Sx = Sy \rightarrow x = y$
    \item $x + 0 = x$
    \item $x + Sy = S(x+y)$
    \item $x \cdot 0 = 0$
    \item $x \cdot Sy = (x \cdot y) + x$
    \item $\lnot (x < 0)$
    \item $x < Sy \leftrightarrow x < y \vee x = y$
    \item $x < y \vee x = y \vee y < x$
    \end{enumerate}
  \item Kenningin $\mathcal{P}\mathcal{A}$ fyrir nátt. tölur hefur sama mál og
    $\mathcal{N}$, er samsemdarkenning og allar frumsendur í $\mathcal{N}$ eru
    frumsendur í $\mathcal{P}\mathcal{A}$, en auk þess heufr $\mathcal{P}\mathcal{A}$
    frumsendugrip.

    \textbf{IND}. $\mathbf{A}_{\mathbf{x}}[0] \wedge \forall \mathbf{x} ( \mathbf{A} \rightarrow \mathbf{A}_{\mathbf{x}}[S\mathbf{x}]) \rightarrow \mathbf{A}$.

    Köllum $\mathcal{P}\mathcal{A}$ \emph{Peano-reikning}.
  \end{enumerate}
\end{skgr}

\begin{skgr}
  Látum $\mathbf{A}$ vera yrðingarsnið a málinu $\mathcal{L}(\mathcal{P})$
  og $\mathbf{x}_1, \dotsc, \mathbf{x}_n$ vera ólíkar yrðingabreytur
  þ.a. allar yrðingabreytur í $\mathbf{A}$ séu meðal þeirra.
  Látum $\mathbf{B}_1, \dotsc, \mathbf{B}_n$ vera yrðingar á máli
  $\mathcal{L}$ fyrstu stéttar. Táknum með
  \[ \mathbf{A}_{\mathbf{x}_1, \dotsc, \mathbf{x}_n}[\mathbf{B}_1, \dotsc, \mathbf{B}_n] \]
  eða
  \[\mathbf{I}^{\mathbf{B}_1, \dotsc, \mathbf{B}_n}_{\mathbf{x}_1, \dotsc, \mathbf{x}_n} \mathbf{A}\]

  stæðuna sem fæst með því að setja $\mathbf{B}_k$ inn fyrir $\mathbf{x}_k$ í $\mathbf{A}$ fyrir
  $k = 1, \dotsc, n$.
\end{skgr}

\begin{setn}
  $\mathbf{A}_{\mathbf{x}_1, \dotsc, \mathbf{x}_n}[\mathbf{B}_1, \dotsc, \mathbf{B}_n]$ er yrðing á málinu $\mathcal{L}$.

  \emph{Sísönnusetning}. Ef $\mathbf{A}$ er sísanna og $\mathbf{A}_1, \dotsc, \mathbf{A}_n$ yrðingar á $\mathcal{L}$, þá er
  $\mathbf{A}_{\mathbf{x}_1 \dotsc, \mathbf{x}_n}[\mathbf{A}_1, \dotsc, \mathbf{A}_n]$ setning
  í $\mathcal{L}$.

  \begin{proof}
    Látum $\mathbf{C}_1, \dotsc, \mathbf{C}_m$ vera sönnun á $\mathbf{A}$ í $\mathcal{P}$.
    Látum $\mathbf{y}_1, \dotsc, \mathbf{y}_r$ vera upptalningu allra yrðingabreyta sem
    koma fyrir í $\mathbf{C}_1, \dotsc, \mathbf{C}_m$ en eru \emph{ekki}
    meðal breytanna
    $\mathbf{x}_1, \dotsc, \mathbf{x}_n$. Látum

    Skrifum $\theta \mathbf{D}$ fyrir $\mathbf{I}^{\mathbf{A}_1, \dotsc, \mathbf{A}_n, \mathbf{B}_1, \dotsc, \mathbf{B}_r}_{\mathbf{x}_1, \dotsc, \mathbf{x}_n, \mathbf{y}_1, \dotsc, \mathbf{y}_r} D$
    þá er $\theta \mathbf{C}_1, \dotsc, \theta \mathbf{C}_m$ sönnun á $\theta \mathbf{A}$ í sérhverri kenningu
    fyrstu stéttar með málið $\mathcal{L}$ og $\theta \mathbf{A}$ er
    $\mathbf{A}_{\mathbf{x}_1, \dotsc, \mathbf{x}_n} [\mathbf{A}_1, \dotsc, \mathbf{A}_n]$
  \end{proof}
\end{setn}

Sumir nota ``sísönnusetning'' yfir eftirfarandi

\begin{setn}[Fylgisetning]
  Látum $\mathcal{T}$ vera kenningu fyrstu stéttar og $\mathbf{A}_1, \dotsc, \mathbf{A}_n, \mathbf{B}$ vera
yrðingar á máli $\mathcal{T}$. G.r.f. að
$\vdash_{\mathcal{T}} \mathbf{A}_1, \dotsc, \vdash_{\mathcal{T}} \mathbf{A}_n$
og að $\mathbf{A}_1 \wedge \dotsb \wedge \mathbf{A}_n \vdash_{\mathcal{T}} \mathbf{B}$
fáist með innsetningu í sísönnu. Þá er $\vdash_{\mathcal{T}}B$.

  \begin{proof}
   $p_1 \wedge \dotsb \wedge p_n \rightarrow q$
   er rökfræðilega jafngilt
   $p_1 \rightarrow p_2 \rightarrow \dotsb \rightarrow p_n \rightarrow q$.

   $\vdots$
  \end{proof}
\end{setn}


\begin{setn}[Fylgisetning]

  Látum $\mathcal{T}$ vera kenningu fyrstu stéttar og $\mathbf{A}, \mathbf{B}, \mathbf{C}$
  vera yrðingar á $\mathcal{T}$.

  $\vdots$
\end{setn}




$\vdots$



\begin{setn}[Fylgisetning]
  Ef $\vdash \mathbf{A} \rightarrow \mathbf{B}$, þá er
  $ \vdash \exists \mathbf{x} \mathbf{A} \rightarrow   \exists \mathbf{x} \mathbf{B}$
  og $\vdash \forall \mathbf{x}\mathbf{A} \rightarrow \forall \mathbf{x} \mathbf{B}$
\end{setn}


\begin{skgr}
  Látum $\mathbf{A}$ vera yrðingu. \emph{Lokun} yrðingarinnar
  $\mathbf{A}$ er yrðing $\forall \mathbf{x}_1, \forall \mathbf{x}_2, \dotsc,
  \forall \mathbf{x}_n \mathbf{A}$ þar sem $\mathbf{x}_1, \dotsc, \mathbf{x}_n$
er einhver upptalning á breytunum sem koma fyrir frjálsar í $\mathbf{A}$
\end{skgr}

\begin{ath}
  Yrðing er \emph{lokuð} þþaa engin breyta komi fyrir frjáls í henni.
\end{ath}

\begin{setn}[Fylgisetn]
  Látum $\mathbf{A}'$ vera lokun $\mathbf{A}$. Höfum $\vdash \mathbf{A}'$
  þþaa $\vdash \mathbf{A}$.
\end{setn}

\section{Afleiðslusetning fyrir umsagnarökfræði}

Látum $\mathbf{A}$ vera lokaða yrðingu og $\mathbf{B}$ vera
yrðingu á málinu $\mathcal{L(T)}$.
Ef $\mathbf{A} \vdash \mathbf{B}$, þá er $\vdash \mathbf{A} \rightarrow \mathbf{B}$.



\begin{setn}[Fylgisetning]
  Látum $\mathbf{A}, \mathbf{B}$ vera yrðingar og $\mathbf{x}_1, \dotsc, \mathbf{x}_n$
vera upptalningu á breytunum sem koma frjálsar fyrir í $\mathbf{A}$. Látum $\mathcal{T}'$
vera  kenninguna sem fæst með því að bæsta nýjum föstum $\mathbf{e}_1, dotsc, \mathbf{e}_n$
við $\mathcal{T}$. Ef
\[
\mathbf{A}_{\mathbf{x}_1, \dotsc, \mathbf{x}_n}[\mathbf{e}_1, \dotsc, \mathbf{e}_n] \vdash_{\mathcal{T}}
\mathbf{B}_{x_n}[\mathbf{e}_1, \dotsc, \mathbf{e}_n]
\]

þá
\[ \vdash_{\mathcal{T}} \mathbf{A} \rightarrow \mathbf{B} \]



\end{setn}


\begin{setn}[Jafngildisetning]
 Látum $ \mathbf{A}'$ vera yrðingu sem fæst með því að s etja yrðingar
 $\mathbf{B}_1', \dotsc, \mathbf{B}_n'$ sumstaðar í stað hlutyrðinga
 $\mathbf{B}_1, \dotsc, \mathbf{B}_n$ í $\mathbf{A}$ (sem hafa ekkert tákn
sameiginlegt tvær og tvær).
Ef $\vdash \mathbf{B}_k' \leftrightarrow \mathbf{B}_k$ fyrir
$k = 1, \dotsc, n$. Þá er $\vdash \mathbf{A}' \leftrightarrow \mathbf{A}$
\end{setn}


\section{Útvíkkanir með skilgreiningum}

Höfðum kenningu $\mathcal{O}_1$ fyrir röðuð mengi;
það var samsemdarsetning með tveimur tvístæðum umsagnartáknum,
'$=$' og '$\leq$'. Nú viljum við skilgreina nýtt tákni '$<$' með

(*) ' $ x < y \rightarrow (x \leq y \wedge x \neq y)$'.

í stað þess að líta á þetta sem skammstöfun er hentugra að bæta við
'$<$' sem nýju umsagnartákni og (*) sem nýrri frumsendu.

Höfðum líka $\mathcal{O}_2$ með táknum
'$=$' og '$<$'. Nú viljum við skilgreina nýtt tákni '$\leq$' með
og nýrri frumsendum.

(*) ' $ x \leq y \rightarrow (x < y \vee x = y)$'.

Þannig fást tvær nýjar kenningar með sömu umsagnartáknum og sömu setningum!

Höfum kenningu A fyrir víxlgrúpur með táknum '$=$' og tvístæðu fallatákni '$+$',
frumsendum
'$(x+y) + z = x + (y + z)$', '$x + y = y + x$',
'$ \exists y \forall x (x +y = x) $', '$\forall y \forall z \exists x (x + y = z)$'

þá má sanna

\[ \exists y ( \forall x (x+y = x) \wedge \forall z (\forall x (x+z = x)) \rightarrow z = y) \]

Þá viljum við bæta við núllstæðu fallakákni (fasta) 0 ásamt nýrri frumsendu;
annaðvhort

'$y = 0 \leftrightarrow \forall x ( x+y = x)$'

eða bara
'$\forall x (x + 0 = x)$'

Sömuleiðis má sanna að fyrir hvert x er til \emph{nákv. eitt} y þ.a.
'$x+y = 0$'; viljum bæta við 1-stæðu fallatákni '$-$' og frumsendu

$'y = -x \leftrightarrow x + y = 0'$

eða

$'x + (-x) = 0'$


\begin{skgr}
\begin{enumerate}
	\item Látum $\mathcal{L}$ vera mál fyrstu stéttar. Mál fyrstu stéttar $\mathcal{L}'$ er \emph{útvíkkun} málsins $\mathcal{L}$
		ef sérhvert eiginlegt tákn í $\mathcal{L}$ er líka samskonar tákn (þ.e. ef \textbf{p} er n-stætt umsagnar tákn í $\mathcal{L}$,
		þá á það að vera n-stætt umsagnar tákn í $\mathcal{L}'$, eins með falla tákn.)
	\item Kenning \(\mathcal{T}'\) fyrstu stéttar er \emph{útvíkkun} annarar kenningar \(\mathcal{T}\) fyrstu stéttar ef $\mathcal{L(T')}$
		útvíkkun $\mathcal{L(T)}$.
	\item Segjum að slík útvíkkun $\mathcal{T'} á \mathcal{T}$ sé \emph{íhaldssöm} ef sérhver setning í $\mathcal{T'}$
		er setning í $\mathcal{T}$
\end{enumerate}
\end{skgr}

Látum nú $\mathcal{T}$ vera kenningu fyrstu stéttar, $\mathbf{D}$ vera yrðingu á máli $\mathcal{T}$ þ.a. engar breytur nema ólíkubreyturnar
$\mathbf{x}_1, \dotsc, \mathbf{x}_n$ komi fyrir frjálsar í $\mathbf{D}$. Bætum við $\mathcal{T}$ nýju umsagnartákni $\mathbf{p}$.
og nýrri frumsendu
\[ p \mathbf{x}_1, \dotsc, \mathbf{x}_n \leftrightarrow \mathbf{D} .\]
Fáum þannig útvíkkun $\mathcal{T'}$ af $\mathcal{T}$.

\begin{setn}
Fyrir hverja yrðingu $\mathbf{A}$ í $\mathcal{T'}$ má búa til yrðingu $\mathbf{A}^{*}$ á málinu $\mathcal{L(T)}$
þ.a. $\vdash_{\mathcal{T'}} A$ þþaa $\vdash_{\mathcal{T}} \mathbf{A}^{*}$. Raunar:

\begin{enumerate}[(i)]
\item $\vdash_{\mathcal{T'}} A \leftrightarrow \mathbf{A}^{*}$
\item $\mathcal{T'}$ er íhaldssöm útvíkkun á $\mathcal{T}$.
\end{enumerate}

\end{setn}


Svipað fyrir fallatákn. Látum $\mathbf{D}$ vera yrðingu í
$\mathcal{L(T)}$ og $\mathbf{x}_1, \dotsc, \mathbf{x}_n$
vera ólíkar breytur þ.a. engar aðrar koma fyrir frjálsar í 
$\mathbf{D}$. og $\mathbf{z}$ vera ólíka öllum í $\mathbf{D}$
þ.a.
\[ \vdash_{\mathcal{T}} \exists y ( \mathbf{D} \wedge \forall z ( \mathbf{D}_{y} [z] \rightarrow \mathbf{z} = \mathbf{y})) \]

þá má bæta við n-stæðu fallatákni $\mathbf{f}$ og bæta við frumsendu
\[ \mathbf{y} = \mathbf{f}_{\mathbf{x}_1 \dotsb \mathbf{x}_n} \leftrightarrow \mathbf{D} \]
(ef $\mathcal{T}$ er samsembnarsetning eða (heldur reynir)
\[ \mathbf{D}_{\mathbf{y}} [\mathbf{f} \mathbf{x}_1 \dotsb \mathbf{x}_n \]

annars )


\chapter{Kafli III. Líkön}

\section{Mynztur og líkön}

\begin{skgr}
  \begin{enumerate}[(1)]
  \item  \emph{Mynztur} fyrir mál \(\mathcal{L} \) fyrstu stéttar er gefið með:
    \begin{enumerate}[(i)]
    \item  Mengi \(|M| \) (sem er ekki tómt)
    \item  Fyrir sérhvert n-stætt fallatákn \(\mathbf{f}\) í \(\mathcal{L}\)
       vörpun: 
       \[ \mathbf{f}_{M}: |M|^n \rightarrow |M| \]
     \item Fyrir sérhvert n-stætt umsagnartákn \(\mathbf{p} \) í \(\mathcal{L}\)
       n-stæð venzl $\mathbf{p}_{M}$, þ.e. hlutmengi í $|M|^{n}$.
    \end{enumerate}
    Ef $\mathcal{L}$ hefur samasemmerki $'='$, þá segjum við að $M$ sé 
    \emph{samsemdarmynztur}. Ef $=_{M}$ er venjuleg samasemdar venzl á $|M|$.
  \item Látum $M$ vera mynztur fyrir $\mathcal{L}$. og $V_{\mathcal{L}}$ vera mengi breytanna í $\mathcal{L}$.

    Vörpun $s: V_{\mathcal{L}} \rightarrow |M|$ kallast \emph{úthlutun}.
  \item Látum s vera slíka úthlutun. Ef $\mathbf{x}$ er breyta og $a \in |M|$, þá táknar
    $s_{\mathbf{x}}[a]$ úthlutunina
    \[s_{\mathbf{x}}[a](\mathbf{y}) := \bcondef s(\mathbf{y}) & \text{ ef $\mathbf{y}$ er ekki $\mathbf{x}$ } \\ a & \text{ ef $\mathbf{y}$ er $\mathbf{x}$} \econdef \]
  \item Fyrir sérhvert heiti $\mathbf{a}$ í $\mathcal{L}$ og sérhverja úthlutun s skilgreinum við stak
    $\mathbf{a}_{M}(s)$ í |M| þannig:
    \begin{enumerate}[(1)]
    \item  Ef $\mathbf{a}$ er breyta $\mathbf{x}$, þá er 
      \[ \mathbf{a}_{M}(s) = s(\mathbf{x}) \]
    \item Ef $\mathbf{a}$ er $\mathbf{f} \mathbf{a}_1 \dotsb \mathbf{a}_n$, þar sem 
      $\mathbf{f}$ er n-stætt fallatákn og $\mathbf{a}_1, \dotsc, \mathbf{a}_n$ eru heiti,
      þá er
      \[ \mathbf{a}_{M} (s) = \mathbf{f}_{M}(\mathbf{a}_{1,M}(s), \dotsc, \mathbf{a}_{n,M}(s)) \]
    \end{enumerate}
    sér í lagi er $\mathbf{a}_{M}(s) = \mathbf{c}_M$ ef a er fastinn $\mathbf{c}$.
  \end{enumerate}
\end{skgr}

\begin{skgr}
  Látum M vera mynztur og s vera úthlutun. Fyrir allar yrðingar $\mathbf{A}$ í $\mathcal{L}$ skilgreinum
við hvað það þýðir að ``s fullnægi $\mathbf{A}$ í M'', skrifað

\[ \models_{M} \mathbf{A} (s) \].

með þrepun
\begin{enumerate}[(1)]
\item  Ef $\mathbf{A}$ er grunnyrðing $\mathbf{p} \mathbf{a}_1 \dotsb \mathbf{a}_n$,
  þá er $\models_{M} \mathbf{A} (s)$ þþaa $(\mathbf{a}_{1,M}(s), \dotsc, \mathbf{a}_{n,M}(s)) \in \mathbf{p}_{M}$.
\item Ef $\mathbf{A}$ er $\lnot \mathbf{B}$, þá er $\models_{M} \mathbf{A}(s)$ þþaa ekki gildi
  $\models_{M} \mathbf{B}(s)$.
\item Ef $\mathbf{A}$ er $\mathbf{B} \vee \mathbf{C}$, þá gildir 
  $\models_{M} \mathbf{A}(s)$ þþaa $\models_{M} \mathbf{B}(s)$ eða $\models_{M} \mathbf{C}(s)$.
\item Ef $\mathbf{A}$ er $\forall x \mathbf{B}$, þá $\models_{M} \mathbf{A}(s)$ þþaa
  $\mathbf{B}_{\mathbf{x}} (s_{\mathbf{x}}[a])$ gildi fyrir öll a úr M
\end{enumerate}

\end{skgr}

\begin{skgr}
  $\models_{M} \mathbf{A}$ þþaa $\models_{M} \mathbf{A}(s)$ f. öll s.
\end{skgr}

\begin{ath}
 $\mathbf{C}_{M}: |M|^{0} \rightarrow |M|$, þar sem $|M|^{0} = \set{0}$
\end{ath}




Við segjum að yrðing $\mathbf{A}$ sé \emph{sönn} í mynztri M ef
\( \models_{M} \mathbf{A} (s) \) fyrir allar úthlutanir s.

\begin{ath}
  Látum \(\mathbf{A}'\) vera lokun yrðingar \(\mathbf{A}\) á málinu
\(\mathcal{L} \). Við höfum
\[ \models_{M} \mathbf{A} \text{ þþaa } \models_{M} \mathbf{A}' \]
\end{ath}

\begin{skgr}
  Látum \( \mathcal{L} \) vera mál fyrstu stéttar. við segjum að yrðing $\mathbf{A}$
sé \emph{röksönn} ef  $\models_{M}\mathbf{A}$ fyrir öll mynstur M.
...
Ef \(\models_{M}\mathbf{A}\) fyrir sérhvert mynztur M fyrir \(\mathcal{L}\)
þ.a. $\models_{M} \mathbf{B}$ fyrir öll $\mathbf{B}$ úr $\mathcal{H}$. 
\end{skgr}


\begin{setn}
  Látum $\mathcal{T}$ vera kenningu fyrstu stéttar á máli \(\mathcal{L}\) sem
  hefur engar \emph{eiginlegar} frumsendur. Þá gildir.
  \begin{enumerate}[(1)]
  \item  Sérhver frumsenda í $\mathcal{T}$ er röksönn
  \item Ef $\mathcal{H} \vdash \mathbf{A}$, þá $\mathcal{H} \models \mathbf{A}$.
  \end{enumerate}

\end{setn}

\begin{proof}
  Þetta er einföld afleiða af skilgreiningu.
  Sönnum t.d. að frumsenda $\mathbf{A} \vee \mathbf{A} \rightarrow \mathbf{A}$.
  sé röksönn. Annars væri til mynztur M fyrir $\mathcal{L}$ og úthlutun s þannig að
  \emph{ekki} gildi 

  \[ \models_{M} (\mathbf{A} \vee \mathbf{A} \rightarrow \mathbf{A}) (s) \]
  þ.e. ekki gildi
  \[ \models_{M} ( \lnot ( \mathbf{A} \vee \mathbf{A} ) \vee \mathbf{A}) (s) \]
  þá væri ekki
  \[\models_{M} ( \lnot (\mathbf{A} \vee \mathbf{A}))(s)\] og ekki $\models_{M} \mathbf{A}(s)$
  og það þýðir að  \( \models_{M} (\mathbf{A} \vee \mathbf{A}) (s) \) og \emph{ekki} $\models_{M} \mathbf{A} (s)$
  En $\models_{M}( \mathbf{A} \vee \mathbf{A}) (s)$ þýðir að $\models_{M} \mathbf{A}(s)$. Höfum þá bæði
  $\models_{M} \mathbf{A}(s)$ og ekki $\models_{M} \mathbf{A}(s)$, sem er mótsögn.
\end{proof}


\begin{setn}[Fylgisetning]
 Látum $\mathcal{T}$ vera kenningu fyrstu stéttar á máli $\mathcal{L}$. Ef M er
 mynztur fyrir $\mathcal{L}$ þ.a. sérhver eiginleg frumsenda í $\mathcal{T}$ sé sönn
 í M, þá er sérhver setning í $\mathcal{T}$ sönn í M, þá er sérhver setning í 
 $ \mathcal{T} $ sönn í M.
\end{setn}

\begin{skgr} Látum $\mathcal{T}$ vera kenningu fyrstu stéttar á máli $\mathcal{L}$ þ.a. 
  sérvher eiinleg frumsenda í $\mathcal{T}$ sé sönn í mynztrinu. Ef $\mathcal{T}$ er samsemdarkenning,
  þá er \emph{samsemdarlíkan} fyrir $\mathcal{T}$ líkan fyrir $\mathcal{T}$ sem er 
  samsemdarmynztur.
\end{skgr}

\begin{ath}
  Ef $\mathcal{T}$ er samsemdarkenning og M er líkan fyrir $\mathcal{T}$, þá eru
  venzlin$=_{M}$ jafngildisvenzl. Látum $M'$ vera mengi jafngildisflokkana. Þá verður $M'$.
  sjálfkrafa að samsemdarlíkani.
\end{ath}

Fyrir n stætt fallatákn $\mathbf{f}$ skilgreinum við
\[\mathbf{f}_{M'}([a_1], \dotsc, [a_n]) := \mathbf{f}_{M} (a_1, \dotsc, a_n) \]
Þar sem $[a_k]$ er jafngildisflokkur $a_k$;
Þetta er ekki vel skilgreint, því að
 \[\vdash_{\mathcal{T}} \mathbf{a}_1 = \mathbf{b}_1 \wedge \mathbf{a}_2 = \mathbf{b}_2
 \wedge \dotsb \wedge \mathbf{a}_n 
= \mathbf{b}_n \rightarrow \mathbf{f}\: \mathbf{a}_1 \dotsb \mathbf{a}_n = \mathbf{f}\: \mathbf{b}_1 \dotsb \mathbf{b}_n \]

svo að fyrir $a_1, \dotsc, a_n, b_1, \dotsc, b_n$ þ.a. 
$a_k =_{M} b_k$ fyrir $k = 1, \dotsc, n$ er $\mathbf{f}_{M} (a_1, \dotsc, a_n) =_{M} \mathbf{f}_{M} (b_1, \dotsc, b_n)$A
Eins með umsagnartáknin.


\begin{daemi}
  Samsemdarlíkan fyrir kenninguna $\mathcal{O}_1$ ( með tákn $'='$ og $'\leq'$
  er raðað mengi. (\emph{Líkan} fyrir $\mathcal{O}_1$ er ``\emph{forraðað}'' mengi.)

  Samsemdarlíkan fyrir kenningarnar $A_1, A_2$
  fyrir víxlgrúpur er víxlgrúpa; samsemdarlíkön fyrir $G_1, G_2$ eru 
  grúpur.

  Hvenær hefur kenning fyrstu stéttar líkan?
  Nauðsynlegt skilyrði er að hún sé \emph{samkvæm}; Ef hún er samkvæm, þá er 
  hún samkvæm m.t.t. kennigarinnar ($\mathbf{A} \rightarrow (\lnot \mathbf{A} \rightarrow \mathbf{B})$ er setningargrip)
  en yrðingarnar $\mathbf{A}$ og $\lnot \mathbf{A}$ geta ekki báðar verið sannar í sama mynztri.
\end{daemi}

Stenum á að sanna:

\begin{setn}[Meginsetning]
  Kenning fyrstu stéttar hefur líkan þþaa hún sé samkvæm
\end{setn}

\begin{setn}[Fullkomleikasetning Gödels]
  Látum $\mathcal{T}$ vera kenningu fyrstu stéttar á máli $\mathcal{L}$.
  Yrðing á málinu $\mathcal{L}$ er setning í $\mathcal{T}$ þþaa
  hún sé sönn í sérhverju líkani fyrir $\mathcal{T}$.
\end{setn}

Til að sjá að FG sé afl. af meginsetningu:

\begin{setn}[Hjálparsetning]
 Lokuð yrðing $\mathbf{A}$ er setning í $\mathcal{T}$ þþaa
 kenningin $\mathcal{T}[\lnot \mathbf{A}]$ sé ósamkvæm.
\end{setn}




\begin{proof}
 Ef $\mathbf{A}$  er setning í $\mathcal{T}$, þá er $\mathbf{A}$ og $\lnot \mathbf{A}$
 setningar í $\mathcal{T}[\lnot \mathbf{A}]$, svo að
 $\mathcal{T}[\lnot \mathbf{A}]$ er ósamkvæm. Ef hinsvegar
 $\mathcal{T}[\lnot \mathbf{A}]$ er  ósamkvæm, þá er $\mathbf{A}$
 setning í $\mathcal{T}[\lnot \mathbf{A}]$, svo að
 $\lnot \mathbf{A} \vdash_{\mathcal{T}} \mathbf{A}$.

 Þar sem $\mathbf{A}$ er lokuð, gefur afleiðsluseting að

 \[\vdash_{\mathcal{T}} \lnot \mathbf{A} \rightarrow \mathbf{A},\]
 en \[ \vdash_{\mathcal{T}} ( \lnot \mathbf{A} \rightarrow \mathbf{A}) \rightarrow \mathbf{A}\]

 skv. sísönnusetningu; svo að $\vdash_{\mathcal{T}} \mathbf{A}$ skv. \textbf{MP}.

\end{proof}

\begin{proof}[Sönnun á Fullkomleikasetningu Gödels]
  G.r.f. að $\mathbf{A}$ sé ekki setning í $\mathcal{T}$ og látum $\mathbf{A}'$
  vera lokun $\mathbf{A}$. Þá er $\mathbf{A}'$ ekki heldur seting í
  $\mathcal{T}$. Þá er $\mathcal{T}[\lnot \mathbf{A}']$ samkvæm skv. HS og hefur
  þá líkan M skv. meginsetningu.

  En þá er M líkan fyrir $\mathcal{T}$ þ.a. $\lnot \mathbf{A}'$ sé sönn í M
  og þá er $\mathbf{A}'$ ósönn í M.
  
\end{proof}




\vdots


\begin{ath}
  Henkin: Fyrir sérhverja lokaða yrðingu \( \exists \mathbf{x}\mathbf{A}\), þá
  er til fasti \(\mathbf{c}\) í $\mathcal{T}$ þ.a. $\vdash  \exists \mathbf{x}\mathbf{A} \rightarrow  \mathbf{A}_{\mathbf{x}}[\mathbf{c}]$

  Meginsetning:
  Kenning hefur líkan þþaa hún sé samkvæm.
  Þ.e. hún hefur ekki bæði $\mathbf{A}$ og $\lnot \mathbf{A}$
\end{ath}


\begin{setn}
  Látum $\mathcal{T}$ vera fullkomna henkin-kenningu
  með stafróf $\mathcal{S}$. Þá hefur $\mathcal{T}$ líkan M
  þ.a. $|M| = \#\mathcal{S}$;
 % ef hún er samsemdarkenning þá hefur hún samsemdalíkan $M$ þ.a. $|M| \leq \#\mathcal{S}$.

  \begin{proof}
    Aðeins er eftir að sanna fyrri fullyrðinguna.
    Látum $|M|$ vera mengi allra ``lokaðra heita'' (þ.e. heiti án breyta) á málinu
    $\mathcal{L}$. Fyrir sérhvert n-stætt fallatákn $\mathbf{f}$
    látum við $\mathbf{f}_M$ vera fallið $|M|^{\alpha} \rightarrow |M|$
    þ.a.

    \[\mathbf{f}_M ( \mathbf{a}_1, \dotsc,\mathbf{a}_n ) := \mathbf{f} \mathbf{a}, \dotsc, \mathbf{a}_n ,\]
    
    og fyrir n-stætt umsagnartákn $\mathbf{p}$ látum við $\mathbf{p}_M$ veera mengi allra
    n-unda $(\mathbf{a}_1, \dotsc, \\mathbf{a}_n)$ þ.a. 
    \(\mathbf{a}_1, \dotsc, \mathbf{a}_n \)
    þ.a. $\vdash_{\mathcal{T}} \mathbf{p} \mathbf{a}_1, \dotsc, \mathbf{a}_n$.
    Okkur nægir að sýna að fyrir lokaða yrðingu $\mathbf{A}$ gildir
    \[ \models_M \mathbf{A} \text{ þþaa } \vdash_{\mathcal{T}} \mathbf{A}; \]
    því að fyrir frumsendur $\mathbf{A}$ í $\mathcal{T}$. 
    Gildir $\vdash_{\mathcal{T}} \mathbf{A}'$; þar sem $\mathbf{A}'$ er lokun
    $\mathbf{A}$; en af því leiðir $\models_M \mathbf{A}'$ þá er líka $\models_M \mathbf{A}$.
    Svo að M er líkan fyrir $\mathcal{T}$. Þrepun yfir lengd yrðinga:
    \begin{enumerate}[(1)]
    \item Ef $\mathbf{A}$ er grunnyrðing, þá er þetta bein afleiðing af skilgr.
    \item $\mathbf{A}$ er $\lnot \mathbf{B}$. Ef $\models_M \mathbf{A}$, þá er ekki
      $\models_M \mathbf{B}$ og því ekki $\vdash_{\mathcal{T}} \mathbf{B}$ skv. þf.
      þar sem $\mathcal{T}$ er fullkomin og $\mathbf{B}$ lokað er
      $\vdash_{\mathcal{T}} \lnot \mathbf{B}$, þ.e. $\vdash_{\mathbf{A}}$. En ef
      ekki $\models_{M} \mathbf{A}$, þá er $\models_M \mathbf{B}$ og því
      $\vdash_{\mathcal{T}} \mathbf{B}$ skv. þf. En þar sem $\mathcal{T}$ er 
      samkvæm er ekki $\vdash_{\mathcal{T}} \lnot \mathbf{B}$ og því ekki
      $\vdash_{\mathcal{T}} A$.
    \item $\mathbf{A}$ er $\mathbf{B} \vee \mathbf{C}$: Ef $\models_M \mathbf{A}$,
      þá er annaðhvort $\models_M \mathbf{B}$ eða $\models_M \mathbf{C}$,
      svo að $\vdash_{\mathcal{T}} \mathbf{B}$ eða $\vdash_{\mathcal{T}} \mathbf{C}$ skv. þf.
      í báðum tilvikum er $\vdash_{\mathcal{T}} \mathbf{A}$ skv. sís.
      Ef ekki $\models_M \mathbf{A}$, þá er hvorki $\models_M \mathbf{B}$
      né $\models_M \mathbf{C}$ og því hvorki $\vdash_{\mathcal{T}}$ né 
      $\vdash_{\mathcal{T}} \mathbf{C}$.
      En $\mathcal{T}$ er fullkomin svo að $\vdash_{\mathcal{T}} \lnot \mathbf{B}$ 
      og $\vdash_{\mathcal{T}} \lnot \mathbf{C}$ og þá ekki $\vdash_{\mathcal{T}} \mathbf{A}$ skv. sís.

    \item $\mathbf{A}$ er $\forall \mathbf{x} \mathbf{B}$. Ef
      $\mathbf{B}$ er lokað, þá fæst $\models_M \mathbf{A}$ þþaa $\models_M \mathbf{B}$ sem skv.
      þf. gildi þþaa $\vdash_{\mathcal{T}} \mathbf{B}$ og það er jafngilt $\vdash_{\mathcal{T}} \mathbf{A}$A

      G.r.f. að $\mathbf{B}$ sé ekki lokað, þá er $\mathbf{x}$ eina breytan sem er frjáls í $\mathbf{B}$.A

      \begin{enumerate}[(a)]
      \item  G.r.f. að $\models_M \mathbf{A}$ en  ekki $\vdash_{\mathcal{T}} \mathbf{A}$.
        Skv. fullkomleika er $\vdash_{\mathcal{T}} \lnot \mathbf{A}$ og því
        $\vdash_{\mathcal{T}} \exists \mathbf{x} \lnot \mathbf{B}$. Þar sem
        $\mathcal{T}$ er hengin er til fasti $\mathbf{c}$ þa.a. 
        $\vdash_{\mathcal{T}} \lnot \mathbf{B}_{\mathbf{x}} [\mathbf{c}]$.
        En vegna $\models_M \forall \mathbf{x} \mathbf{B}$ svo að
        $\models_M \mathbf{B}_{\mathbf{x}} [\mathbf{c}]$ fyrir öll
        $\mathbf{c}$ í M.
        Skv. þf er $\vdash_{\mathcal{T}} \mathbf{B}_{\mathbf{x}} [\mathbf{c}]$. 
        Þetta er mótsögn, því að $\mathcal{T}$ er samkvæm.
      \item  G.rf. að $\vdash_{\mathcal{T}} \mathbf{A}$ en ekki $\models_M \mathbf{A}$.
        Vegna ekki $\models_M \forall \mathbf{x} \mathbf{B}$ er til úthlutun s
        þ.a. ekki $\models_M \forall \mathbf{x} \mathbf{B} (s)$ og því til $\mathbf{c}$ úr
        $|M|$ þ.a. ekki $\models_M \mathbf{B}_{\mathbf{x}} ( s_{\mathbf{x}}[\mathbf{c}])$;
        en það jafngildir $\models_M \mathbf{B}_{\mathbf{x}}[\mathbf{c}]$. En 
        $\vdash_{\mathcal{T}} \forall \mathbf{x} \mathbf{B}$ og því
        $\vdash_{\mathcal{T}} \mathbf{B}_{\mathbf{x}} [\mathbf{c}]$ og því
        $\models_M \mathbf{B}_{\mathbf{x}} [\mathbf{c}]$ skv. þf. 

        Þetta er mótsögn. Því sést:
        Ef $\vdash_{\mathcal{T}} \mathbf{A}$, þá $\vdash_{\mathcal{T}} \mathbf{A}$,
        þá $\models_M \mathbf{A}$.
      \end{enumerate}
    \end{enumerate}
    \end{proof}
\end{setn}

\begin{setn}[Löwenheim-Skolem-setningin]
Ef kenning fyrstu stéttr hefur teljanlegt stafróf,
þá hefur hún teljanlegt líkan.
\end{setn}

\begin{skgr}
  \begin{enumerate}[(1)]
  \item  Segjum að kenning $\mathcal{T}$ sé \emph{hluti}
    af kenningu $\mathcal{T}'$ ef $\mathcal{T}$ og $\mathcal{T}'$ hafa
    sama mál og sérhver eiginleg frumsenda í $\mathcal{T}$ er frumsenda 
    í $\mathcal{T}'$
  \item Segjum að kenning $\mathcal{T}$ sé \emph{endanleg frumsenduð}
    ef fjöldi eiginlegra frumsenda í $\mathcal{T}$ er endanleg.
  \end{enumerate}
\end{skgr}

\begin{ath}
 Hér teljum við \textbf{Eq 1} og \textbf{Eq 2} til rökfrumsenda fyrir
 samsemdarkenningar.
\end{ath}

\begin{setn}[Þjöppunarsetning]
  Yrðin í kenningu $\mathcal{T}$ er sönn í $\mathcal{T}$ ef (þ.e. sönn
  í sérhverju líkani fyrir $\mathcal{T}$ ) þþaa
  hún sé sönn í endanlega frumsenduðum hlut af $\mathcal{T}$.
  Sama gildir fyrir samsemdarkenningu
\end{setn}

\begin{proof}
  Höfum $\mathcal{T} \models \mathbf{A}$ þþaa $\vdash_{\mathcal{T}} \mathbf{A}$
  skv. fullkomleika setningu. En í sönnun $\mathbf{A}$ eru aðeins endalega
  margar eiginlegar frumsendur notaðar, svo að $\vdash_{\mathcal{T}} \mathbf{A}$
  jafngildir því að til sé endanlega frumsendaður hluti $\mathcal{T}_{1}$
  af $\mathcal{T}$ þ.a. $\vdash_{\mathcal{T}_1} \mathbf{A}$, en það jafngildir
  $\mathcal{T}_1 \models \mathbf{A}$
\end{proof}
\begin{setn}[Fylgisetn]
  Kenning $\mathcal{T}$ hefur líkan þþaa sérhver endanlega
  frumsendaður hluti af $\mathcal{T}$ hafi líkan.
  
  Samsemdarkenning hefur samsemdarlíkan þþaa sérhver endanlega frumsendaður
  hluti af $\mathcal{T}$ hafi samsemdarlíkan
\end{setn}

\begin{proof}
  Notum þjöppunarsetningu á yrðingu $\mathbf{A}$ af gerðinni
  $\mathbf{B} \wedge \lnot \mathbf{B}$. Kenning hefu líkan þþaa
  hún sé samkvæm þþaa ekki gildir $\vdash_{\mathcal{T}} \mathbf{B} \wedge \lnot \mathbf{B}$
  sérhver endanlega frumsendaður hluti af $\mathcal{T}$ hafi líkan.
\end{proof}

\begin{daemi}
  Notum þetta á kenningu $\mathcal{F}$ fyrir svið (stundum kallaðir kroppar);
  hún hefur svo fasta 0 og 1 og tvö tvístæð fallatákn $+$ og $\cdot$, er samsemdar
  kenning og hefur eiginlegar frumsendur.
\end{daemi}


$\vdots$

Sýnum að ekki er til einföld útvíkkun $\mathcal{T}$ á $\mc{F}$ þ.a.
samsemdar líkön $\mc{T}$ séu nákvæmlega öll endanleg svið. G.r.f. að
slík kenning sé til. Látum $\mb{B}_n$ vera yrðingin sem segir að til séu
$n$ ólík stök, t.d. er $\mb{B}_{3}$ yrðingin
\[\exists x \exists y \exists z ( x \neq y \wedge x \neq z \wedge y \neq z)\]

ef við bætum öllum $\mb{B}_n$ við $\cT$ sem nýjum frumsendum, þá getur kenningin
$\cT'$ sem þannig fæst ekki haft líkan. Þar með er til endanlega frumsendaður hluti
$\cT''$ af $\cT'$ sem hefur ekkert líkan.

Látum $n_0$ vera stærra en öll $n$ þ.a. $\mb{B}_n$ sé eiginleg frumsenda í
$\cT''$ og K vera endanlegt svið þ.a. $\#K \geq n_0$. Þá er K líkan fyrir
$\cT''$ sem er mótsögn.

\begin{daemi}
  Látum $\cL$ vera mál með jafnaðarmerki, föstum '0', '1',
  falla táknum $'+'$ og $'\cdot'$ og tvístæðu umsagnartákni $<$.
  Þá er rauntalnasviðið $\R$ mynstur fyri $\cL$ með augljósum hætti.
  Látum $\cT$ vera mengi allra yrðinga á $\cL$ sem eru sannar í $\R$.
  Þá er $\cT$ kenning sem hefur teljanlegt stafróf og því teljanlegt líkan.

  Svo að $\cT$ gefur ekki fullkomna lýsingu á $\R$.

  Látum nú $\cL^{*}$ vera málið sem fæst með því að bæta við $\cL$ fasta 
  $\mb{r}_x$ fyrir sérhverja rauntölu $x$ og lítum á $\R$
  sem mynstur fyrir $\cL^*$  með því að 
  $(\mb{r}_x)_{\R} = x$. Látum $\cT^*$ vera mengi allra
  yrðinga í $\cL^*$ sem eru sannar í $\R$.
  Þá er $\R$ líkan fyrir $\cT^*$. Bætum fasta $\mb{c}$ við 
  $\cL^*$ og frumsendunum $\mb{r}_x < \mb{c}$ f. öll
  $x \in \R$; fáum þá kenningu $\cT^{**}$ sýnum að hún hefur líkan:
  Látum $x_1, \dotsc, x_n \in \R$ veljum $y > x_1, \dotsc, x_n$
  og setjum $\mb{c}_{\R} = y$


  Þá fæst líkan fyrir $\cT^*$ að viðbættum fumsendu
  $\mb{r}_{x_k} < r < \mb{c}$ fyrir $k = 1, \dotsc, n$; nefnilega 
  $\R$ sjálft!

  En þá hefur $\cT^{**}$ líkan ${}^* \R$ skv. þjöppunarsetningu. Þetta er 
  svið þ.a. allar setningar í $\cT^*$ eru sannar í ${}^* \R$; það inniheldur
  $\R$ sem hlutsvið (öllu heldur einsmóta eintak, nefninlega 
  $\set{ (\mb{r}_x)_{{}^* \R}: x \in \R}$ og til er tala 
  $y$ í ${}^* \R$, nefninlega $y := \mb{c}_{{}^* \R}$,
  þ.a. $x < y$ fyrir öll $x$ úr $\R$, því ${}^* \R$ er
  \emph{óarkímedískt raðsvið} (og þá er $0 < \1{y} < x$ f. öll $x \in R$
\end{daemi}


\begin{setn}[Tarski]
  Látum $\cT$ vera samsemdarkenningu með stafrófi $\mc{S}$ 
  og $\lambda$  vera fjöldatölu þ.a. $\lambda \geq \# \mc{S}$.
  Ef $\cT$ hefur óendanlegt samsemdarlíkan, þá hefur það samsemdarlíkan með fjölda
  tölu $\lambda$.
\end{setn}

\begin{proof}
  Búum til nýja kenningu $\cT'$ með því að bæta mengi $C$ af nýjum
  föstum við $\cT$, þar sem $\# C = \lambda$, og hverja tvo ólíka
  fasta $\bc$ og $\bc'$ úr $C$ nýrri frumsendu $\bc \neq \bc'$.
  Sýnum að $\cT'$ hafi samsemdarlíkan.

  Skv. þjöppunarsetningu nægir að sýna að sérhver endanlega frumsendaður
  hluti af $\cT'$ hafi samsemdarlíkan. Látum $\bc_1, \dotsc, \bc_n$ vera
  fastana sem koma fyrir í nýjum
  frumsendum í $\cT_1'$ og $M$ vera óendanlegt samsemdarlíkan fyri $\cT$
  og $a_1, \dotsc, a_n$ 
vera ólík stök í $|M|$; gerum $M$ að mynztri fyrir $\cT_1'$ með því að láta
  $\bc_{k,M}$ vera $a_k$ og $\bc_M$ vera hvað sem er ef $\bc$ er í
  $C \setminus \set{\bc_1, \dotsc, \bc_n}$. Þá er $M$ samsemdarlíkan fyrir
  $\cT_1'$.

  Fjöldatala stafrófs $\cT'$ er í hæsta $\lambda + \lambda = \lambda$,
  svo að að getum valið $M$ með $\# |M| \leq \lambda$; en
  $\bc_M \neq \bc_M'$ f. öll $\bc, \bc'$ ílík í $C$. Svo að 
  $\# |M| \geq \lambda$ því er $\# |M| = \lambda$
  
  
\end{proof}

\begin{ath}
  Fyrir fjöldatölur $\lambda, \mu$A þ.a. $\mu \leq \lambda$, $\lambda$ óendanlegt
  $\mu \neq 0$, er 
  \[ \mu + \lambda = \mu X = X \]
\end{ath}


\chapter{IV}


\section{Rakin föll}

í þessum kafla er ``tala'' náttúruleg tala, ``fall'' er vörpun
$\N^{n} \to \N$, ``venzl'' er hlutmengi í $\N^{n}$. Leyfum okkur að nota
rökfræði tákn í yfirmálinu.

Kennifall n-stæðra vensla $R$ er 
$c_R: \N^n \to \N$,

\[ c_R (x_1, \dotsc, x_n) := \bcondef 0 & \Ef R(x_1, \dotsc,x_n) \\ 1 & \Ef \lnot R(x_1,\dotsc,x_n).\\ \econdef \]

(Skrifum $R(x_1,\dotsc, x_n)$ í stað $(x_1, \dotsc, x_n) \in R$.)

\begin{skgr}
  \emph{Rakin föll} eru skilgrein með þrepun þannig:

  \begin{enumerate}[R\arabic*]
  \item  Eftirfarandi föll eru rakin:
    \begin{enumerate}[(a)]
    \item \emph{Núllfallið} Z þ.a. $Z(x) = 0$
    \item \emph{Eftirfarafallið} $N(x) = x+1$
    \item Ofanvörpun $I_i^n(x_1, \dotsc, x_n) := x_i$
    \end{enumerate}
  \item Ef $G,H_1, \dotsc, H_n$ eru rakin föll, þ'er fallið
    $F$ þ.a.
    \[ F(x_1,\dotsc,x_n) = G(H_1(x_1,\dotsc,x_n), \dotsc, H_n(x_1,\dotsc,x_n))\]
    rakið fall.
  \item Ef $G,H$ eru rakin föll og $F$ er skilgreint
    \begin{align*}
      F(x_1, \dotsc, x_n,0) & := G(x_1, \dotsc,x_n),\\
      F(x_1, \dotsc,x_n, yH) & := H(x_1, \dotsc, x_n, y, F(x_1, \dotsc, x_n, y))
    \end{align*}
    rakið fall.
  \item Ef $G$ er rakið fall, þ.a. $\forall x_1, \dotsc, \forall x_n  \exists y (G(x_1, \dotsc, x_n,y) = 0)$
    þá er fallið $F$ þ.a. 
    \[ F(x_1, \dotsc, x_n) := \mu y G(x_1, \dotsc, x_n,y) = 0)\]
    þar sem $\mu y(G(x_1,\dotsc,x_n,y) = 0)$ er minnsta $y$ þ.a.
    $G(x_1,\dotsc,x_n,y) = 0$ er rakið.
  \end{enumerate}

 Fall sem fæst með því að nota eingöngu $R_1, R_2, R_3$ 
 kallast \emph{frumstætt rakið fall}.

 Við segjum að venzl $R$ séu rakin ef kennifallið 
 $c_R$ er rakið. Ef \[\forall x_1, \dotsc, \forall x_n  \exists y R(x_1, \dotsc, x_n,y)\]
 þá skrifum við
 \[\mu y R (x_1, \dotsc, x_n,y) := \mu y (G_R(x_1,\dotsc,x_n,y) = 0)\]
\end{skgr}


\begin{setn}
  \begin{enumerate}(1)
  \item Ef $G: \N^n \to \N$ er [frumstætt9 rakið fall,
    $i_1, \dotsc, i_k \in \set{1, \dotsc, n}$ og $F$ er gefið með
    \[ F(x_1, \dotsc,x_n) = G(x_{i_1}, \dotsc x_{i_k})\]
    þá er $F$ [frumstætt] rakið fall.
  \item Núllfallið $Z^n(x_1,\dotsc,x_n) = 0)$ er frumstætt rakið fall.
  \item Fyrir $k \in \N$ er fastafallið $K^n_k$ þ.a.
    $K^n_k ( \xxn) = k$ frumstætt rakið fall.
  \item í R4 má taka $n=0$ ef $H$ er [frumstætt] rakið fall,
    $k \in \N$ og $F: \N \to \N$ er skilgreint með
    \[ F(0) = k,\]
    \[ F(y+1) = H(y,F(y)) \]
    þá er $F$ [frumstætt] rakið fall.
  \end{enumerate}
\end{setn}


\begin{proof}
  \begin{enumerate}[(1)]
  \item Höfum $F(\xxn) = G(I^n_{i_1} (\xxn), \dotsc, I^n_{i_k}(\xxn))$
  \item $Z^n(\xxn) = Z(I^n_1 ( \xxn ))$.
  \item $K^n_0 (\xxn) = Z^(\xxn)$ og
    $K^n_{k+1} (\xxn) = N(K^n_k (\xxn))$.
  \item Skv. R3 er $G: N^2 \to \N$ þ.a.
    \[ G(x_1, 0) = K^1_k (x_1), \]
    \[ G(x_1, y +1) = H(I^2_2 (x_1, y), G(x_1,y))\]
  \end{enumerate}
  [frumstætt] rakið fall, og $F(y) = G(y,y)$
\end{proof}


\begin{setn}[og skilgreining]
  Eftirfarandi föll eru frumstæð rakin föll.
  \begin{enumerate}[(1)]
  \item $A(x,y) := x+y$.
  \item $M(x,y) := x \cdot y$.
  \item $V(x,y) := x ^ y$.
  \item $ \delta (x) := \bcondef x-1 & \Ef x > 0, \\ 0 & \Ef x = 0. \econdef$
  \item $ x-y := \bcondef x-y & \Ef x \geq y, \\ 0 & \Ef x < y. \econdef$
  \item $ |x-y| := \bcondef x-y & \Ef x \geq y, \\ y - x & \Ef x < y. \econdef$
  \item $ sg(x) := \bcondef 0 & \Ef x = 0 , \\ 1 & \Ef x > 0. \econdef$
  \item $ \overline{sg}(x) := \bcondef 1 & \Ef x = 0 , \\ 0 &  \Ef x > 0. \econdef$
  \item $x!$.
  \item $\min(\xxn)$.
  \item $\max(\xxn)$.
  \item $rm(x,y) = \text{ afgangur þegar x er deilt í y}$.
  \item $qf(x,y) = \text{ kvóti þegar x er deilt í y}$.
  \end{enumerate}
\end{setn}

\begin{proof}
  \begin{enumerate}[(1)]
  \item 
    \begin{gather*}
    A(x,0) = I^1_1(x),\\
    A(x,y+1) = N(A(x,y))
    \end{gather*}
  \item 
    \begin{gather*}
    M(x,0) = Z(x),\\
    M(x,y+1) = A(M(x,y),x)
    \end{gather*}
  \item 
    \begin{gather*}
    V(x,0) = K^1_1(x),\\
    V(x,y+1) = M(V(x,y),x)
    \end{gather*}
  \item 
    \begin{gather*}
    \delta(0) = 0,\\
    \delta(y+1) = y
    \end{gather*}
  \item 
    \begin{gather*}
    x - 0 = x,\\
    x - (y+1) = \delta(x-y)
    \end{gather*}

  \item 
    \begin{gather*}
    |x - y| = A(x-y,y-x),\\
    \end{gather*}

  \item 
    \begin{gather*}
    sg(0) = 0,\\
    sg(y+1) = K^1_1(y)
    \end{gather*}
  \item 
    \begin{gather*}
    \overline{sg}(x) = 1 - sg(x)
    \end{gather*}
  \item
    \begin{gather*}
      0! = 1 = K^1_1(x),\\
      (y+1)! = y! \cdot (y+1) = M(y!,y+1)
    \end{gather*}
  \item 
    \begin{gather*}
      \min(x,y) = x - (x-y), \\
      \min(\xxn, x_{n+1}) = \min( \min( \xxn), x_{n+1})
    \end{gather*}
  \item 
    \begin{gather*}
      \max(x,y) = y + (x-y), \\
      \max(\xxn, x_{n+1}) = \max( \max( \xxn), x_{n+1})
    \end{gather*}
  \item 
    \begin{gather*}
    rm(x,0) = 0,\\
    rm(x,y+1) = N(rm(x,y)) \cdot sg(|x-N(rm(x,y))|)
    \end{gather*}
  \item 
    \begin{gather*}
    qf(x,0) = 0,\\
    qf(x,y+1) = qf(x,y)) + \overline{sg}(|x-N(rm(x,y))|)
    \end{gather*}
  \end{enumerate}
\end{proof}

\begin{setn}
  Ef $F$ er [frumstætt] rakið fall, þá eru eftirfarandi föll [frumstætt] rakin:
  \begin{gather*}
    \sum_{y < z} F(\xxn,y), ( =: G(\xxn,z)) \\
    \sum_{y \leq z} F(\xxn,y),  \\
    \prod_{y < z} F(\xxn,y),  \\
    \prod_{y \leq z} F(\xxn,y),  \\
  \end{gather*}
\end{setn}

\begin{proof}
  \begin{gather*}
    G(\xxn,0) = 0 \\
    G(\xxn, z+1) = G(\xxn,z) + F(\xxn,z)\\
    \vdots
  \end{gather*}
  o.s.frv.
\end{proof}


\begin{daemi}
  Látum $t(0) = 1$ og $t(x)$ vera fjöldi deila
  tölunnar $x$ ef $x \geq 1$. Þá er
  $t$ frumstætt rakið, því að
  \[t(x) = \sum_{y \leq x} \overline{sg}(rm(y,x))\]
\end{daemi}

\begin{skgr}
  Setjum
 \[\mu y_{y< z} R(\xxn,y) := \bcondef \text{ minnsta $y$ þ.a. $y < z$ og $R(\xxn,y)$} & \Ef \text{ slíkt y er til} \\ z & \text{ annars}. \econdef\]
 \[ \forall y_{y<z} R(\xxn,y): \leftrightarrow \forall y ( y < z \rightarrow R(\xxn,y)),\]
 \[ \forall y_{y<z} R(\xxn,y): \leftrightarrow \exists y ( y < z \wedge R(\xxn,y));\]
 hliðstætt fyrir $\mu y_{y \leq z}, \forall y_{y\leq z}, \exists y _{y \leq z}$.
\end{skgr}

\begin{setn}
  Venzl sem fást úr [frumstætt] röknum venzlum með því að nota rökvirkjana
  '$\lnot$', '$\vee$' og takmarkaða magnara  eru [frumstætt] rakin.
  Föll sem fást úr [frumstæðum] röknum föllum með virkjunum
  $\mu y_{y < z}$ og $\mu y_{y \leq z}$ eru [frumstætt] rakin.
\end{setn}

\begin{proof}
  \begin{gather*}
    C_{\lnot R} (\xxn) = 1 - C_R(\xxn). \\
    C_{R_1 \vee R_2} (\xxn) = C_{R_1} (\xxn) \cdot C_{R_2}(\xxn)
  \end{gather*}
  Ef $\theta(\xxn,y): \leftrightarrow \exists y_{y < z} R(\xxn,y)$ þá er
  \[ C_R(\xxn) = \prod_{y<z} C_R(\xxn, y)\]
  $\exists y_{y\leq z}$ er jafngilt $\exists y_{y <  z+1}$, $\forall y_{y\ < z}$
  er jafngilt $\lnot \exists y_{y <  z}$ og
  \[\mu y_{y <  z} R(\xxn, y) = \sum_{y<z} (\prod_{u \leq y} C_R (\xxn, u))\]
  Því að $ \prod_{u < y}C_R(\xxn,u)$ er 1 f. öll $y$ þ.a. $R(\xxn,u)$ sé rangt.

  Fyrir öll $u$, en verður $0$ um leið og til er $u \leq y$ þ.a. $R(\xxn,u)$
  sé rétt. Svo að $\sum_{y<z} \prod_{u \leq y} C_R(\xxn,u)$ er földi allra staka
  frá $0$ upp í $y-1$
  þar sem $y < z$ er fyrsta $y$ þ.a. $R(\xxn, y)$ sé rétt, ef slíkt $y$ er til 
  en jafnt $z$ ef slíkst $y$ er ekki til.
\end{proof}


\begin{setn}
  Venzlin $x = y$, $x < y$, $x \leq y$ , $x > y$, $x \geq y$, $x | y$, $x \equiv y (mod z)$.
  og $Pr(x)$ eru frumstæði rakin, þar sem $Pr(x)$ þyðir að $x$ sé frumtala.
  \begin{proof}
    \begin{gather*}
      C_=(x,y) = sg(|x-y|),\\
      C_<(x,y) = \overline{sg}(y - x),\\
      x \leq y \leftrightarrow x < y \vee x = y\\
      C_|(x,y) = sg(rm(x,y)),\\
      x \equiv y (mod z) \leftrightarrow z | |x-y|\\
      C_{Pr} (x) = sg(|t(x)-2|)\\
    \end{gather*}
    því að $p$ er prímtala þþaa fjöldi talna sem gengur upp í $p$ sé $2$.
  \end{proof}
\end{setn}

\begin{daemi}
  \begin{enumerate}[(1)]
  \item Látup $p_x$ vera $(x+1)$-stu frumtöluna í vaxandi röð. Þá
    er $p_x$ frumstætt rakið fall, því að
    \begin{gather*}
      p_0 = 2,\\
      p_{x+1} = \mu y_{y \leq (p_x)!+1} (p_x < y \wedge Pr(y)).
    \end{gather*}
    sérhverja nátt. tölu $x\geq 1$ má skrifa með
    nákv. einum hætti sem margfeldi
    \[ x = p_0^{v_0(x)} p_1^{v_1(x)} p_2^{v_2(x)} \dotsb \]
    þar sem $v_k(x) \in \N$ o(g $v_k(x) = 0$ f. öll nógu stór k);
    setjum til þæginda
    $v_k(x) := 0$. Sérhvert fall $v_k(x)$ er frumstætt rakið, því að
    \[ v_j(x) = \mu y_{y < x}(p_j^y | x \wedge \lnot p_j^{y+1} | x).\]
  \end{enumerate}
\end{daemi}

\begin{setn}
 Látum $G_1, \dotsc, G_k$ vera [frumstætt] rakin föll og $R_1, \dotsc, R_k$ vera
 [frumstæð] rakin þ.a. fyrir sérhvert $(\xxn)$ sé nákvæmlega eitt
 af $R_1(\xxn), \dotsc, R_k(\xxn)$ satt. Þá er fallið $F$ þ.a.
 \[ F(\xxn) := \bcondef G_1(\xxn) & \Ef R_1(\xxn)\\ \vdots & \vdots \\ G_k(\xxn) & \Ef R_k(\xxn) \econdef \]
 er frumstætt rakið fall.
 \begin{proof}
   \[F = G_1 \cdot C_{\lnot R_1} + \dotsb + G_k \cdot C_{\lnot R_k} \]
 \end{proof}
\end{setn}

Oft eru föll $f$ skilgreind með þrepun þ.a.
í skgr. á $f(y+1)$ eru öll gildi $f(0), \dotsc, f(y)$
notuð. Setjum
\[ f^{\#}(\xxn,y) = \prod_{u < y}p_n^{f(\xxn,u)} \]
Getum reiknað $f$ útfrá $f^{\#}$ með
\[ f(\xxn,y) = v_y (f^{\#}(\xxn,y+1)) \]

\begin{setn}
  Ef $H$ er [frumstætt] rakið fall og f fullnægi
  \[ f(\xxn, y) = H(\xxn,y, f^{\#}(\xxn,y))\]
  þá er f [frumstætt] rakið fall.
\end{setn}

\begin{proof}
  $f^{\#}$ er [frumstætt] rakið, því að
  \begin{align*}
    f^{\#}(\xxn,0) = 1& ,\\
    f^{\#}(\xxn,y+1) &= f^{\#}(\xxn,y) \cdot p_y^{f(\xxn,y)}\\
    &= f^{\#}(\xxn,y) \cdot p_y^{H(\xxn,y,f^{\#}(\xxn,y)))}\\
  \end{align*}
  þar með er $f(\xxn,y) = v_y(f^{\#}(\xxn,y+1))$ það líka.
\end{proof}

\begin{daemi}
  Fibonaci-runan er skilgreind með 
  $f(0) = 1, f(1) = 1$ og \[f(k+2) = f(k)+ f(k+1)\].
  Höfum:
  \[f(k) = \overline{sg}(k) + \overline{sg}(|k-1|) + (v_{k-1}(f^{\#}(k))+v_{k-2}(f^{\#}(k))) \cdot sg(k-2) \]

  þar sem  
  \[ H(y,z) =  \overline{sg}(y) + \overline{sg}(|y-1|) + (v_{y-1}(z) + v_{y-2}(z))\cdot sg(y-1)\]
  er frumstætt rakið fall, svo að $f$ er frumstætt.
\end{daemi}

\begin{setn}[Hjálparsetning]
  Fallið $Q: \N^2 \to \N, Q(x,y) = (x+y)^2 +x + 1$ er eintækt.
  \begin{proof}
    G.r.f. að $Q(x,y) = Q(s,t)$. Ef $x+y < s+t$, þá er
     \[ Q(x,y) \leq (x+y +1)^2 \leq (s+t)^2 < Q(s,t),\]
     sem er mótsögn, eins ef $s+t < x+y$. Því er $x+y = s+t$,
     en þá fæst að $x = s$ og þá $y = t$.
  \end{proof}
\end{setn}

\begin{setn}
  Til er ákveðið frumstætt rakið fall $\beta: \N^2 \to \N$
  þ.a. $\beta (a,i) \leq a - 1$ og þ.a. fyrir öll $a_0, \dotsc, a_{n-1}$ er til
  tala $a$ þ.a.
  \[\beta(a,i) = a_i \]
  fyrir öll $i$ þ.a. $0 \leq i < n$
  \begin{proof}
    Setjum $Q(x,y) := (x+y)^2 + x +1$ og
    \[ \beta(a,i) = \mu x_{x \leq a - 1} \exists y_{y<a} \exists z_{z<a} (a = Q(y,z) \wedge 1 + (Q(x,i)+1) \cdot z | y ) \]
    Látum $a_0, \dotsc, a_{n-1}$ vera fegin. Fáum
    \[ c:= \max (Q(a_0,0) + 1, \dotsc, Q(a_{n-1},n-1) +1 ) \]
    
    Setjum $ z := c!$. Fyrir $j < l < c$ eru tölurnar
    $1 +jz$ og $1 + lz$ ósamþátta, því að sameiginlegur frumþáttur
    $p$ gengur upp í $(1+lz) - (1+jz) = (l-j)z$, en $l-j < c$, svo að
    $l-j | z$ og því $p|z$, sem er fráleitt (þá fengist $p | 1$). Skv.
    kínversku leifasetningunni er til tala $y$ þ.a. fyrir $j = =1, \dotsc, c-1$
    gildi $1+jz|y$ þþaa $j$ sé ein af tölum $Q(a_i,i)+1, i \in \set{0,\dotsc,n-1}$.
    Setjum $a := Q(y,z)$. Þá er $a_i < y < a$ og $ z < a$.

    Skv. HS fæst: Til að sýna að $\beta(a,i) = a_i$ fyrir $i = 0, \dotsc, n-1$
    nægir að sýna að $a_i$ sé minnsta tala $x$ þ.a.
    $1+(Q(x,i)+1)\cdot z$ gangi upp í $y$. En ef $x < a_i$,
    þá er $Q(x,i) < Q(a_i,1) < c$ og $Q(x,i)$ er
    ekki ein af tölum $Q(a_j,i)$, þ.a. $1 + (Q(x,i)+1)\cdot z$
    gengur ekki upp í $y$.
  \end{proof}
\end{setn}

\begin{skgr}
  Köllum $\beta$ Gödel-fallið.

\end{skgr}

\begin{ath}
  
  Vegna $\beta(a,i) \leq a - 1$ er 
  \[\beta(0,i) = 0\]
  og fyrir $a \neq 0$ er $\beta(a,i) < a$.
\end{ath}

\begin{ath}
  Reynir notar í þessum kafla að ofan táknið $\dot{-}$,
  en ég hef táknað það með $-$. Það ætti þó að laga þetta.
\end{ath}


\section{Framsetning í $\mathcal{N}$}

Rifjum upp að $\cN$ er samsemdarkenningin með fasta '$0$', einstætt
fallatákn '$S$', tvö tvístæð fallatákn '$+$' og '$\cdot$' og
eitt tvístætt umsagnartákn '$<$' (auk '$=$') og eftirfarandi eiginlegum
frumsendum:

\begin{enumerate}[label=\textbf{N}\arabic*]
\item  \( Sx \neq 0\)
\item  $ Sx = Sy \rightarrow x = y$.
\item  $ x + 0 = x$.
\item  $ x + Sy = S(x+y)$.
\item  $ x \cdot 0 = 0$.
\item  $ x \cdot Sy = (x \cdot y) + x$.
\item  $ \lnot (x < 0)$.
\item  $ x < Sy \leftrightarrow x < y \vee x = y$.
\item  $ x < y \vee x = y \vee x > y$.
\end{enumerate}

Heitin '$0$', '$S0$', '$SS0$', $\dotsc$ o.s.frv. kallast
tölutákn látum $\mb{k}_n$ vera tölutáknið sem hefur nkvl. $n$ '$S$'.

\begin{skgr}
  Látum $\cT$ vera kenningu með sama máli og $\cN$. Við segjum
  að yrðingin $\bA$ ásamt ólíkum breytum 
  $\bx_1, \dotsc, \bx_n ,\mathbf{y}$ sé framsetning á n-stæðu fallinu
 $F$ í $\cT$ ef fyrir öll $a_1, \dotsc, a_n$ í $\N$ gildir
 \[ \vdash_{\cT} \bA_{\bx_1, \dotsc, \bx_n} [ \bk_{a_1}, \dotsc, \bk_{a_n}] \leftrightarrow y = \bk_b \]
 þar sem $ b = F( a_1, \dotsc, a_n)$. Segjum að $F$ sé framsetjanlegt í
 $\cT$ ef slík framsetning er til.

 Segjum að yrðing $\bA$ ásamt ólíkum breytum $\bxxn$ sé framsetning í $\cT$
 á n-stæðum venzlum $P$ ef f. öll $\aan$ úr $\N$ gildi
 \begin{enumerate}[(i)]
 \item  ef $P(\aan)$, þá $\vdash_{\cT} \bA_{\bxxn}[\bk_{a_1}, \dotsc, \bk_{a_n}]$;
 \item  ef $\lnot P(\aan)$, þá $\vdash_{\cT} \lnot \bA_{\bxxn}[\bk_{a_1}, \dotsc, \bk_{a_n}]$
 \end{enumerate}
 segum þá að $P$ sé \emph{framsetjanlegt} í $\cT$.

 Segjum að heiti $\ba$ ásamt ólíkum breytum $\bxxn$ sé framsetning á 
 n-stæðu falli $F$ í $\cT$ ef f. öll $\aan$ í $\N$
 gildir:
 \[ \vdash_{\cT} \ba_{\bxxn}[\bk_{a_1}, \dotsc, \bk_{a_n}] = \bk_b \]
 þar sem $b = F(\aan)$
\end{skgr}

\begin{ath}
  Ef heitið $\ba$ ásamt $\bxxn$ er framsetning á $F$ í $\cT$ og
  $\by$ er ný breyta, þá er yrðingin $\by = \ba$ ásamt
  $\bxxn, \by$ framsetning á $F$ í $\cT$.
\end{ath}

\begin{daemi}
  \begin{enumerate}[(1)]
  \item  Yrðingin '$ x = y$' ásamt '$x$', '$y$' er framsetning á 
    venzlum $=$ í $\cN$. Til að sanna það þarf að sýna:
    \begin{enumerate}[(i)]
    \item  ef $m = n$, þá er $\bk_m = \bk_n$
    \item  ef $m \neq n$, þá er $\bk_m \neq \bk_n$
    \end{enumerate}

    Staðhæfing (i) fæst úr $\mathbf{EQ}_1$ með innsetningu.
    Til að sýna (ii) má gera ráð fyrir að $m > n$.
    Notum þrepun yfir $n$. Ef $n = 0$, fæst þetta úr \textbf{N1}.
    Látum þá $n > 0$. Með innsetningu í \textbf{N2} fæst
    \[ \vdash_{\cN} \bk_m = \bk_n \rightarrow \bk_{m-1} = \bk_{n-1}.\]
    Skv. þf. er $\vdash_{\cN} \bk_{m-1} \neq \bk_{n-1} $ og þá
    $\vdash_{\cN} \bk_m \neq k_n$ skv. sís.
  \item Heitið '$0$' ásamt '$x$' er framsetning á núllfallinu
    $Z: \N \to \N$, því að $\vdash_{\cN} 0 = 0$ og
    $0_x [\bk_a]$ er '$0$'.
  \item Heitið '$Sx$' ásamt '$x$' er framsetning á eftrifara fallinu
    $N(x) = x + 1$ í $\cN$; það þýðir að
    \[ \vdash_{\cN} S \bk_n = \bk_{n+1}.\]
    En $S\bk_n = \bk_{n+1}$, svo að þetta fæst úr
    $\mathbf{EQ}_1$ með innsetningu.
  \item Heitið '$x_i$' ásamt '$x_1$',$\dotsc$,'$x_n$' er
    framsetning á ofanvarpinu $I^n_i$. Ef
    $\bA$ er $'x_i'$, þá er $\bA_{\bxxn}[\bk_{a_1}, \dotsc, \bk_{a_n}]$
    heitið $\bk_{a_i}$ og 
    \[ \vdash_{\cN} \bk_{a_i} = \bk_{a_i} \]
    skv. \textbf{EQ1}
  \item Heitið $'x+y'$ ásamt $'x', 'y'$ er framsetning í $\cN$ á
    samlagningunni $\N^2 \to \N$,
    $(x,y) \mapsto x+y$. Til að sanna það þarf að sýna að
    \[ \vdash_{\cN} \bk_m + \bk_n = \bk_{m+n} \]
    Þrepum yfir $n$. Fyrir $ n =0$ er þetta
    \[ \vdash_{\cN} \bk_m + 0 = \bk_{m} ,\]
    sem fæst úr \textbf{N3} með innsetningu.
    G.r.f. að $\vdash_{\cN} \bk_m + \bk_n = \bk_{m+n}$
    Fyrir fast tiltekið $n$ (og öll $m$). Skv.
    setningu um samsendarkenningar
    \[ \vdash_{\cN} S(\bk_m + \bk_n) = S \bk_{m+n}\]
    þ.e.

    \[ \vdash_{\cN} S(\bk_m + \bk_n) = \bk_{m+n+1}\]
    skv. N4 er
    \[ \vdash_{\cN} S(\bk_m + \bk_n) = \bk_{m} + \bk_{n+1}\]
    svo að
    \[ \vdash_{\cN} \bk_m + \bk_{n+1} = \bk_{m+n+1}.\]
    Eins sést að tilteki '$x\cdot y$' ásamt $'x'$, '$y$' er
    framsetning  á margfölduninni $\N^2 \to \N$ í $\cN$.
  \item  Yrðingin $'x < y'$ ásamt '$x$' og '$y$' er
    framsetning í $\cN$ á venzlunum $<$ í $\N^2$. Sýna þarf:
    \begin{enumerate}[(i)]
    \item  Ef $m < n$ ,þá er $\vdash_{\cN} \bk_m < \bk_n$
    \item  Ef $m \geq n$ ,þá er $\vdash_{\cN} \lnot (\bk_m < \bk_n)$ 
    \end{enumerate}
    Fyrir $n = 0$ gildir (i) sjálfkrafa og (ii) fæst úr \textbf{N7}.
    G.r.f að (i) og (ii) gildi fyrir tiltekið $n$. skv. \textbf{N8}
    fæst
    \[ \vdash_{\cN} \bk_m < \bk_{n+1} \leftrightarrow (\bk_m < \bk_n \vee \bk_m = \bk_n) (^*) \]
    Gerum fyrst ráð fyrir að $m < n+1$. Ef $m<n$ þá er $\vdash_{\cN} \bk_m < \bk_n$
    skv. þf. en ef $ m = n$, þá er $\vdash_{\cN} \bk_m = \bk_n$.

    Í báðum tilvikum gefur $(^*)$ okkur að $\vdash_{\cN} \bk_m < \bk_{n+1}$.
    Gerum næst ráð fyrir að $m \geq n+1$. Þá er $m \geq n$
    því $\vdash_{\cN} \lnot (\bk_m < \bk_n)$ skv. þf. og 
    $m \neq n$ og því $\vdash_{\cN} \lnot (\bk_m < \bk_{n+1})$
    skv. $(^*)$ og \emph{sís}.
  \end{enumerate}
\end{daemi}

\begin{setn}
  Venzl P eru framsetjanleg í $\cN$ þþaa kennifallið $C_P$ sé framsetjanlegt í $\cN$
\end{setn}

\begin{proof}
  Gerum fyrst r.f. að $\bA$ ásamt $\bxxn$ sé framsetning á $P$ í $\cN$
  látum $\bB$ vera yrðinguna
  \[ ( \bA \wedge \by = \bk_0) \vee (\lnot \bA \wedge \by = \bk_1)\]
  Þar sem $\by$ er ný breyta. sýnum að $\bB$
  ásamt $\bxxn, \by$ er framsetning á $C_P$.
  G.r.f. að $C_P (\aan) = 0$. Þá er $P(\aan)$
  og því $\vdash_{\cN} \bA_{\bxxn} [\bk_{a_1}, \dotsc,\bk_{a_n}]$; af því og \emph{sís}
  fæst
  \[ \vdash_{\cN} \bB_{\bxxn} [ \bk_{a_1}, \dotsc,\bk_{a_n} ] \leftrightarrow \by = \bk_0 \]
  Samskonar rök ef $C_P (\aan) = 1$.

  Gerum næst r.f. að $\bA$ ásamt $\bxxn, y$ sé framsetning á $C_P$. Sýnum að
  $\bA_{\by}[0]$ ásamt $\bxxn$ er framsetning á $P$.
  \begin{enumerate}[(i)]
  \item Ef $P(\aan)$, þá er $C_P(\aan) = 0$ og því
    \[\vdash_{\cN} \bA_{\bxxn}[\bk_{a_1}, \dotsc,\bk_{a_n}] \leftrightarrow \by = \bk_0 \]
    Setjum $\bk_0$ inn fyrir $\by$. Vegna $\vdash_{\cN} \bk_0 = \bk_0$
    fæst \[\vdash_{\cN} ( \bA_{\by}[0] )_{\bxxn}[\bk_{a_1}, \dotsc,\bk_{a_n}]\]
  \item Ef $\lnot P(\aan)$ þá er $C_P ( \aan) = 1$
    og \[\vdash_{\cN} \bA_{\bxxn} [ \bk_{a_1}, \dotsc,\bk_{a_n}] \leftrightarrow \by = \bk_1.\]
    Setjum $\bk_0$ inn fyrir $\by$, vegna $\vdash_{\cN} k_0 = k_1$, fæst
    \[ \vdash_{\cN} \lnot (\bA_{\by}[0])_{\bxxn} [\bk_{a_1}, \dotsc,\bk_{a_n}]\]
  \end{enumerate}
\end{proof}












\[\vdots\]


\begin{proof}
  \[\vdash_{\cN} \bk_b < \bk_c \wedge \bA_{\bxxn} [ \bk_{a_1},
  \dotsc,\bk_{a_n},\bk_b] .\] Af innsetningarsetningu leiðir að
  \[\vdash_{\cN} \exists y ( y < \bk_b \wedge \bA_{\bxxn} [ \bk_{a_1},
  \dotsc,\bk_{a_n}])\] þ.e.
 
  \[\vdash_{\cN} \bB_{\bxxn,\mb{w}} [ \bk_{a_1},
  \dotsc,\bk_{a_n},\bk_b] .\]

  G.r.f. að $\lnot S(\aan,c)$. Þá gildir $\lnot R(\aan, i)$ fyrir öll
  $i = 1, \dotsc, c-1$.

  Og því að
  \[\vdash_{\cN} \lnot \bA_{\bxxn,\by} [ \bk_{a_1},
  \dotsc,\bk_{a_n},\bk_i]\] fyrir öll $i = 1, \dotsc, c-1$.  Ef $\bC$
  er \(\vdash_{\cN}\bA_{\bxxn} [ \bk_{a_1}, \dotsc,\bk_{a_n}]\) má
  skrifa þetta
  \[\vdash_{\cN} \lnot \bC_{y} [ \bk_{i}] , i = 0, \dotsc, c-1 \]

  skv. HS2 er

  \[\vdash_{\cN} \lnot (y < \bk_c \wedge \bC)\]

  Skv. \textbf{Ath} fæst

  \[ \vdash_{\cN} \forall y \lnot (y < \bk_c \wedge \bC) \] þ.e.

  \[ \vdash_{\cN} \lnot \exists y (y < \bk_c \wedge \bC) \]

  og það þýðir:

  \[ \vdash_{\cN} \lnot \bB_{\bxxn} [\bkaan, \bk_c] \]
\end{proof}


\begin{setn}[HS7]
  Gödelfallið $\beta$ er framsetjanlegt í $\cN$.
\end{setn}
\begin{proof}
  Munum að
  \[ \beta(a,i) = \mu x_{\bx \leq \dot{-} 1} \exists y_{y<a} \exists z_{z < a}\]

\[\vdots\]

Ljóst er að $Q$.
\[\vdots\]

eru framsetjanleg í $\cN$ og þá er $\beta(a,i) = \mu x R(x,i)$
framsetjanlegt í $\cN$.
\end{proof}

\begin{setn}[HS8]

Látum $G,H$ vera framsetjanlegt í $\cN$, og $F$ vera skilgreint með
\begin{gather*}
  F(\xxn,0) = G( \xxn)\\
  F(\xxn, y+1) = H(\xxn,F(\xxn,y))
\end{gather*}
þá er $F$ framsetjanlegt í $\cN$.
\end{setn}

\begin{proof}
  Látum $\bA$ ásamt $\bxxn, \by$ vera framsetningu á $G$ í $\cN$
  og $\bC$ ásamt $\bxxn,\by,\bz,\bv$ vera framsetningu á Gödelfallinnu $\beta$
  í $\cN$. Látum $\bD$ vera

  \begin{gather*}
    \exists \mb{u} [ \exists \mb{w} ( \bB_{\mb{u},\mb{y},\mb{z}} [
    \mb{u}, 0, \mb{w}] \wedge \bA_y [w]) \wedge
    \bB_{\mb{u},\mb{y},\mb{z}}[\mb{u},\mb{y},\mb{z}]\\ \wedge
    \exists \mb{w} (\mb{w} < \by \rightarrow \exists \mb{t} \exists \mb{s} ( \bB_{\mb{y},\mb{z}}[\mb{w};\mb{t}] \wedge \bB_{\mb{y},\mb{z}}[S\mb{w}, \mb{s}] \wedge C_{\mb{u},\mb{y},\mb{z}}[\mb{w}, \mb{t}, \mb{s}]]]]
  \end{gather*}

  Viljum sýna að $\bD$ ásamt $\bxxn, \by, \bz $ sé framsetning á $F$ í $\cN$
  þ.e.: Ef $F(\aan,b) = c$ þá er
  \[\vdash_{\cN} \bD_{\bxxn,\by} [\bkaan, \bk_b] \leftrightarrow \bz = \bk_c\]
  Við gefum óformlega sönnun og látum lesanda eftir að þýða yfir í $\cN$.
  Með því að nota $\bA, \bB$ og $\bC$ sem yrðingar um Föllin
  $G, \beta, H$ og skrifa $'z'$ fyrir $\bz$ sést að 
  yrðingin $\bD_{\bxxn,y}[\bkaan,\bk_b]$ þýðir að eftirfarandi skilyrðum sé fullnægt.
  \begin{enumerate}[(i)]
  \item Til er w þ.a. 
    \[ \beta(u,0) = w \]
    og \[G(\aan) = w\]
  \item $\beta(u,b) = z$

  \item fyrir öll $w < b$ eru til $t$ og $s$ þ.a.

    \begin{gather*}
      t = \beta(u,w), s = \beta(u,w+1)\\
      s = H(\aan,w,t)
    \end{gather*}
  \end{enumerate}

  Með þí að setja $d_i = \beta(u,i)$ sést að þetta segir eftirfarandi:

  \[ (*) \bcondef d_0 = C(\aan) \\ d_{i+1} = H(\aan,i,d_i) \text{ fyrir } i < 0
  \\ d_b = z \econdef \]
  En af því sést að $d_i = F(\aan,i)$ og þá $z = F(\aan,b) = c$.
  Öfugt ef $z := F(\aan,b)$ og skilgreinum $d_i := F(\aan,i)$,
  þá fullnægja $k_i$-in $(*)$. Látum u vera
  $\beta(u,i) = d_i$ fyrir $i = 0, \dotsc,b$, þá er skilyrði (i)-(iii) fullnægt.

\end{proof}
Þar með er aðal setningin sönnuð!

\chapter{Ófullkomleikasetningar}

\section{Gödel-tölusetning}

Látum $\beta$ vera Gödel-fallið. Fyrir sérhverja 
n-un ($\aan$) táknum við með
\[< \aan > \]

minnstu tölu $a$ þ.a. $\beta(a,0) = n$ og 
$\beta(a,i) = a_i$ fyrir $i = 1, \dotsc, n$. Köllum
$<\aan>$ runutölu $n$-undarinnar
$(\aan)$. Leyfum $n=0$ og höfum $<> = 0$.

Fyrir fast $n$ er 
\[ F(\aan) := < \aan>\]
rakið fallið því að
\[< \aan> = \mu x(\beta(x,0) = n \wedge \beta(x,1) = a_1 \wedge \dotsb \wedge \beta(x,n) = a_n )\]

Setjum líka

\begin{gather*}
  ln(a) = \beta(a,0)\\
  (a)_i = \beta(a,i+1)
\end{gather*}
(Vegnn $\beta(a,i) \leq a \dda -1$ sést:
Ef $a \neq < >$, þá er $ln(a) < a$ og $(a)_i < a$.)
Ef $a = < a_0 \dotsc, a_{n-1}$, þá er
\begin{gather*}
  n = ln(a)\\
  (a)_i = a_i \text{ fyrir } i = 0, \dotsc, n-1
\end{gather*}

Látum Seq vera mengi allra rauntalna. Það er rakin einstæð venzl vegna
$Seq(a) \leftrightarrow \forall x_{x < a} ( ln(x) \neq ln(a) \vee \exists i_{i < ln(a)} ((x)_i = (a)_i))$.

Látum nú $\cT$ vera kenningu fyrstu stéttar með mál $\cL$.

Komum okkur saman um eftirfarandi:

\begin{enumerate}[(1)]
\item Breytunum í $\cL$ hefur verið raðað í ákveðna röð
  \[ \bz_0, \bz_1, \bz_2, \dotsc\]
  Úthlutum breytunni $\bz_i$ tölunni $2i$; köllum $2i$
  \emph{tákntölu} breytunnar $\bz_i$ og skrifum
  \[\tau(\bz_i) = 2i\]
\item Sérhverju öðru táknu $\bv$ í $\cL$ hefur verið úthlutað
  tiltekinni oddatölu $\tau(\bv)$ þ.a. ólík tákn fái ólíkar tölur.
  Köllum $\tau(\bv)$ tákntölu táknsins $\bv$.
\item Eftirfarandi föll eru rakin:
  \begin{itemize}
  \item \[\phi(x,n): \leftrightarrow x \text{ er tákntala n-stæðs fallatákns  í } \cL\]
  \item \[\omega(x,n): \leftrightarrow x \text{ er tákntala n-stæðs umsagnartákns í } \cL\]
  \end{itemize}

  \begin{ath}
   T.d. má setja 
   \[ \tau{\lnot} := 3, \tau(v) := 5, \tau(\forall) := 7 \]
   Ef við getum raðað fyrir hvert n n-stæða fallatáknunum í röði
   \[\mb{f}_0^n, \mb{f}_1^n, \dotsc \]
   (endanlega eða óendanlega), þá má setja
   \[ \tau(\mb{f}_k^n) := 1 + 8\cdot 2^n 3^k \]
   Eins ef raða má n-stæðu umsagnartáknum í röð
   \[\mb{p}_0^n, \mb{p}_1^n, \dotsc \]
   þá má setja
   \[ \tau(\mb{p}_k^n) :=3 + 8\cdot 2^n 3^k \]

   Næst úthlutum við sérhverju heiti eða
   yrðingu $\bu$ úr $\cL$ náttúrulegri tölu
   sem við skrifum:
   \[ \god{\bu}, \]
   sem við köllum \emph{Gödel-tölu} $\bu$
   með þrepun.
  \end{ath}
\end{enumerate}


\begin{setn}[Gödel-tölusetning]
  \begin{enumerate}[(1)]
  \item  Ef $\bx$ er breyta, sem við lítum á sem heiti, þ.e.
    runu af lengf 1, þá er Gödel-talan
    \[ \god{\bx} =  \braket{\tau(\bx)} \]
  \item Ef $\mb{f}$ er n-stætt fallatákn og
    $\baan$ eru heiti, þá er
    \[\god{\mb{f} \baan} = \braket{\tau(\mb{f}), \god{\ba_1}, \dotsc, \god{\ba_n}}\]
  \item Ef $\mb{p}$ er n-stætt umsagnartákn, og $\baan$ eru heiti, þá er
    \[\god{\mb{p} \baan} = \braket{\tau(\mb{p}), \god{\ba_1}, \dotsc, \god{\ba_n}}\]
  \item 
    Ef $\bA$ er yrðing, þá er
    \[\god{\lnot \bA} = \braket{\tau(\lnot),\god{\bA}}\]
  \item Ef $\bA$ og $\bB$ eru yrðingar, þá er
    \[\god{\vee \bA \bB} = \braket{\tau(\vee),\god{\bA},\god{\bB}}\]
  \item 
    Ef $\bA$ er yrðing, þá er
    \[\god{\forall \bx \bA} = \braket{\tau(\forall),\god{\bx},\god{\bA}}\]
  \end{enumerate}
  Þarsem föllin $\braket{\aan}$ eru reiknanleg getum við reiknað
  $\god{\mb{n}}$ ef tákntölur allra tákna eru gefnar. Getum ákvarðað
  hvort gefin tala $a$ er Gödel-talal heitis eða yrðingar. Þetta má sjá með
  þrepun yfir $a$.
  
  Athugum fyrst hvort a er runu tala $\neq \braket{}$; það getum við, því að
  $Seq$ er rakið og því reiknanlegt.

  Ef a er ekki runutala, þá er a ekki Gödel-tala heitis né yrðingar.  Ef a er
  runutala, þá
  finnum við tölurnar $a_0, \aan$ þ.a. $a = \braket{a_0, \aan}$
  sem við getum af því að $lh$ og $ a \vdash (a)$; [hérna gæti staðið 
  $a \to (a)$, sé það ekki.] eru reiknanleg.
  Athugum hvort eitt af eftirfarandi gildir:
  \begin{enumerate}[(1)]
  \item  $\ba_0$ er jöfn tala.
  \item $\ba_0$ er tákntala n-stæðs fallatákns og $\aan$
    Gödel-tölur heita; þetta er unnt skv. þf, því að
    $a_T < a$ og fallið $\phi(x,n)$ er reiknanlegt.
  \item $a_0$ er tákntala n-stæðs umsagnar tákns og
    $\aan$ eru Gödel-tölur heita
  \item $a_0 = \tau(\lnot)$, $n = 1$ og $a_n$ er Gödel-tala yrðingar.
  \item $a_0 = \tau(\vee)$, $n=2$ og $a_1, a_2$ eru Gödel-tölur yrðinga.

  \item $a_0 = \tau(\forall)$, $n = 2$, $a_1$ er jöfn tala og 
    $a_2$ er Gödel-tala yrðingar.

    
  \end{enumerate}
    Ef einu af þessum skilyrðum er fullnægt, þá er a Gödel-tala heitis eða
    yrðingar, annars ekki.
\end{setn}

\begin{skgr}
 
  Látum $\cT$ vera kenningu fyrstu stéttar á máli $\cL$. Táknum með
 \[\Thm_{\cT}\]
 megi allra Gödel-talna allara setninga í $\cT$.
 Segjum að $\cT$ sé \emph{ákvarðanleg} ef $\Thm_{\cT}$
 er rakið.
\end{skgr}

Skilgreinum nokkur venzl og föll; af skilgreiningunum sést
að þau eru rakin.
\begin{enumerate}
\item \[Var(a) \leftrightarrow a = \braket{(a)_0} \wedge \exists y_{y \leq a} ((a)_0 = 2y)\]
  \emph{Ath}. $Var(a)$ þýðir að $a = \god{\bx}$ fyrir einhverja
  \emph{breytu} $\bx$
\item \[Term(a) \leftrightarrow a = Var(a) \vee [Seq(a) \wedge \phi((a)_0, lh(a) \dda 1) \wedge \forall u_{u < lh(a) \dda 1} Term((a)_{n+1}) ]\]
  \emph{Ath}. $Term(a)$ þýðir annaðhvort að  $a = \god{\bx}$ fyrir einhverja
  \emph{breytu} $\bx$, eða að $a$ sé runu tala $a = \braket{x_0, \xxn}$
  þar sem $x_0$ er tákntala n-stæðs fallatákns og $\xxn$ eru Göde-tölur heita;
  m.ö.o. $Term(a)$ þýðir að til sé $\ba$ þ.a. $a = \god{\ba}$. Term er rakið,
  því að í skilgr. koma aðeins fyrir $Term(b)$ með $b<a$.
\item  \[Afor(a) \leftrightarrow Seq(a) \wedge \omega((a)_0, lh(a) \dda 1) \wedge \forall u_{u < lh(a) \dda 1} Term((a)_{n+1}) \]
  \emph{Ath}. $Afor(a)$ þýðir að $a$ er Gödel-tala grunnyrðingar.

\item 
  \begin{gather*}
    For(a) \leftrightarrow Afor(a) \vee [Seq(a) \wedge lh(a) = 2 \wedge (a)_0 = \tau(\lnot) \wedge For((a)_1) ]\\
    \vee [Seq(a) \wedge lh(a) = 3 \wedge (a)_0 = \tau(\vee) \wedge For((a)_1) \wedge For((a)_2) ]\\
    \vee [Seq(a) \wedge lh(a) = 3 \wedge (a)_0 = \tau(\forall) \wedge Var((a)_1) \wedge For((a)_2) ]\\
  \end{gather*}
  \emph{Ath}. $For(a)$ þýðir að að sé gödel tala yrðingar.
\item 

\[ Sub(a,b,c) \bcondef
c & \Ef Var(a) = b\\
\braket{(a)_0, Sub((a)_1,b,c)} & \Ef a = \braket{(a)_0,(a)_1}\\
\braket{(a)_0, Sub((a)_1,b,c),Sub((a)_2,b,c)} & \Ef a = \braket{(a)_0,(a)_1,(a)_2} \Og (a)_0 \neq \tau(\forall)\\
\braket{(a)_0,(a)_1,Sub((a)_2,b,c)} & \Ef a = \braket{\tau(\forall),(a)_1,(a)_2} \Og (a)_1 \neq b\\
a & \text{annars}
 \econdef \]

\emph{Ath}. 
$Sub(\god{\ba},\god{\bx},\god{\bb}) = \god{\ba_{\bX} [\bb]}$
og
$Sub(\god{\bA},\god{\bx},\god{\ba}) = \god{\bA_{\bX} [\ba]}$
$\bx$ er bretya, $\ba, \bb$ eru heiti og $\bA$ yrðing.

\item 
\[Fr(a,b) \leftrightarrow \bcondef a = b & \Ef Var(a),\\
Fr((a)_1,b) & \Ef a = \braket{(a)_0, (a)_1},\\
Fr((a)_1,b) \vee Fr((a)_2,b) & \Ef a = \braket{(a)_0, (a)_1,(a)_2} \wedge (a)_0 \neq \tau(\forall),\\
Fr((a)_2,b) \wedge (a)_1 = b & \text{annars}
\econdef
\]

\emph{Ath.} $Fr(\god{\bA},\god{\bx})$ þýðir að $\bx$ komi fyrir frjálst í $\bA$.

\item 

\[
Subtl(a,b,c) \leftrightarrow \bcondef
Subtl((a)_1,b,c) & \Ef a = \braket{(a)_0, (a)_1}\\
Subtl((a)_1,b,c) \wedge Subtl((a)_2,b,c) & \Ef a = \braket{(a)_0, (a)_1, (a)_2} \\
& \wedge (a)_0 \neq \tau(\forall)\\
Subtl((a)_2,b,c) \wedge  \lnot Fr((a)_2,b) \vee \lnot Fr(c,(a)_1) & \Ef a = \braket{\tau(\forall), (a)_1, (a)_2}\\
& \wedge (a)_1 \neq b,\\
0 = 0 & \text{annars}
\econdef
\]

\emph{Ath}. $Subtl(\god{\bA}, \god{\bx}, \god{\ba})$ þýði að
$\bA$ sé innsetjanlegt fyrir $\bx$ í $\bA$.

\item 

\[
Ax_1(a) \leftrightarrow \exists x_{x < a} (For(x) \wedge a = \braket{\tau(\vee), \braket{\tau(\lnot),\braket{\tau(\vee),x,x}},x})
\]

\emph{Ath}. $Ax_1(a)$ þýðir að $a$ sé Gödel-tala frumsendu af gerðinni
\[ \bA \vee \bA \to \bA\] í pólskum rithætti. $\vee \lnot \vee \bA \bA \bA$

\item 
\[
Ax_2(a) \leftrightarrow \exists x_{x < a} \exists y_{y < a} (For(x) \wedge For(y) \wedge
 a = \braket{\tau(\vee), \braket{\tau(\lnot),x} \braket{\tau(\vee),y,x}})
\]

\item 
  \begin{gather*}
    Ax_3(a) \leftrightarrow \exists x_{x < a} \exists y_{y < a} \exists z_{z < a} (For(x) \wedge For(y) \wedge For(z)\\
    \wedge a = \braket{\tau(\vee), \braket{\tau(\lnot),\braket{\tau(\vee), \braket{\tau(\lnot),x},y}},\\
      \braket{\tau(\vee), \braket{\tau(\lnot), \braket{\tau(\vee),z,a}}, \braket{\tau(\vee),y,z}}})
\end{gather*}


\item 
  \begin{gather*}
    Ax_4(a) \leftrightarrow \exists x_{x < a} \exists y_{y < a} \exists z_{z < a}\\
    ( Var(x) \wedge For(y) \wedge Term(z) \wedge Subtl(y,x,z) \wedge\\
    a = \braket{\tau(\vee),\braket{\tau(\lnot), \braket{ \tau(\forall),x,y}}, Sub(y,x,z)}
    )
\end{gather*}

\item 
  \begin{gather*}
    Ax_5(a) \leftrightarrow \exists x_{x < a} \exists y_{y < a} \exists z_{z < a}\\
    ( Var(x) \wedge For(y) \wedge For(z) \wedge \lnot Fr(y,x) 
    \wedge a = \braket{\tau(\vee), \braket{\tau(\lnot),\\
        \braket{\tau(\forall),x, \braket{\tau(\vee),y,z}}}, \braket{\tau(\vee),y, \braket{\tau(\forall),x,z}}}).
\end{gather*}

\item  
\[ Eq_1(a) \leftrightarrow \exists_{x < a} (Var(x) \wedge a = \braket{\tau(=),x,x} \]

\end{enumerate}



\end{document}