\documentclass[12pt]{article}

% not needed with polyglossia
\usepackage[utf8]{inputenc}
\usepackage[T1]{fontenc}

%\usepackage{polyglossia}
%\setdefaultlanguage{icelandic}


\usepackage{graphics,amsmath,amsfonts,amsbsy,amssymb,amsthm}
\usepackage{fancyvrb}
\usepackage[a4paper]{geometry}
\usepackage{graphicx}
\usepackage{hyperref}
\usepackage{datatool}
\usepackage{float}
\usepackage{mdframed}
\usepackage{listingsutf8}
\usepackage{enumerate}
\usepackage{comment}
\usepackage{epstopdf}
\usepackage{caption}
\usepackage{subcaption}
\usepackage{tikz}
\usepackage{enumitem}
\usepackage{mathtools}
\usepackage{tabu}

\usepackage{accents}

\setlength{\parskip}{8pt plus 1pt minus 1pt}
%Verdur ad vera her, sumir pakkar dependa a thetta.
\usepackage[icelandic]{babel}

%viljum ekki númeraða kafla á dæmum

\newcommand{\nonums}{\setcounter{secnumdepth}{-1}}

%flýtiskipanir
\newcommand{\e}{\textbf}
\newcommand{\R}{\mathbb{R}}

\newcommand{\X}{\mathbb{X}}
\newcommand{\Y}{\mathbb{Y}}


%\newcommand{\R}{\Real}
%\newcommand{\C}{\Complex}
%\newcommand{\Z}{\Integer}
%\newcommand{\N}{\Natural}
%\newcommand{\Q}{\Rational}

\newcommand{\K}{\mathbb{K}}
\newcommand{\C}{\mathbb{C}}
\newcommand{\Con}{\mathcal{C}}
\newcommand{\Z}{\mathbb{Z}}
\newcommand{\N}{\mathbb{N}}
\newcommand{\Q}{\mathbb{Q}}
\newcommand{\f}{\frac}
\newcommand{\1}{\frac{1}}
\newcommand{\eps}{\f{\epsilon}}
\newcommand{\Lra}{\Leftrightarrow}
\newcommand{\Th}{\text{ þegar }}
\newcommand{\Ef}{\text{ ef }}
\newcommand{\Og}{\text{ og }}


\newcommand{\inner}[1]{\accentset{\circ}{#1}}
\newcommand{\eR}{\widetilde{\R}}

\newcommand{\com}[1]{\set{\text{#1}}}
\newcommand{\Com}[1]{\set{\text{Athsmd: \text{#1}}}}

\newcommand{\ub}[2]{\underbrace{#1}_{\text{#2}}}
\newcommand{\ubt}[2]{$\ub{\text{#1}}{#2}$}


\newenvironment{inum}{\begin{enumerate}[label=(\roman*).]}{\end{enumerate}}
\newenvironment{anum}{\begin{enumerate}[label=(\alph*).]}{\end{enumerate}}


\newcommand{\bcondef}{\left\{ \begin{array}{l l}}
\newcommand{\econdef}{\end{array} \right.}
\DeclarePairedDelimiter{\condef}{\bcondef}{\econdef}

\DeclarePairedDelimiter{\ceil}{\lceil}{\rceil}
\DeclarePairedDelimiter{\floor}{\lfloor}{\rfloor}
\DeclarePairedDelimiter{\set}{\{}{\}}
\DeclarePairedDelimiter{\braket}{\langle}{\rangle}


\newenvironment{lausn}{\begin{proof}[Lausn]}{\end{proof}}

\newcommand{\sep}{\;|\;}

\newcommand{\fig}[2]{
\begin{figure}[H]
  \centering
  \includegraphics{#1}
  \caption{#2}
  \label{fig:#1}
\end{figure}
}


\newtheorem*{setn}{Setning}
\newtheorem*{hsetn}{Hjálparsetning}
\lstset{  literate={á}{{\'a}}1
                  {ó}{{\'o}}1
                  {ú}{{\'u}}1
                  {ð}{{\dh}}1
                  {í}{{\'i}}1
                  {é}{{\'e}}1
                  {ö}{{\"o}}1
                  {þ}{{\th}}1
                  {æ}{{\ae}}1
                  {ý}{{\'y}}1
                  {Á}{{\'A}}1
                  {Ó}{{\'O}}1
                  {Ú}{{\'U}}1
                  {Ð}{{\DH}}1
                  {Í}{{\'I}}1
                  {É}{{\'E}}1
                  {Ö}{{\"O}}1
                  {Þ}{{\TH}}1
                  {Æ}{{\AE}}1
                  {Ý}{{\'Y}}1}


\theoremstyle{definition}
\newtheorem*{skgr}{Skilgreining}
\newtheorem*{daemi}{Dæmi}
\newtheorem*{frumsenda}{Frumsenda}

\theoremstyle{remark}
\newtheorem*{ath}{Athugasemd}


\title{Rökfræði}
\author{Matthías Páll Gissurarson}

\begin{document}
\maketitle


% \section{Rökfræði}\label{ruxf6kfruxe6uxf0i}

% \subsection{Athugasemdir um texta}\label{athugasemdir-um-texta}

% Stærðfræði jöfnur eru ritaðar á LaTeX mátann, umlukið hornklofum, þ.e.
% {[} x + 2 = 4 {]}. Innskot í texta eru táknuð með pípum, þ.e. \textbar{}
% innskot \textbar{}. Orð sem sérstök áhersla er lögð á er táknað með
% stjörnum, þ.e. \emph{orð}. Feitletruð tákn og orð eru tvíundirstrikuð á
% töflu, en táknuð með feitletrun hér, t.d. \textbf{breyta}. Texta sem
% ætlað er til að sé lesinn beint í LaTeX er táknaður á milli tveggja
% samsemdarmerkja með skástriki fyrir framan, þ.e. =

% \begin{gather} 2 + x \end{gather}

% =.

% \subsection{Formáli um texta.}\label{formuxe1li-um-texta.}

% \subsubsection{Breytur, rökbreytur og
% stæður}\label{breytur-ruxf6kbreytur-og-stuxe6uxf0ur}

% Við tölum um mál sem notuð eru í stærðfræði, sem við köllum
% \emph{viðfangsmál} á máli sem við köllum \emph{yfirmál}. Viðfangsmálið
% notar \emph{breytur}, þ.e. bókstafi sem eru notaðir sem heiti. Þurfum
% nöfn yfir bókstafina. Nafn bókstafs fæst með því að setja hann í
% einfaldar enskar gæsalappir. Þegar að við segjum að x sé talan þ.a. {[}2
% + x = 4 {]}, þá er `x' heitið sem við gefum óþekktu stærðinni x í dæminu
% `x' er þriðji síðasti stafurinn í latneska stafrófinu, en x er það ekki.

% Búum til \emph{táknasamstæður}, stytt í \emph{stæður}, með því að skrifa
% bókstafi og sérstök tákn hvert á eftir öðru, og nafn slíkrar stæðu fæst
% með því að setja einfaldar enskar gæsalappir utan um stæðuna. Til dæmis
% er `tala' orðið sem fæst með því að skrifa fyrst bókstafinn `t', næst
% bókstafinn `a', svo bókstafinn `l' og loks bókstafinn `a'. \textbar{}
% Þetta sýnir að bókstafur getur komið tvisvar fyrir í stæðu og röðin
% skiptir máli \textbar{}. Líka er `x' heiti óþekktu stærðarinnar `x' í
% jöfnunni `2 + x = 4', og í þeirri jöfnu eru notuð tvö sérstök tákn,
% samlagningar merkið `+' og samasemmerkið `='; og í jöfnunni `4 - (3-1) =
% 2' koma fyrir frádáttarmerkið `-' og vinstri sviginn `(' og hægri
% sviginn `)'. Getum líka sagt að stæðan `)(+-()2++++' sé óskiljanlega
% mynduð og merkingarlaus. \textbar{} Athugið að við notum hérna `' til að
% bæta læsileika, en það er þó ekki tekið með í þessu. \textbar{}

% Þutfum breytur á yfirmálinu, t.d. til að tala um almennar breytur (og
% fleira) í viðfangsmálinu. Breyta á yfirmálinu eru feitletraðir bókstafir
% (undirstrikaðir á töflu). Getum sagt: Látum \textbf{a} og \textbf{b}
% vera bókstafi. Þá er \textbf{ab} stæðan sem fæst með því að skrifa fyrst
% stafinn \textbf{a} og svo stafinn \textbf{b}. Ef \textbf{a} er `a' og
% \textbf{b} er `b', þá er \textbf{ab} stæðan `ab'. Ef \textbf{a} er `b'
% og \textbf{b} er `a', þá er \textbf{ab} stæðan `ba'. Ef \textbf{a} er
% `x' og \textbf{b} er `x', þá er stæðan \textbf{ab} stæðan `xx'.

% Ef \textbf{A}, \textbf{B} eru stæður, þá er \textbf{AB} stæðan sem fæst
% með því að skrifa stæðuna \textbf{A} og strax á eftir stæðuna
% \textbf{B}. Notum sama tákn fyrir bókstaf \textbf{a} og stæðuna
% \textbf{a} sem hefur bara einn bókstaf, nefnilega bókstafinn \textbf{a}.
% Ef \textbf{a} er `a' og \textbf{B} er stæðan `bba', þá er \textbf{aB}
% stæðan `abba' en \textbf{Ba} stæðan `bbaa'. Ef \textbf{a} er `a', þá
% skrifum við ``a\textbf{B}'' í stað ``\textbf{aB}'', frekar en
% ``\,`a'\textbf{B}''. \textbar{} Nafn stæðu á viðfangsmálinu táknað með
% tvöföldum gæsalöppum á yfirmálinu, þ.e. ``a\textbf{B}'' er nafn
% stæðunnar a\textbf{B}. \textbar{} Sama um stök tákn meðan stæðan hefur
% a.m.k. eina breytu á yfirmálinu. T.d. er 2 + \textbf{x} = 4 stæðan sem
% fæst með því að skrifa `2', `+', breytuna \textbf{x}, jafnaðarmerkið `='
% og `4' í þessari röð, frekar en að segja að það sé stæðan `2+'x'=4'.

% \subsection{Yrðingarökfræði}\label{yruxf0ingaruxf6kfruxe6uxf0i}

% Yrðingarökfræði fjallar um samtengingar, næstum eingöngu um

% \begin{itemize}
% \item
%   \ldots{} eða \ldots{},
% \item
%   \ldots{} og \ldots{},
% \item
%   ef \ldots{}, þá \ldots{},
% \item
%   \ldots{} þá og því aðeins að \ldots{} (ef og aðeins ef líka notað),
% \end{itemize}

% og neitunina

% \begin{itemize}
% \itemsep1pt\parskip0pt\parsep0pt
% \item
%   ekki.
% \end{itemize}

% Viðfangsefni hennar eru \emph{fullyrðingar}, þ.e. setningar sem eru
% annaðhvort sannar eða ósannar. \textbar{} ``En'' þýðir það sama og
% ``og'', eini munurinn eru blæbrigði. Eða er ekki útilokandi. \textbar{}

% \begin{itemize}
% \itemsep1pt\parskip0pt\parsep0pt
% \item
%   Ef yrðingin \textbf{A} er sönn, þá er yrðingin ``ekki \textbf{A}''
%   ósönn, en ef yrðingin \textbf{A} er ósönn, þá er yrðingin " ekki
%   \textbf{A}" sönn.
% \item
%   Ef yrðingin \textbf{A} og \textbf{B} eru báðar sannar, þá er
%   ``\textbf{A} og \textbf{B}'' sönn, annars er hún ósönn.
% \item
%   Ef yrðingar \textbf{A} og \textbf{B} eru báðar ósannar, þá er yrðingin
%   ``\textbf{A} eða \textbf{B}'' ósönn, annars er hún sönn.
% \item
%   Ef yrðingin \textbf{A} er sönn og \textbf{B} er ósönn, þá er yrðingin
%   ``ef \textbf{A} þá \textbf{B}'' ósönn, annars er hún sönn.
% \item
%   Yrðingin ``\textbf{A} þþaa \textbf{B}'' er sönn ef \textbf{A} og
%   \textbf{B} eru báðar sannar, eða báðar ósannar, annars eru hún ósönn.
% \end{itemize}

% Yrðingin ``\textbf{A} og \textbf{B}'' segir hið sama og ``ekki gildir
% (ekki \textbf{A} eða ekki \textbf{B})''. Yrðingin ``ef \textbf{A}, þá
% \textbf{B}'' segir hið sama og ``(ekki \textbf{A}) eða \textbf{B}''.
% Yrðingin ``\textbf{A} þþaa \textbf{B}'' segir hið sama og ``(ef
% \textbf{A}, þá \textbf{B}) og (ef \textbf{B}, þá \textbf{A})''.

% \subsubsection{Formlegt mál fyrir yrðinga
% rökfræði}\label{formlegt-muxe1l-fyrir-yruxf0inga-ruxf6kfruxe6uxf0i}

% Almennt er formlegt mál gefið með því að tiltaka \emph{stafróf}
% (bókstafi og tákn) fyrir málið og mengi af stæðum á málinu sem við
% köllum \emph{segðir}. Venjulega viljum við að ganga megi í endanlega
% mörgum skrefum úr skugga um hvort stæða er segð eða ekki. \textbar{} Röð
% tákna í segðum skiptir máli \textbar{} Venjulega er segðir skilgreindar
% með \emph{reglu}, svokölluðum \emph{myndunarreglum}.

% \emph{Skilgreining}. Skilgreinum formlegt mál \emph{L}(\emph{P}) fyrir
% yrðingarrökfræði þannig:

% \begin{itemize}
% \item
%   \begin{enumerate}
%   \def\labelenumi{(\Alph{enumi})}
%   \itemsep1pt\parskip0pt\parsep0pt
%   \item
%     Táknin í \emph{L}(\emph{P}) eru
%   \end{enumerate}

%   \begin{itemize}
%   \item
%     \begin{enumerate}
%     \def\labelenumi{(\roman{enumi})}
%     \itemsep1pt\parskip0pt\parsep0pt
%     \item
%       fjögur sérstök tákn, nefnilega \emph{vinstri sviginn} `(',
%       \emph{hægri sviginn} `)', \emph{neitunartáknið} $'\lnot'$
%       \emph{eðunartáknið} $' \vee '$;
%     \end{enumerate}
%   \item
%     \begin{enumerate}
%     \def\labelenumi{(\roman{enumi})}
%     \setcounter{enumi}{1}
%     \itemsep1pt\parskip0pt\parsep0pt
%     \item
%       bókstafirnir `p', `q', `r', `s', `t', `{[} p\_1 {]}', `{[} q\_2
%       {]}', `{[} r\_1 {]}', \ldots{} við leyfum alla stafi í latneska
%       stafrófinu, hugsanlega tölusetta með með lágvísi sem er náttúruleg
%       tala skrifuð í tugakerfi; köllum bókstafi \emph{yrðingabreytur}.
%     \end{enumerate}
%   \end{itemize}
% \item
%   \begin{enumerate}[<+->]
%   \def\labelenumi{(\Alph{enumi})}
%   \setcounter{enumi}{1}
%   \itemsep1pt\parskip0pt\parsep0pt
%   \item
%     Úr táknunum má setja saman leyfilegar samsetningar (segðir formlega
%     málsins) sem kallast \emph{yrðingasnið} með því að nota eftirfarandi
%     myndunarreglur.
%   \end{enumerate}

%   \begin{itemize}
%   \item
%     \begin{enumerate}
%     \def\labelenumi{(\roman{enumi})}
%     \itemsep1pt\parskip0pt\parsep0pt
%     \item
%       Ef \textbf{x} er yrðingabreyta, þá er stæðan \textbf{x} sem hefur
%       bara eitt tákn, nefnilega \textbf{x}, yrðingasnið
%     \end{enumerate}
%   \item
%     \begin{enumerate}
%     \def\labelenumi{(\roman{enumi})}
%     \setcounter{enumi}{1}
%     \itemsep1pt\parskip0pt\parsep0pt
%     \item
%       Ef \textbf{A} er yrðingasnið, þá er {[} \not **A** {]}
%       yrðingasnið.
%     \end{enumerate}
%   \item
%     \begin{enumerate}
%     \def\labelenumi{(\roman{enumi})}
%     \setcounter{enumi}{2}
%     \itemsep1pt\parskip0pt\parsep0pt
%     \item
%       ef \textbf{A} og \textbf{B} er yrðingasnið, þá er ( \textbf{A} {[}
%       \vee {]} \textbf{B}) yrðingasnið.
%     \end{enumerate}
%   \end{itemize}
% \end{itemize}

% Köllum ( {[} \vee {]} \textbf{A} ) \emph{neitun} yrðingasniðsins
% \textbf{A} og lesum ``ekki \textbf{A}'', og (\textbf{A} {[} \vee {]}
% \textbf{B}) \emph{eðun} yrðingasniðanna \textbf{A} og \textbf{B} og
% lesum ``\textbf{A} eða \textbf{B}'' í töluðu máli.

% \emph{Dæmi}. Skv. (i) eru `p' og `q' yrðingasnið, skv. (ii) eru þá `(
% {[} \not p {]} )' og `( {[} \not q {]} )' yrðingasnið, skv. (iii) eru þá
% `{[} (( \not p ) \vee (\not q)) {]}' yrðingasnið, skvi (ii) er `{[} (
% \not ((\not p) \vee (\not q )))' yrðingasnið. Eins sést að

% \begin{verbatim}
% '((\not p) \vee q)' og  '(\not (( \not (( \not p) \vee q)) \vee ( \not ((\not q) \vee p))))'
% \end{verbatim}

% \emph{Skilgreining}. Tökum upp \emph{skammstafanir}: Ef \textbf{A} og
% \textbf{B} eru yrðingasnið, þá er + (i) ({[} \textbf{A} \wedge **B** {]}
% ) skammstöfun fyrir '{[} ( \not ((\not **A** ) \vee (\not **B\textbf{
% ))); + (ii) {[} ( }A** \leftarrow **B**) skammstöfun fyrir '((
% \not **A**) \vee **B\textbf{); + (iii) '{[} ( }A**
% \leftrightarrow **B\textbf{ ) {]} skammstöfun fyrir '{[} (( }A**
% \leftarrow **B\textbf{ ) \wedge (}B** \leftarrow **A**))' sem er
% skammstöfun fyrir (\not (( \not (( \not **A** ) \vee **B** )) \vee (
% \not ((\not **B** ) \vee **A**))))

% \emph{Afleiddar myndunarreglur}. Ef \textbf{A}, \textbf{B} eru
% yrðingasnið, þá eru {[} ( \textbf{A} \wedge **B\textbf{ ) {]}, {[} (
% }A** \leftarrow **B\textbf{) {]} og {[} ( }A** \leftrightarrow **B** )
% {]} yrðingasnið.

% *Skilgr\textbf{. Látum {[} }x\_1\textbf{, \dotsc, }x\_n\textbf{ {]} vera
% ólíkar yrðingabreytur og }A\textbf{ vera yrðingasnið, {[} }B\_1\textbf{,
% \dotsc, }B\_n** {]} vera stæður á málinu \emph{L}(\emph{P}). Við táknum
% með

% \begin{verbatim}
% \[ **A**_{ **x**_1, \dotsc, **x**_n} [ **B**_1, \dostc, **B**_n ] \]
% \end{verbatim}

% eða I\^{}\{ \textbf{B}\emph{1, \dotsc, \textbf{B}}n
% \}\emph{\{\textbf{x}}1, \dotsc, \textbf{x}\_n\} \textbf{A}

% stæðuna sem fæst úr \textbf{A} með því að setja \textbf{B}\emph{k inn í
% stað \textbf{x}}k á hverjum þeim stað þar sem \textbf{x}\_k kemur fyrir
% í \textbf{A} fyrir öll {[} k = 1, \dotsc, n {]}.

% \emph{Viðvörun}. Þetta þarf að gerast samtímis.

% \textbf{Dæmi}: Látum \textbf{x} vera `p', \textbf{y} vera `q',
% \textbf{A} vera `{[} ((p \leftarrow q) \vee p) {]}', B vera `{[}( p
% \leftarrow q ){]}', \textbf{C} vera `{[}( q \wedge p ) {]}'. þá er

% \begin{verbatim}
% \[ **I**^{**B**, **C**}_{**x**, **y**} **A** \] stæðan '\[((( p \leftarrow q) \leftarrow (q \wedge p)) \vee (q \leftarrow p)\]',

% \[ I^{**B**}_{**x**} I^{**C**}_{**y**} **A** stæðan '((( p \leftarrow q) \leftarrow (q \wedge (p \leftarrow q))) \vee (q \leftarrow p))',


% \[ I^{**C**}_{**y**} I^{**B**}_{**x**} **A** stæðan '((((q \wedge p) \leftarrow p) \leftarrow (q \wedge p) \vee ((q \vee p))',
% \end{verbatim}

% \emph{Setning} Ef {[} \textbf{x}\emph{1, \dotsc, \textbf{x}}n {]} eru
% ólíkar breytur, {[} \textbf{A}, \textbf{B}\emph{1, \dotsc,
% \textbf{B}}n\textbf{, eru yrðingasnið, þá er
% }A\textbf{\emph{\{\textbf{x}}1, \dotsc, }x**\_n\} {[}\textbf{B}\emph{1,
% \dotsc, \textbf{B}}n{]} yrðingasnið. Þetta má sanna með þrepun yfir
% lengd \textbf{A}.

% Getum notað: \emph{Þrepunarlögmál fyrir stæður í }L\emph{(}P\emph{)}

% Látum \emph{E} vera eiginleika sem stæða í \emph{L}(\emph{P}) getur haft
% eða ekki og g.r.f. að

% \begin{itemize}
% \item
%   \begin{enumerate}[<+->]
%   \def\labelenumi{(\roman{enumi})}
%   \itemsep1pt\parskip0pt\parsep0pt
%   \item
%     ef \textbf{x} er yrðingarbreyta, þá hefur \textbf{x} eiginleika
%     \emph{E};
%   \end{enumerate}
% \item
%   \begin{enumerate}[<+->]
%   \def\labelenumi{(\roman{enumi})}
%   \setcounter{enumi}{1}
%   \itemsep1pt\parskip0pt\parsep0pt
%   \item
%     ef \textbf{A} er yrðingasnið með eiginleikann \emph{E}, þá hefur {[}
%     (\not A) {]} eiginleika \emph{E};
%   \end{enumerate}
% \item
%   \begin{enumerate}[<+->]
%   \def\labelenumi{(\roman{enumi})}
%   \setcounter{enumi}{2}
%   \itemsep1pt\parskip0pt\parsep0pt
%   \item
%     ef \textbf{A}, \textbf{B} eru yrðingasnið með eiginleika \emph{E},
%     þá hefur {[}(\textbf{A} \vee **B**){]} eiginleikann \emph{E}
%   \end{enumerate}
% \end{itemize}

% Þá hafa öll yrðingasnið eiginleikann \emph{E}.

% \emph{Reglur um svigasetningu} Leyfum okkur að sleppa svigum ef við
% getum sett þá einn aftur með því að fylgja eftirfarandi reglum: Alltaf
% má sleppa yztu svigum.

% Röðum röktáknunum í röð:

% \begin{verbatim}
% \[ '\not', '\vee', '\wedge', '\leftarrow', '\leftrightarrow' \]
% \end{verbatim}

% til að setja inn svia setjum við fyrst sviga utan um `\not', og sytzta
% yrðungasnið sem kemur á eftir því í stæðunni. Setjum næst sviga utanum
% `{[} \vee {]}' og styzta yrðingarsnið báðu megin. Ef það kemur oftar en
% einu sinni fyrir, þá göngum við á röðina frá hægri til vinztri. Næst er
% það sama fyrir {[}`\wedge'{]}, svo fyrir {[}`\leftarrow'{]} og svo loks
% fyrir {[} `\leftrightarrow ' {]}. Síðast yzta sviga.

% \emph{Dæmi}

% \begin{itemize}
% \itemsep1pt\parskip0pt\parsep0pt
% \item
%   `{[} p \wedge q \wedge r \leftarrow \not p \leftarrow r
%   \leftrightarrow \not \not q {]}', verður fyrst
% \item
%   `{[} p \wedge q \wedge r \leftarrow \not p \leftarrow r
%   \leftrightarrow \not (\not q) {]}', næst
% \item
%   `{[} p \wedge q \wedge r \leftarrow (\not p) \leftarrow r
%   \leftrightarrow \not (\not q) {]}', næst
% \item
%   `{[} p \wedge q \wedge r \leftarrow (\not p) \leftarrow r
%   \leftrightarrow ( \not (\not q)) {]}', næst
% \item
%   `{[} p \wedge (q \wedge r) \leftarrow (\not p) \leftarrow r
%   \leftrightarrow ( \not (\not q)) {]}', næst
% \item
%   `{[} ( p \wedge (q \wedge r)) \leftarrow (\not p) \leftarrow r
%   \leftrightarrow ( \not (\not q)) {]}', næst
% \item
%   `{[} ( p \wedge (q \wedge r)) \leftarrow ( (\not p) \leftarrow r )
%   \leftrightarrow ( \not (\not q)) {]}', næst
% \item
%   `{[} (( p \wedge (q \wedge r)) \leftarrow ( (\not p) \leftarrow r ) )
%   \leftrightarrow ( \not (\not q)) {]}', loks
% \item
%   `{[} (( p \wedge (q \wedge r)) \leftarrow ( (\not p) \leftarrow r )
%   \leftrightarrow ( \not (\not q)) ){]}'.
% \end{itemize}

% Athugum sérstaklega að

% \begin{verbatim}
% \[ **A** \leftarrow **B** \leftarrow **C** \]
% \end{verbatim}

% þýðir

% \begin{verbatim}
% \[ **A** \leftarrow (**B** \leftarrow **C**) \]
% \end{verbatim}

% svo að í

% \begin{verbatim}
% \[ ( **A** \leftarrow **B** ) \leftarrow **C** \]
% \end{verbatim}

% má ekki sleppa svigum.

% \subsubsection{Pólskur ritháttur}\label{puxf3lskur-rithuxe1ttur}

% Búum til nýtt mál \emph{L}'(\emph{P}) með sama stafróf og
% \emph{L}(\emph{P}) en nýjum myndunarreglum

% \begin{itemize}
% \item
%   \begin{enumerate}[<+->]
%   \def\labelenumi{(\roman{enumi})}
%   \itemsep1pt\parskip0pt\parsep0pt
%   \item
%     ef \emph{x} er yrðingabreyta, þá er \emph{x} segð í
%     \emph{L}'(\emph{P})
%   \end{enumerate}
% \item
%   \begin{enumerate}[<+->]
%   \def\labelenumi{(\roman{enumi})}
%   \setcounter{enumi}{1}
%   \itemsep1pt\parskip0pt\parsep0pt
%   \item
%     ef \emph{A} er segð í \emph{L}'(\emph{P}), þá er {[} \not **A** {]}
%     segð í \emph{L}'(\emph{P})
%   \end{enumerate}
% \item
%   \begin{enumerate}[<+->]
%   \def\labelenumi{(\roman{enumi})}
%   \setcounter{enumi}{2}
%   \itemsep1pt\parskip0pt\parsep0pt
%   \item
%     Ef \emph{A}, \emph{B} er segð í \emph{L}`(\emph{P}), þá er {[}
%     \vee *A* \emph{B} {]} segð í \emph{L}'(\emph{P}).
%   \end{enumerate}
% \end{itemize}

% Til að sjá að það megi lesa \emph{A} og \emph{B} út úr {[} \vee *A*
% \emph{B} {]}. Þurfum að sjá: Ef \emph{A}, \emph{B}, \emph{C}, \emph{D}
% eru segðir og {[} \vee *A* \emph{B} {]} er sama stæða og {[} \vee *C*
% \emph{D} {]}, þá er \emph{A} sama og \emph{C} og \emph{B} sama og
% \emph{D}.

% Ef {[} \vee *A* \emph{B} {]} er sama og {[} \vee *C* \emph{D} {]}, þá er
% önnurstæðan \emph{A}, \emph{C} upphafskafli í hinni.

% Það nægir að sýna:

% Ef \emph{A}, \emph{C} eru segðir á \emph{L}'(\emph{P}) og önnur er
% upphafskafli hinnar, þá eru þær sama segðin.

% \emph{Sö}: Þrepum yfir lengd minni segðar. Ef lengdin er 1, þá er minni
% stæðan breyta, en einu stæðurnar í \emph{L}`(\emph{P}) sem byrja á
% breytu eru breytur. Ef lengdin er stærri en 1, þá byrja annaðvhort báðar
% á'{[} \not {]}' eða báðar á {[} \vee {]}.

% Í fyrra tilvikinu má skrifa \textbf{A} sem {[}~\not **A**\_1 {]} og
% \textbf{C} sem {[}~\not **C\textbf{\emph{1 {]}, þar sem {[}
% \textbf{A}}1, }C\_1\textbf{ {]} eru segðir og önnur upphafskafli hinnar,
% svo að {[} }A\textbf{\emph{1 er \textbf{C}}1 {]} skv. þf. og þá
% }A\textbf{ sama og }C\textbf{. Í seinna tilviki er }A** stæðan {[}
% \vee **A\textbf{\emph{1 \textbf{A}}2 {]} og }C** er {[}
% \vee **C\textbf{\emph{1, \textbf{C}}2 {]}, þar sem {[}
% }A\textbf{\emph{1, \textbf{A}}2, }C\textbf{\emph{1, \textbf{C}}2 {]} eru
% segðir í \emph{L}'(\emph{P}), og önnur af segð {[} }A\textbf{\emph{1,
% \textbf{C}}1 {]} er upphafssegð hinnar. Skv. Þ.f. er {[}
% }A\textbf{\emph{1{]} sama og {[} \textbf{C}}1 {]} og þá er {[}
% }A\textbf{\emph{2 {]} sama og {[} \textbf{C}}2 {]} og þvi }A\textbf{
% sama og }C**.

% Látum \emph{p} vera mengi stæða í \emph{L}(\emph{P}) og \emph{p}`vera
% mengi stæða í \emph{L}'(\emph{P}). Fáum varpanir

% {[}

% \begin{itemize}
% \itemsep1pt\parskip0pt\parsep0pt
% \item
%   \theta: \emph{p} \leftarrow *p*'
% \item
%   \eta: \emph{p}' \leftarrow *p*
% \end{itemize}

% {]}

% með þrepun þ.a.

% {[} + \theta[x] er \textbf{x} ef x er breyta + \theta[(\not **A**)] er
% \not \theta[**A**], + \theta[(**A** \vee **B**)] er
% \vee \theta[**A**] \theta[**B**]
% {]} og {[} + \eta[x] er \textbf{x} ef x er breyta + \eta[\not **A**] er
% (\not \theta[**A**]), + \eta[ \vee **A** **B**] er (
% \theta[**A**] \vee \theta[**B**]), {]} sem eru andhverfur hvor annarar.
% Skammstafanir í \emph{L}'(\emph{P}) {[} + \wedge **A\textbf{ }B** fyrir
% \not \vee \not **A** \not **B** + \leftarrow **A\textbf{ }B** fyrir
% \vee \not **A\textbf{ }B** + \leftarrow **A\textbf{ }B** fyrir
% \not \vee \not \vee \ot **A\textbf{ }B** \not \vee \not **B\textbf{
% }A**. {]}

% Segjum að yrðing hafi \emph{sanngildið} 0 ef hún er ósönn og
% \emph{sanngildið} 1 ef hún er sönn.

% \emph{Skilgreining}

% \begin{enumerate}[<+->]
% \def\labelenumi{\arabic{enumi}.}
% \itemsep1pt\parskip0pt\parsep0pt
% \item
%   Látum $n \in \N, n \geq 1$. Vörpun {[} f: \set{0,1}\^{}n \to \set{0,1}
%   kallast n-\emph{stætt sannfall} (\emph{Boole-fall}) 2. Látum \$
%   \textbf{x}\_1, \dotsc **x**\_n \$ vera yrðingabreytur
% \end{enumerate}

% \ldots{}

% \begin{verbatim}
% \[ f_{**x**_1, \dotsc, x_n}^{**A**} : \set{0,1}^n \to \set{0,1} \]
% þannig að

% +   (i). Ef **A** er yrðingabreyta $ **x**_k $, og $ $ t = (t_1, \dotsc, t_n) \in \set{0,1}^n $, þá er
%     \[ f_{**x**_1, \dotsc, x_n}^{**A**} (t) := t_k \].
% +   (ii). Ef **A** er yrðingarsnið $ ( \not **t**)$, þá er
%     \[ f_{**x**_1, \dotsc, x_n}^{**A**} (t) := \bcondef 1 & \ef f_{**x**_1, \dotsc, x_n}^{**C**} (t) = 0 \\ 0 & \annars \econdef \]
% +   (iii). Ef **A** er yrðingarsnið $ ( **C** \vee **D**)$, þá er
%     \[ f_{**x**_1, \dotsc, x_n}^{**A**} (t) = \bcondef 0 & \ef f_{**x**_1, \dotsc, x_n}^{**C**} = f_{**x**_1, \dotsc, x_n}^{**D**} (t) = 0 \\ 1 & \annars \econdef \]

% Köllum $f_{**x**_1, \dotsc, x_n}^{**A**}$ sannfall yrðingarsniðsins **A** m.t.t. $ **x**_1 \dotsc, **x**_n $.
% \end{verbatim}

% Sannföllum yrðingasniða má lýsa með töflum, svokölluðum
% \emph{sanntöflum}. Tafla fyrir n-stætt yrðingarsnið hefur $2^n$ reiti.
% Einföldustu töflunar fyrir yrðingarsnið með aðeins einni breytu, t.d.
% `p' og `$\not p $'; töflur fyrir þau eru =

% \begin{tabular}[| c || c ]
%         p & p \\
%         \hline \\
%         0 & 0   \\
%         1 & 1 \\
%         \hline
%        \end{tabular}

% \begin{tabular}[| c || c ]
%         p & \not p \\
%         \hline \\
%         0 & 1   \\
%         1 & 0 \\
%         \hline
%        \end{tabular}

% það eru fleiri yrðinarsnið með aðeins eina breytu, t.d. `\$ p
% \vee \not p $' og '$ p \wedge \not p \$' töflur fyrir þau eru

% \begin{verbatim}
% \begin{tabular}[| c || c ]
%     p & p \\
%     \hline \\
%     0 & 0   \\
%     1 & 1 \\
%     \hline
% \end{tabular}

% \begin{tabular}[| c || c ]
%     p & p \vee \not p \\
%     \hline \\
%     0 & 1   \\
%     1 & 1 \\
%     \hline
% \end{tabular}

% \begin{tabular}[| c || c ]
%     p & p \wedge \not p \\
%     \hline \\
%     0 & 0   \\
%     1 & 0 \\
%     \hline
% \end{tabular}
% \end{verbatim}

% sanntöflur fyrir ``samtengingarnar''

% \begin{verbatim}
% \begin{tabular}[| c  | c || c ]
%     p & q & p \vee q \\
%     \hline \\
%     0 & 0 & 0 \\
%     0 & 1 & 1 \\
%     1 & 0 & 1 \\
%     1 & 1 & 1 \\
%     \hline
% \end{tabular}

% \begin{tabular}[| c  | c || c ]
%     p & q & p \wedge q \\
%     \hline \\
%     0 & 0 & 0 \\
%     0 & 1 & 0 \\
%     1 & 0 & 0 \\
%     1 & 1 & 1 \\
%     \hline
% \end{tabular}

% \begin{tabular}[| c  | c || c ]
%     p & q & p \leftarrow q \\
%     \hline \\
%     0 & 0 & 1 \\
%     0 & 1 & 1 \\
%     1 & 0 & 0 \\
%     1 & 1 & 1 \\
%     \hline
% \end{tabular}

% \begin{tabular}[| c  | c || c ]
%     p & q & p \leftrightarrow q \\
%     \hline \\
%     0 & 0 & 1 \\
%     0 & 1 & 0 \\
%     1 & 0 & 0 \\
%     1 & 1 & 1 \\
%     \hline
% \end{tabular}
% \end{verbatim}

% Finnum sanntöflu fyrir `((p \leftarrow q) \wedge p) \leftarrow q'

% \begin{tabular}[| c  | c || c | c | c ]
%         p & q & p \leftarrow q  & (p \leftarrow q) \wedge p & ((p \leftarrow q) \wedge ) \leftarrow q \\
%         \hline \\
%         0 & 0 & 1 & 0 & 1 \\
%         0 & 1 & 1 & 0 & 1 \\
%         1 & 0 & 0 & 0 & 1 \\
%         1 & 1 & 1 & 1 & 1 \\
%         \hline
%     \end{tabular}

% \emph{Skilgreining}

% \begin{enumerate}[<+->]
% \def\labelenumi{\arabic{enumi}.}
% \itemsep1pt\parskip0pt\parsep0pt
% \item
%   Látum \textbf{A} vera yrðingasnið og \$ \textbf{x}\emph{1, \dotsc,
%   \textbf{x}}n \$ vera upptalningu á breytunum í \textbf{A}. Við segjum
%   að \textbf{A} sé sísanna \emph{sísanna} (tautology) ef sannfallið \$
%   f\_\{\textbf{x}\emph{1, \dotsc, x}n\}\^{}\{\textbf{A}\} \$ er
%   fastafallið 1. (Þetta er óháð hvernig við tölusetjum breyturnar). Við
%   segjum að \textbf{A} sé \emph{mótsögn} ef sannfall þess er 0.
% \item
%   Segjum að yrðingarsnið \textbf{B} sé \emph{rökafleiða} yrðingarsniðs
%   \textbf{A} ef $**A** \leftarrow **B** $ sé sísanna.
% \item
%   Segjum að yrðingarsnið \textbf{A} og \textbf{B} séu \emph{rökfræðilega
%   jafngild} ef $(**A** \leftrightarrow **B**)$ er sísanna.
% \end{enumerate}

% \emph{Dæmi}: Yrðingarsniðin `$p \vee \not p$' og `\$ (( p \leftarrow q)
% \wedge p) \leftarrow $' eru sísönnur. Því er 'q' rökafleiðing af '$(p
% \leftarrow q) \wedge p $'. Auðvelt er að sjá að 'p' og '$ \not \not p
% $' rökfræðilega jafngild, eins eru '$ p \leftarrow q $' og '$ \not q
% \leftarrow \not p \$' rökfræðilega jafngild.

% \emph{Setning}. Ef \textbf{A} og $**A** \to **B**$ eru sísönnur, þá er
% \textbf{B} sísanna.

% \emph{Sönnun}. Látum $**x**_1, \dotsc, **x**_n $ vera upptalningu
% bretynanna sem koma fyrir í \textbf{A} eða \textbf{B}. G.r.f. að
% \textbf{B} sé ekki sísanna. Þá er til stak \$ t = (t\_1, \dotsc, t\_n)
% \in \set{0,1}\^{}n\$ þ.a. $f_{**x**_1, \dotsc, x_n}^{**B**} (t) = 0)$.
% Nú er \textbf{A} sísanna, svo að \$ f\_\{\textbf{x}\emph{1, \dotsc,
% x}n\}\^{}\{\textbf{A}\} (t) = 1\$ og þá \$ f\_\{\textbf{x}\emph{1,
% \dotsc, x}n\}\^{}\{\not **A\textbf{\} = 0$. En þá er $
% f\_\{}x\textbf{\emph{1, \dotsc, x}n\}\^{}\{}A** \leftarrow **B\textbf{\}
% (t) = f\_\{}x**\emph{1, \dotsc, x}n\}\^{}\{ \not **A** \vee **B\textbf{
% \} = 0$, í móstögn við að $}A** \to **B**\$ sé sísanna.

% \emph{Setning}. Gr.r.f að \textbf{A} sé sísanna,
% $**x**_1, \dotsc, **x**_n$ vera ólíkar yrðingabreytur og
% $**B**_1, \dotsc, **B**_n$ vera yrðingarsnið. Þá er
% $**A**_{**x**_1, \dotsc, **x**_n} [**B**_1, \dotsc, **B**_n]$ sísanna.

% ``\emph{Sö}'': Sannfall fyrir
% $**A**_{**x**_1, \dotsc, **x**_n} [**B**_1, \dotsc, **B**_n]$ er
% samskeyting af $f_{**x**_1, \dotsc, x_n}^{**A**}$ og einhverju öðru, en
% \$ f\_\{\textbf{x}\emph{1, \dotsc, x}n\}\^{}\{\textbf{A}\}\$ er
% fastafallið, svo að útkoman er fastafallið 1.

% \emph{Setning}. Látum $f: \set{0,1}^n \to \set{0,1}$ vera sannfall. Þá
% er til yrðingarsnið \textbf{A} ásamt ólíkum breytum \$
% \textbf{x}\emph{1, \dotsc, \textbf{x}}n\$, þ.a. {[} f =
% f\_\{\textbf{x}\emph{1, \dotsc, x}n\}\^{}\{\textbf{A}\}. {]}

% Geymum sönnun.

% Tvístæð sannföll eru $2^{2^} = 2^4 = 16 $ talsins. Látum $*L*^{*}$ vera
% mál, útvíkkun á \emph{L}(\emph{p}) með tvístæðum röktánum að auki fyrir
% þau tvístæðu sannföll sem vantar.

% Þar af tvö, `\$ \downarrow $' og '$ \textbar{} \$' með sanntöflu

% \begin{verbatim}
% \begin{tabular}[| c  | c || c ]
%     p & q & p \downarrow q \\
%     \hline \\
%     0 & 0 & 1 \\
%     0 & 1 & 0 \\
%     1 & 0 & 0 \\
%     1 & 1 & 0 \\
%     \hline
% \end{tabular}
% \end{verbatim}

% og

% \begin{verbatim}
% \begin{tabular}[| c  | c || c ]
%     p & q & p | q \\
%     \hline \\
%     0 & 0 & 1 \\
%     0 & 1 & 1 \\
%     1 & 0 & 1 \\
%     1 & 1 & 0 \\
%     \hline
% \end{tabular}
% \end{verbatim}

\begin{proof}
  $\hdots$ \\
  $ \textbf{P}_{ij}$ vera yrðingabreytuna $\textbf{x}_j$ ef talan $1$ stendur í
  j\-ta dálki i\-tu línu, en látum $\textbf{P}_{ij}$ vera $\lnot x_j$ ef
  þar stendur 0.  Látum $A_{i}$ vera

  \[ \textbf{P}_{i,1} \wedge \textbf{P}_{i,2} \wedge \dotsb \wedge
  \textbf{P}_{i,n} \] fyrir $i = 1, \dotsc, 2^n$ og $A$ vera
  \[ \textbf{A}_{i_1} \vee \textbf{A}_{i_2} \vee \dotsb \vee \textbf{A}_{i_n} \] þar
  sem $i_1, \dotsc, i_r$ eru númerin á i.\\

  $\hdots$ \\
  Ef engar slíkar línur eru til, látum þá $\textbf{A}$ vera
  \[ \textbf{x}_1, \wedge \dotsb \wedge \textbf{x}_n \wedge \lnot \textbf{x}_1 \wedge
  \dotsb \wedge \lnot \textbf{x}_n \]
\end{proof}

töluðum um mál $L^{*}$ með 16 táknum fyrir öll tvístæð yrðingasnið, þar á meðal
$\vee$, $\wedge$, $\rightarrow$, $\leftrightarrow$, og líka einstæða táknið $\lnot$,



$\hdots$\\


vegna þess að $'p \vee q'$ er jafngilt $'\lnot ( \lnot p \wedge  \lnot q)'$
er $\set{'\lnot', '\wedge'}$. Mengið $\set{'\lnot'}$ getur ekki verið nægjanlegt.
En $'p \vee q'$ er jafngilt $'((\lnot p) \rightarrow q'$ svo að mengið
$\set{'\lnot', '\rightarrow'}$ er nægjanlegt. Atugum að $'\lnot p'$ er jafngilt
\[p \downarrow p\]
og
$'p \wedge q'$ er jafngilt $'(( p \downarrow p) \downarrow (q \downarrow q))'$
svo að $\set{'\downarrow'}$ er nægjanlegt. Líka er
$'\lnot p'$ jafngilt $'p | p'$.
og $'p \vee q'$ er jafngilt $'(p | p) | (q|q)'$ svo að $\set{'|'}$ er nægjanlegt.

segð $'(p \rightarrow)'$ verðr jafngild

\['(((p \downarrow  p) \downarrow (( q \downarrow q) \downarrow (q \downarrow q))) \downarrow (( p \downarrow p) \downarrow ((q \downarrow q) \downarrow (q \downarrow q))))' \]

sem í pólskum rithætti verður $ \downarrow \downarrow \downarrow  p p  \downarrow \downarrow q q  \downarrow q q  \downarrow \downarrow p p  \downarrow \downarrow q q  \downarrow q q'$

\begin{setn}
Ef \textbf{s} er tvístætt tákn í $L^{*}$ og $\set{\textbf{s}}$ er
nægjanlegt, þá er $\textbf{s}$ annaðhvort $'\downarrow'$ eða $'|'$.

\end{setn}


\section{Frumsendur fyrir yrðingarökfræði}

\begin{skgr}
  \emph{Formleg kenning $\mathcal{T}$} er gefin með:
  \begin{anum}
  \item Gefið er formlegt mál $\mathcal{L(T)}$, \emph{mál kenningarinnar}.
  \item  Gefið er tiltekið mengi $\mathcal{A}$ af segðum á $\mathcal{L(T)}$
    sem kallast \emph{frumsendur} kenningarinnar.
  \item Gefnar eru endanlega margar \emph{rökreglur}. Hver rökregla segir
    að segðir af tiltekini gerð hafi einhverja segð af tiltekkinni gerð sem \emph{afleiðingu}.
  \end{anum}
    Venjulega er krafizt að við getum gengið úr skugga um í endanlega mörgum skrefum
    hvort segð er frumsenda eða ekki. Regla sem leyfir okkur að búa til frumsendur í
    $\mathcal{T}$ kallast \emph{frumsendugrip}.

    \emph{Sönnun} í $\mathcal{T}$ er endanleg runa $\textbf{A}_1, \dotsc, \textbf{A}_n$
    af segðum á $\mathcal{L(T)}$ þ.a. fyrir sérhvert $k = 1, \dotsc, n$ sé $\textbf{A}_k$
    annaðhvort frumsenda eða afleiðing af einhverjum af segðunum
    $\textbf{A}_1, \dotsc, \textbf{A}_{k-1}$ skv. rökreglum kenningarinnar.

    Segð $\textbf{A}$ er \emph{setning} í $\mathcal{T}$ ef hún er síðasta setningin
    í sönnun; skrifum þá
    \[ \vdash_{\mathcal{T}} \textbf{A} \text{ eða } \vdash \textbf{A}\]
    skrifum
    \[ \mathcal{H} \vdash_{\mathcal{T}} \textbf{A} \]
    þar sem $\mathcal{H}$ er mengi af segðum ef
    $\textbf{A}$ er setning í $\mathcal{T[H]}$ sem fæst með því að bæta
    öllum segðunum í $\mathcal{H}$ við $\mathcal{T}$ sem nýjum frumsendum.
    Ef $\mathcal{H}$ hefur bara endanlega margar segðir,
    $\textbf{H}_1, \dotsc, \textbf{H}_n$ , þá skrifum við
    \[\textbf{H}_1, \dotsc, \textbf{H}_n \vdash_{\mathcal{T}} \textbf{A}\]
    í stað $\mathcal{H} \vdash_{\mathcal{T}} A$.
\end{skgr}

\begin{skgr}
  Skilgreinum formlega kenningu $\mathcal{P}$ fyrir yrðingarökfræði þannig:
  \begin{anum}
  \item Mál kenningarinnar er $\mathcal{L(P))}$
  \item Frumsendur kenningarinnar eru gefnar með eftirfarandi frumsendugripum:
    \begin{enumerate}[label=\textbf{F}\arabic*]
    \item \[ \textbf{A} \vee \textbf{A} \rightarrow \textbf{A} \]
    \item \[ \textbf{A} \rightarrow \textbf{B} \vee \textbf{A} \]
    \item \[ ( \textbf{A} \rightarrow \textbf{B}) \rightarrow (\textbf{C} \vee \textbf{A} \rightarrow \textbf{B} \vee \textbf{C})\]
    \end{enumerate}
  \item Kenningin hefur aðein eina rökreglu: \emph{modus ponens}, og er þannig:
    \[ \textbf{MP}: \text{ Af } \textbf{A} \rightarrow \textbf{B} \text{ og } \textbf{A} \text{ leiðir } \textbf{B}\]
    Fáum strax \emph{afleidda rökreglu}; \emph{innsetningarreglu}:

    \textbf{Inn}: Látum \textbf{A} vera yrðingarsnið, $ \textbf{x}_1, \dotsc, \textbf{x}_n$
    vera ólíkar yrðingabreytur og $ \textbf{B}_1, \dotsc, \textbf{B}_n$ vera
    yrðingasnið. Ef $\vdash \textbf{A}$, þá er $\textbf{A}_{\textbf{x}_1, \dotsc, \textbf{x}_n}[\textbf{B}_1, \dotsc, \textbf{B}_n]$
    \begin{proof}
      Ef $\textbf{A}_1, \dotsc, \textbf{A}_m$ er sönnun á $A$, þá er
      \[\textbf{A}_{ 1 \textbf{x}_1, \dotsc, \textbf{x}_n}[\textbf{B}_1, \dotsc, \textbf{B}_n], \dotsc, \textbf{A}_{m \textbf{x}_1, \dotsc, \textbf{x}_n}[\textbf{B}_1, \dotsc, \textbf{B}_n]\]
      sönnun á $\textbf{A}_{\textbf{x}_1, \dotsc, \textbf{x}_n}[\textbf{B}_1, \dotsc, \textbf{B}_n]$

    \end{proof}

  \end{anum}
\end{skgr}

\emph{Afleidd rökreglar $\textbf{R1}$}. Látum $\mathcal{H}$ vera mengi af yrðingasniðum
og $\textbf{A,B,C}$ vera yrðingasnið.

Ef $\mathcal{H} \vdash \textbf{A} \rightarrow \textbf{B}$ og $\mathcal{H} \vdash \textbf{C} \vee \textbf{A}$
þá $\mathcal{H} \vdash \textbf{B} \vee \textbf{C}$æ

\begin{proof}
  Notum $\textbf{MP}$ tvisvar á $\textbf{F3}$
\end{proof}

$\vdots$\\


\begin{setn}[Afleidd rökregla \textbf{R6}]
  Ef $\mathcal{H} \vdash \textbf{A} \rightarrow \textbf{C}$ og
  $\mathcal{H} \vdash \lnot \textbf{A} \rightarrow \textbf{C}$,
  þá $\mathcal{H} \vdash \textbf{C}$.
\end{setn}

\begin{setn}[Afleidd rökregla \textbf{R7}]
    Ef $\mathcal{H} \vdash \textbf{A} \rightarrow \textbf{B}$,
    þá $ \mathcal{H} \vdash \textbf{A} \rightarrow \textbf{C} \vee \textbf{B}$.
\end{setn}


$\vdots$\\

\begin{setn}[Fylgisetning]
  Við höfum
  \[ \mathbf{A} \vdash \mathbf{B} \text{ þþaa } \vdash \mathbf{A} \rightarrow \mathbf{B} \]

\end{setn}

\begin{setn}
  Sérhver setning i $\mathcal{P}$ er sísanna.
  \begin{proof}
    '$p \vee p \rightarrow p$, '$p \rightarrow q \vee p$' og
    '$(p \rightarrow q) \rightarrow (r \vee p \rightarrow q \vee r)$'
    eru sísönnur, svo að frumsendur er sísönnur og \textbf{MP} varðveitir sísönnur.
  \end{proof}
\end{setn}
\section{Fullkomleikasetning fyrir yrðingarökfræði}

Sérhver sísanna er setning í formlegu kenningunni $\mathcal{P}$
\begin{skgr}
  Kenning $\mathcal{T}$ er \emph{samkvæm} ef til er segð á máli $\mathcal{T}$
  sem er ekki setning í $\mathcal{T}$.
\end{skgr}

\begin{setn}
  $\mathcal{P}$ er samkvæm.
    \begin{proof}
      Til eru yrðingarsnið sem eru ekki sísönnur.
    \end{proof}
\end{setn}


$\vdots$\: \: \:hér vantar nokkra daga.


Athugum að í '$\exists y (2 \cdot y = x )$', sem er yrðing í málinu $\mathcal{N}$,
þar sem $2$ er skammstöfun fyrir $SS0$, er $y$ buyndin breyta þar sem hún kemur fyrir, en $x$
ekki; þetta er fullyrðing um $x$, en ekki um $y$. Gætum eins skrifað
'$\exists t (2 \cdot t = x )$', $\exists z (2 \cdot z = x )$, og
allar þessar fullyrðingar segja ``x er jöfn tala''. Viljum nú setja heiti inn fyrir
breytur í fullyrðingum en bara þar sem þær eru frjálsar; viljum að fullyrðingin segi
hið sama um hlutinn með heitið eins og hún segir um það sem breytan stendur
fyrir; t.d. segir '$\exists t (2 \cdot t = 2 )$' að $2$ sé jöfn tala, sem er rétt,
og '$\exists t (2 \cdot t = 3 )$', sem er rangt. En setjum nú $y+1$ inn fyrir $x$
í '$\exists y (2 \cdot y = y+1 )$' sem segir alls ekki að $y+1$
sé jöfn tala, heldur að jafnan '$2y = y+1$ hafi lausn (í $\N$), sem er rétt.

Þetta er af því að heitið inniheldur breytuna '$y$', sem verður bundin þegar
að við setjum hana inn. Þetta verður að banna!

\begin{ath}
 \textbf{x} er bundin á tilteknum stað í yrðingu \textbf{A} ef
staðurinn er í hlutfullyrðingunni í A af gerðinni
$\forall \mathbf{x} \mathbf{B}$ (eða $\exists x B$).
\end{ath}

\begin{skgr}
  \begin{enumerate}[(1)]
  \item Við segjum að heiti $\mathbf{a}$ á máli fyrstu stéttar
    $\mathcal{L}$ sé \emph{innsetjanlegt} í yrðingu \textbf{A} á
    $\mathcal{L}$ ef eftirfarandi skilyrði sé fullnægt. Fyrir breytu
    \textbf{y} aðra en \textbf{x} sem kemur fyrir í \textbf{a}
    inniheldur yrðingin \textbf{A} enga hlutyrðingu af gerðinni
    $\forall \mathbf{y} \mathbf{B}$ ( eða $\exists \mathbf{y}
    \mathbf{B}$) þannig að breytan $\mathbf{x}$ komi fyrir frjáls í
      $\mathbf{B}$.
    \item Látum \textbf{A} vera yrðingu á $\mathcal{L}$,
      $\mathbf{x}_1, \dotsc, \mathbf{x}_n$ ólíkar breytur og
      $\mathbf{a}_1, \dotsc, \mathbf{a}_n$ heiti þ.a. fyrir
      hvert $k = 1, \dotsc, n$ sé $\mathbf{a}_k$ innsetjanlegt
      fyrir $\mathbf{x}_k$ í $\mathbf{A}$.
      Táknum með $\mathbf{A}_{\mathbf{x}_1, \dotsc, \mathbf{x}_n}[\mathbf{a}_1, \dotsc, \mathbf{a}_n]$
      eða $I^{\mathbf{a}_1, \dotsc, \mathbf{a}_n}_{\mathbf{x}_1, \dotsc, \mathbf{x}_n} \mathbf{A}$
      yrðinguna sem fæst með því að setja inn $\mathbf{a}_k$ inn fyrir $\mathbf{x}_k$ í
      $\mathbf{A}$ fþar sem $\mathbf{x}_k$ kemur frjálst fyrir í $\mathbf{A}$ fyrir
      öll $k = 1, \dotsc, n$ \emph{samtímis}.

    \item Ef $\mathbf{b}$ er heiti, þá er
      \[ \mathbf{b}_{\mathbf{x}_1, \dotsc, \mathbf{x}_n} [ \mathbf{a}_1, \dotsc, \mathbf{a}_n] \]
      eða
      \[I^{\mathbf{a}_1, \dotsc, \mathbf{a}_n}_{\mathbf{x}_1, \dotsc, \mathbf{x}_n} \mathbf{b}\]
      heitið sem fæst með því að setja $\mathbf{a}_k$ inn fyrir $\mathbf{x}_k$ í $\mathbf{b}$
      fyrir $k= 1, \dotsc, n$ \emph{samtímis}.
    \end{enumerate}
\end{skgr}

\begin{ath}
  Yrðingarnar $I^{\mathbf{a}, \mathbf{b}}_{\mathbf{x}, \mathbf{y}} \mathbf{A}$,
  $I^{\mathbf{a}}_{\mathbf{x}} I^{\mathbf{b}}_{\mathbf{y}} \mathbf{A}$ og
  $I^{\mathbf{b}}_{\mathbf{y}} I^{\mathbf{a}}_{\mathbf{x}} \mathbf{A}$
  get allar verið ólíkar, þar sem við erum ekki að setja allt inn á sama tíma.
\end{ath}

Við erum alltaf að gera ráð fyrir að við séum með innsetjanlegt dót.


\begin{skgr}[\emph{Formleg kenning fyrstu stéttar}] $\mathcal{T}$ er gefin með
  eftirfarandi hættir:
  \begin{enumerate}[A.]
  \item Gefið er formlegt mál $\mathcal{L(T)}$ fyrstu stéttar sem við köllum
    \emph{mál kenningarinnar}.
  \item Frumsendur kenningarinnar skiptast í tvo flokka;
    \emph{eiginlegar frumsendur}, sem geta verið hvaða yrðingar sem er á málinu
    $\mathcal{L(T)}$, og \emph{rökfrumsendur} sem myndaðar eins fyrir allar svona
    kenningar og eru búnar til með fimm frumsendum:
    \begin{enumerate}[\textbf{F\arabic*}]
    \item  $\mathbf{A} \vee \mathbf{A} \rightarrow \mathbf{A} $.
    \item  $ \mathbf{A} \rightarrow \mathbf{B} \vee \mathbf{A}$.
    \item  $ ( \mathbf{A} \rightarrow \mathbf{B}) \rightarrow (\mathbf{C} \vee \mathbf{A} \rightarrow \mathbf{B} \vee \mathbf{C})$.
    \item  $ \forall \mathbf{x} \mathbf{A} \rightarrow \mathbf{A}_{\mathbf{x}} [\mathbf{a}]$, ef $\mathbf{a}$
      fyrir $\mathbf{x}$  í $\mathbf{A}$.
    \item $\forall \mathbf{x} ( \mathbf{A} \vee \mathbf{B} ) \rightarrow \mathbf{A} \vee \forall \mathbf{x} \mathbf{B}$ ef $\mathbf{x}$ kemur hvergi fyrir frjáls í $\mathbf{A}$.
    \end{enumerate}

    Hér getur $\mathbf{A}, \mathbf{B}, \mathbf{C}$ verið hvaða yrðing á $\mathcal{L(T)}$
    sem vera skal.
  \item Kenningin hefur tvær rökreglur, \emph{modus ponens} og \emph{alhæfingu}:
    \begin{itemize}
    \item \textbf{MP}. Af $\mathbf{A} \rightarrow \mathbf{B}$ og $\mathbf{A}$ leiðir $\mathbf{B}$
    \item \textbf{Alh.} Af $\mathbf{A}$ leiðir $\forall \mathbf{x} \mathbf{A}$.
    \end{itemize}
  \end{enumerate}
\end{skgr}

\begin{ath}
  Ef ég segi: ``Ef x og y eru rauntölur, þá er $x + y = y+x$'', þá meina ég
  ``Fyrir öll x og y gildir: Ef x og y eru rauntölur þá er $x  + y = y + x$''.


  \textbf{Alh}. þýður ekki að $\mathbf{A} \rightarrow \forall \mathbf{x} \mathbf{A}$ gildi !!!
\end{ath}

\begin{skgr}[\emph{Samsendarkenningin}] er formleg kenning fyrstu stéttar sem hefur
  samsemdarmerkið ``$=$'' sem tví stætt umsagnartákn (hugsanlega skilgreint) og
  þ.a.


  \textbf{Eq1}. $\vdash_{\mathcal{T}}$ '$x = x$'

  og þ.a. fyrir sérhverja grunnyrðingu $\mathbf{A}$ sé


 \textbf{Eq2}. $\vdash_{\mathcal{T}}$ '$x = y$' $\rightarrow ( \mathbf{A}_z[\mathbf{x}] \rightarrow \mathbf{A}_z[\mathbf{y}])$.

  (að því gefnu að $\mathbf{x}, \mathbf{y}$ sé innsetjanleg fyrir $\mathbf{z}$ í $\mathbf{A}$).

  Við teljum samsemdarkenninguna með rökfrumsendum, en ekki eiginlegum frumsendum.
\end{skgr}

\begin{daemi}
  \begin{enumerate}[(1)]
  \item Skilgreinum kenningu $\mathcal{O}_1$ umm röðuðu mengi
    sem samsemdarkenningu á málinu $\mathcal{L}$ $(\mathcal{O}_{1})$ sem hefur aðeins
    tvístæða umsagnartáknið '$\leq$' og þrjár eiginlegar frumsendur


    '$x \leq x$', ' $x \leq y \wedge y \leq x \rightarrow x = y$', '$x \leq y \wedge y \leq z \rightarrow x \leq z$'

    Skilgreinum kenningum $\mathcal{O}_2$ um stantar raðir sem samsendar kenningu
    með aðeins tvístæða umsagnar táknið ``$<$'' og tvær eiginlegar frumsendur


    '$\lnot (x < x)$' og $' x < y \wedge y < z \rightarrow x < z$'

  \item
    Skilgreinum tvær kenningar $\mathcal{G}_1$ og $\mathcal{G}_2$ um grúpur,
    báðar sem samsendarkenningar.

    Sú fyrri hefur mál með einu tvístæðu fallatákni,
    '$\cdot$' og tveimur frumsendum


    '$x \cdot (y \cdot z) = (x \cdot y) \cdot z$'

    og

    '$\exists y (\forall x (( x \cdot y = x) \wedge (y \cdot x) = x) \wedge \forall x \exists z ((x \cdot z = y) \wedge (z \cdot x = y)))$'

    seinnni kenningin hefur mál með tvístæðu fallatákni '$\cdot$', einum
    fasta $e$ og einu einstæðu fallatákni $J$; skrifum $x^{-1}$sem skammstöfum fyrir '$Jx$'; höfum þrjár frumsendur

    '$x \cdot (y \cdot z) = (x \cdot y) \cdot z$', '$(x \cdot e = x) \wedge ( e \cdot x = x)$' og '$(x\cdot x^{-1} = e) \wedge (x^{-1} \cdot x = e)$'
  \end{enumerate}


\end{daemi}

\begin{skgr}
  \begin{enumerate}[(1)]
  \item  Kenningin $\mathcal{N}$ fyrir náttúrulegar tölur er skilgreind sem
    samsemdarkenning á málinu $\mathcal{L}(\mathcal{N})$
    ,með fasta '$0$', tvístæðum fallatáknum '$+$' og '$\cdot$', tvístæðu umsagnartákni '$<$' og einstæðu fallatákni '$S$' og eftirfarandi
frumsendum:
    \begin{enumerate}[\textbf{N\arabic*}]
    \item $Sx \neq 0$
    \item $Sx = Sy \rightarrow x = y$
    \item $x + 0 = x$
    \item $x + Sy = S(x+y)$
    \item $x \cdot 0 = 0$
    \item $x \cdot Sy = (x \cdot y) + x$
    \item $\lnot (x < 0)$
    \item $x < Sy \leftrightarrow x < y \vee x = y$
    \item $x < y \vee x = y \vee y < x$
    \end{enumerate}
  \item Kenningin $\mathcal{P}\mathcal{A}$ fyrir nátt. tölur hefur sama mál og
    $\mathcal{N}$, er samsemdarkenning og allar frumsendur í $\mathcal{N}$ eru
    frumsendur í $\mathcal{P}\mathcal{A}$, en auk þess heufr $\mathcal{P}\mathcal{A}$
    frumsendugrip.

    \textbf{IND}. $\mathbf{A}_{\mathbf{x}}[0] \wedge \forall \mathbf{x} ( \mathbf{A} \rightarrow \mathbf{A}_{\mathbf{x}}[S\mathbf{x}]) \rightarrow \mathbf{A}$.

    Köllum $\mathcal{P}\mathcal{A}$ \emph{Peano-reikning}.
  \end{enumerate}
\end{skgr}

\begin{skgr}
  Látum $\mathbf{A}$ vera yrðingarsnið a málinu $\mathcal{L}(\mathcal{P})$
  og $\mathbf{x}_1, \dotsc, \mathbf{x}_n$ vera ólíkar yrðingabreytur
  þ.a. allar yrðingabreytur í $\mathbf{A}$ séu meðal þeirra.
  Látum $\mathbf{B}_1, \dotsc, \mathbf{B}_n$ vera yrðingar á máli
  $\mathcal{L}$ fyrstu stéttar. Táknum með
  \[ \mathbf{A}_{\mathbf{x}_1, \dotsc, \mathbf{x}_n}[\mathbf{B}_1, \dotsc, \mathbf{B}_n] \]
  eða
  \[\mathbf{I}^{\mathbf{B}_1, \dotsc, \mathbf{B}_n}_{\mathbf{x}_1, \dotsc, \mathbf{x}_n} \mathbf{A}\]

  stæðuna sem fæst með því að setja $\mathbf{B}_k$ inn fyrir $\mathbf{x}_k$ í $\mathbf{A}$ fyrir
  $k = 1, \dotsc, n$.
\end{skgr}

\begin{setn}
  $\mathbf{A}_{\mathbf{x}_1, \dotsc, \mathbf{x}_n}[\mathbf{B}_1, \dotsc, \mathbf{B}_n]$ er yrðing á málinu $\mathcal{L}$.

  \emph{Sísönnusetning}. Ef $\mathbf{A}$ er sísanna og $\mathbf{A}_1, \dotsc, \mathbf{A}_n$ yrðingar á $\mathcal{L}$, þá er
  $\mathbf{A}_{\mathbf{x}_1 \dotsc, \mathbf{x}_n}[\mathbf{A}_1, \dotsc, \mathbf{A}_n]$ setning
  í $\mathcal{L}$.

  \begin{proof}
    Látum $\mathbf{C}_1, \dotsc, \mathbf{C}_m$ vera sönnun á $\mathbf{A}$ í $\mathcal{P}$.
    Látum $\mathbf{y}_1, \dotsc, \mathbf{y}_r$ vera upptalningu allra yrðingabreyta sem
    koma fyrir í $\mathbf{C}_1, \dotsc, \mathbf{C}_m$ en eru \emph{ekki}
    meðal breytanna
    $\mathbf{x}_1, \dotsc, \mathbf{x}_n$. Látum

    Skrifum $\theta \mathbf{D}$ fyrir $\mathbf{I}^{\mathbf{A}_1, \dotsc, \mathbf{A}_n, \mathbf{B}_1, \dotsc, \mathbf{B}_r}_{\mathbf{x}_1, \dotsc, \mathbf{x}_n, \mathbf{y}_1, \dotsc, \mathbf{y}_r} D$
    þá er $\theta \mathbf{C}_1, \dotsc, \theta \mathbf{C}_m$ sönnun á $\theta \mathbf{A}$ í sérhverri kenningu
    fyrstu stéttar með málið $\mathcal{L}$ og $\theta \mathbf{A}$ er
    $\mathbf{A}_{\mathbf{x}_1, \dotsc, \mathbf{x}_n} [\mathbf{A}_1, \dotsc, \mathbf{A}_n]$
  \end{proof}
\end{setn}

Sumir nota ``sísönnusetning'' yfir eftirfarandi

\begin{setn}[Fylgisetning]
  Látum $\mathcal{T}$ vera kenningu fyrstu stéttar og $\mathbf{A}_1, \dotsc, \mathbf{A}_n, \mathbf{B}$ vera
yrðingar á máli $\mathcal{T}$. G.r.f. að
$\vdash_{\mathcal{T}} \mathbf{A}_1, \dotsc, \vdash_{\mathcal{T}} \mathbf{A}_n$
og að $\mathbf{A}_1 \wedge \dotsb \wedge \mathbf{A}_n \vdash_{\mathcal{T}} \mathbf{B}$
fáist með innsetningu í sísönnu. Þá er $\vdash_{\mathcal{T}}B$.

  \begin{proof}
   $p_1 \wedge \dotsb \wedge p_n \rightarrow q$
   er rökfræðilega jafngilt
   $p_1 \rightarrow p_2 \rightarrow \dotsb \rightarrow p_n \rightarrow q$.

   $\vdots$
  \end{proof}
\end{setn}


\begin{setn}[Fylgisetning]

  Látum $\mathcal{T}$ vera kenningu fyrstu stéttar og $\mathbf{A}, \mathbf{B}, \mathbf{C}$
  vera yrðingar á $\mathcal{T}$.

  $\vdots$
\end{setn}




$\vdots$



\begin{setn}[Fylgisetning]
  Ef $\vdash \mathbf{A} \rightarrow \mathbf{B}$, þá er
  $ \vdash \exists \mathbf{x} \mathbf{A} \rightarrow   \exists \mathbf{x} \mathbf{B}$
  og $\vdash \forall \mathbf{x}\mathbf{A} \rightarrow \forall \mathbf{x} \mathbf{B}$
\end{setn}


\begin{skgr}
  Látum $\mathbf{A}$ vera yrðingu. \emph{Lokun} yrðingarinnar
  $\mathbf{A}$ er yrðing $\forall \mathbf{x}_1, \forall \mathbf{x}_2, \dotsc,
  \forall \mathbf{x}_n \mathbf{A}$ þar sem $\mathbf{x}_1, \dotsc, \mathbf{x}_n$
er einhver upptalning á breytunum sem koma fyrir frjálsar í $\mathbf{A}$
\end{skgr}

\begin{ath}
  Yrðing er \emph{lokuð} þþaa engin breyta komi fyrir frjáls í henni.
\end{ath}

\begin{setn}[Fylgisetn]
  Látum $\mathbf{A}'$ vera lokun $\mathbf{A}$. Höfum $\vdash \mathbf{A}'$
  þþaa $\vdash \mathbf{A}$.
\end{setn}

\section{Afleiðslusetning fyrir umsagnarökfræði}

Látum $\mathbf{A}$ vera lokaða yrðingu og $\mathbf{B}$ vera
yrðingu á málinu $\mathcal{L(T)}$.
Ef $\mathbf{A} \vdash \mathbf{B}$, þá er $\vdash \mathbf{A} \rightarrow \mathbf{B}$.



\begin{setn}[Fylgisetning]
  Látum $\mathbf{A}, \mathbf{B}$ vera yrðingar og $\mathbf{x}_1, \dotsc, \mathbf{x}_n$
vera upptalningu á breytunum sem koma frjálsar fyrir í $\mathbf{A}$. Látum $\mathcal{T}'$
vera  kenninguna sem fæst með því að bæsta nýjum föstum $\mathbf{e}_1, dotsc, \mathbf{e}_n$
við $\mathcal{T}$. Ef
\[
\mathbf{A}_{\mathbf{x}_1, \dotsc, \mathbf{x}_n}[\mathbf{e}_1, \dotsc, \mathbf{e}_n] \vdash_{\mathcal{T}}
\mathbf{B}_{x_n}[\mathbf{e}_1, \dotsc, \mathbf{e}_n]
\]

þá
\[ \vdash_{\mathcal{T}} \mathbf{A} \rightarrow \mathbf{B} \]



\end{setn}


\begin{setn}[Jafngildisetning]
 Látum $ \mathbf{A}'$ vera yrðingu sem fæst með því að s etja yrðingar
 $\mathbf{B}_1', \dotsc, \mathbf{B}_n'$ sumstaðar í stað hlutyrðinga
 $\mathbf{B}_1, \dotsc, \mathbf{B}_n$ í $\mathbf{A}$ (sem hafa ekkert tákn
sameiginlegt tvær og tvær).
Ef $\vdash \mathbf{B}_k' \leftrightarrow \mathbf{B}_k$ fyrir
$k = 1, \dotsc, n$. Þá er $\vdash \mathbf{A}' \leftrightarrow \mathbf{A}$
\end{setn}


\section{3. útvíkkanir með skilgreiningum}

Höfðum kenningu $\mathcal{O}_1$ fyrir röðuð mengi;
það var samsemdarsetning með tveimur tvístæðum umsagnartáknum,
'$=$' og '$\leq$'. Nú viljum við skilgreina nýtt tákni '$<$' með

(*) ' $ x < y \rightarrow (x \leq y \wedge x \neq y)$'.

í stað þess að líta á þetta sem skammstöfun er hentugra að bæta við
'$<$' sem nýju umsagnartákni og (*) sem nýrri frumsendu.

Höfðum líka $\mathcal{O}_2$ með táknum
'$=$' og '$<$'. Nú viljum við skilgreina nýtt tákni '$\leq$' með
og nýrri frumsendum.

(*) ' $ x \leq y \rightarrow (x < y \vee x = y)$'.

Þannig fást tvær nýjar kenningar með sömu umsagnartáknum og sömu setningum!

Höfum kenningu A fyrir víxlgrúpur með táknum '$=$' og tvístæðu fallatákni '$+$',
frumsendum
'$(x+y) + z = x + (y + z)$', '$x + y = y + x$',
'$ \exists y \forall x (x +y = x) $', '$\forall y \forall z \exists x (x + y = z)$'

þá má sanna

\[ \exists y ( \forall x (x+y = x) \wedge \forall z (\forall x (x+z = x)) \rightarrow z = y) \]

Þá viljum við bæta við núllstæðu fallakákni (fasta) 0 ásamt nýrri frumsendu;
annaðvhort

'$y = 0 \leftrightarrow \forall x ( x+y = x)$'

eða bara
'$\forall x (x + 0 = x)$'

Sömuleiðis má sanna að fyrir hvert x er til \emph{nákv. eitt} y þ.a.
'$x+y = 0$'; viljum bæta við 1-stæðu fallatákni '$-$' og frumsendu

$'y = -x \leftrightarrow x + y = 0'$

eða

$'x + (-x) = 0'$


\begin{skgr}
\begin{enumerate}
	\item Látum $\mathcal{L}$ vera mál fyrstu stéttar. Mál fyrstu stéttar $\mathcal{L}'$ er \emph{útvíkkun} málsins $\mathcal{L}$
		ef sérhvert eiginlegt tákn í $\mathcal{L}$ er líka samskonar tákn (þ.e. ef \textbf{p} er n-stætt umsagnar tákn í $\mathcal{L}$,
		þá á það að vera n-stætt umsagnar tákn í $\mathcal{L}'$, eins með falla tákn.)
	\item Kenning \(\mathcal{T}'\) fyrstu stéttar er \emph{útvíkkun} annarar kenningar \(\mathcal{T}\) fyrstu stéttar ef $\mathcal{L(T')}$
		útvíkkun $\mathcal{L(T)}$.
	\item Segjum að slík útvíkkun $\mathcal{T'} á \mathcal{T}$ sé \emph{íhaldssöm} ef sérhver setning í $\mathcal{T'}$
		er setning í $\mathcal{T}$
\end{enumerate}
\end{skgr}

Látum nú $\mathcal{T}$ vera kenningu fyrstu stéttar, $\mathbf{D}$ vera yrðingu á máli $\mathcal{T}$ þ.a. engar breytur nema ólíkubreyturnar
$\mathbf{x}_1, \dotsc, \mathbf{x}_n$ komi fyrir frjálsar í $\mathbf{D}$. Bætum við $\mathcal{T}$ nýju umsagnartákni $\mathbf{p}$.
og nýrri frumsendu
\[ p \mathbf{x}_1, \dotsc, \mathbf{x}_n \leftrightarrow \mathbf{D} .\]
Fáum þannig útvíkkun $\mathcal{T'}$ af $\mathcal{T}$.

\begin{setn}
Fyrir hverja yrðingu $\mathbf{A}$ í $\mathcal{T'}$ má búa til yrðingu $\mathbf{A}^{*}$ á málinu $\mathcal{L(T)}$
þ.a. $\vdash_{\mathcal{T'}} A$ þþaa $\vdash_{\mathcal{T}} \mathbf{A}^{*}$. Raunar:

\begin{enumerate}[(i)]
\item $\vdash_{\mathcal{T'}} A \leftrightarrow \mathbf{A}^{*}$
\item $\mathcal{T'}$ er íhaldssöm útvíkkun á $\mathcal{T}$.
\end{enumerate}

\end{setn}


Svipað fyrir fallatákn. Látum $\mathbf{D}$ vera yrðingu í
$\mathcal{L(T)}$ og $\mathbf{x}_1, \dotsc, \mathbf{x}_n$
vera ólíkar breytur þ.a. engar aðrar koma fyrir frjálsar í 
$\mathbf{D}$. og $\mathbf{z}$ vera ólíka öllum í $\mathbf{D}$
þ.a.
\[ \vdash_{\mathcal{T}} \exists y ( \mathbf{D} \wedge \forall z ( \mathbf{D}_{y} [z] \rightarrow \mathbf{z} = \mathbf{y})) \]

þá má bæta við n-stæðu fallatákni $\mathbf{f}$ og bæta við frumsendu
\[ \mathbf{y} = \mathbf{f}_{\mathbf{x}_1 \dotsb \mathbf{x}_n} \leftrightarrow \mathbf{D} \]
(ef $\mathcal{T}$ er samsembnarsetning eða (heldur reynir)
\[ \mathbf{D}_{\mathbf{y}} [\mathbf{f} \mathbf{x}_1 \dotsb \mathbf{x}_n \]

annars )


\section{Kafli III. Líkön}

\subsection{Mynztur og líkön}

\begin{skgr}
  \begin{enumerate}[(1)]
  \item  \emph{Mynztur} fyrir mál \(\mathcal{L} \) fyrstu stéttar er gefið með:
    \begin{enumerate}[(i)]
    \item  Mengi \(|M| \) (sem er ekki tómt)
    \item  Fyrir sérhvert n-stætt fallatákn \(\mathbf{f}\) í \(\mathcal{L}\)
       vörpun: 
       \[ \mathbf{f}_{M}: |M|^n \rightarrow |M| \]
     \item Fyrir sérhvert n-stætt umsagnartákn \(\mathbf{p} \) í \(\mathcal{L}\)
       n-stæð venzl $\mathbf{p}_{M}$, þ.e. hlutmengi í $|M|^{n}$.
    \end{enumerate}
    Ef $\mathcal{L}$ hefur samasemmerki $'='$, þá segjum við að $M$ sé 
    \emph{samsemdarmynztur}. Ef $=_{M}$ er venjuleg samasemdar venzl á $|M|$.
  \item Látum $M$ vera mynztur fyrir $\mathcal{L}$. og $V_{\mathcal{L}}$ vera mengi breytanna í $\mathcal{L}$.

    Vörpun $s: V_{\mathcal{L}} \rightarrow |M|$ kallast \emph{úthlutun}.
  \item Látum s vera slíka úthlutun. Ef $\mathbf{x}$ er breyta og $a \in |M|$, þá táknar
    $s_{\mathbf{x}}[a]$ úthlutunina
    \[s_{\mathbf{x}}[a](\mathbf{y}) := \bcondef s(\mathbf{y}) & \text{ ef $\mathbf{y}$ er ekki $\mathbf{x}$ } \\ a & \text{ ef $\mathbf{y}$ er $\mathbf{x}$} \econdef \]
  \item Fyrir sérhvert heiti $\mathbf{a}$ í $\mathcal{L}$ og sérhverja úthlutun s skilgreinum við stak
    $\mathbf{a}_{M}(s)$ í |M| þannig:
    \begin{enumerate}[(1)]
    \item  Ef $\mathbf{a}$ er breyta $\mathbf{x}$, þá er 
      \[ \mathbf{a}_{M}(s) = s(\mathbf{x}) \]
    \item Ef $\mathbf{a}$ er $\mathbf{f} \mathbf{a}_1 \dotsb \mathbf{a}_n$, þar sem 
      $\mathbf{f}$ er n-stætt fallatákn og $\mathbf{a}_1, \dotsc, \mathbf{a}_n$ eru heiti,
      þá er
      \[ \mathbf{a}_{M} (s) = \mathbf{f}_{M}(\mathbf{a}_{1,M}(s), \dotsc, \mathbf{a}_{n,M}(s)) \]
    \end{enumerate}
    sér í lagi er $\mathbf{a}_{M}(s) = \mathbf{c}_M$ ef a er fastinn $\mathbf{c}$.
  \end{enumerate}
\end{skgr}

\begin{skgr}
  Látum M vera mynztur og s vera úthlutun. Fyrir allar yrðingar $\mathbf{A}$ í $\mathcal{L}$ skilgreinum
við hvað það þýðir að ``s fullnægi $\mathbf{A}$ í M'', skrifað

\[ \models_{M} \mathbf{A} (s) \].

með þrepun
\begin{enumerate}[(1)]
\item  Ef $\mathbf{A}$ er grunnyrðing $\mathbf{p} \mathbf{a}_1 \dotsb \mathbf{a}_n$,
  þá er $\models_{M} \mathbf{A} (s)$ þþaa $(\mathbf{a}_{1,M}(s), \dotsc, \mathbf{a}_{n,M}(s)) \in \mathbf{p}_{M}$.
\item Ef $\mathbf{A}$ er $\lnot \mathbf{B}$, þá er $\models_{M} \mathbf{A}(s)$ þþaa ekki gildi
  $\models_{M} \mathbf{B}(s)$.
\item Ef $\mathbf{A}$ er $\mathbf{B} \vee \mathbf{C}$, þá gildir 
  $\models_{M} \mathbf{A}(s)$ þþaa $\models_{M} \mathbf{B}(s)$ eða $\models_{M} \mathbf{C}(s)$.
\item Ef $\mathbf{A}$ er $\forall x \mathbf{B}$, þá $\models_{M} \mathbf{A}(s)$ þþaa
  $\mathbf{B}_{\mathbf{x}} (s_{\mathbf{x}}[a])$ gildi fyrir öll a úr M
\end{enumerate}

\end{skgr}

\begin{skgr}
  $\models_{M} \mathbf{A}$ þþaa $\models_{M} \mathbf{A}(s)$ f. öll s.
\end{skgr}

\begin{ath}
 $\mathbf{C}_{M}: |M|^{0} \rightarrow |M|$, þar sem $|M|^{0} = \set{0}$
\end{ath}




Við segjum að yrðing $\mathbf{A}$ sé \emph{sönn} í mynztri M ef
\( \models_{M} \mathbf{A} (s) \) fyrir allar úthlutanir s.

\begin{ath}
  Látum \(\mathbf{A}'\) vera lokun yrðingar \(\mathbf{A}\) á málinu
\(\mathcal{L} \). Við höfum
\[ \models_{M} \mathbf{A} \text{ þþaa } \models_{M} \mathbf{A}' \]
\end{ath}

\begin{skgr}
  Látum \( \mathcal{L} \) vera mál fyrstu stéttar. við segjum að yrðing $\mathbf{A}$
sé \emph{röksönn} ef  $\models_{M}\mathbf{A}$ fyrir öll mynstur M.
...
Ef \(\models_{M}\mathbf{A}\) fyrir sérhvert mynztur M fyrir \(\mathcal{L}\)
þ.a. $\models_{M} \mathbf{B}$ fyrir öll $\mathbf{B}$ úr $\mathcal{H}$. 
\end{skgr}


\begin{setn}
  Látum $\mathcal{T}$ vera kenningu fyrstu stéttar á máli \(\mathcal{L}\) sem
  hefur engar \emph{eiginlegar} frumsendur. Þá gildir.
  \begin{enumerate}[(1)]
  \item  Sérhver frumsenda í $\mathcal{T}$ er röksönn
  \item Ef $\mathcal{H} \vdash \mathbf{A}$, þá $\mathcal{H} \models \mathbf{A}$.
  \end{enumerate}

\end{setn}

\begin{proof}
  Þetta er einföld afleiða af skilgreiningu.
  Sönnum t.d. að frumsenda $\mathbf{A} \vee \mathbf{A} \rightarrow \mathbf{A}$.
  sé röksönn. Annars væri til mynztur M fyrir $\mathcal{L}$ og úthlutun s þannig að
  \emph{ekki} gildi 

  \[ \models_{M} (\mathbf{A} \vee \mathbf{A} \rightarrow \mathbf{A}) (s) \]
  þ.e. ekki gildi
  \[ \models_{M} ( \lnot ( \mathbf{A} \vee \mathbf{A} ) \vee \mathbf{A}) (s) \]
  þá væri ekki
  \[\models_{M} ( \lnot (\mathbf{A} \vee \mathbf{A}))(s)\] og ekki $\models_{M} \mathbf{A}(s)$
  og það þýðir að  \( \models_{M} (\mathbf{A} \vee \mathbf{A}) (s) \) og \emph{ekki} $\models_{M} \mathbf{A} (s)$
  En $\models_{M}( \mathbf{A} \vee \mathbf{A}) (s)$ þýðir að $\models_{M} \mathbf{A}(s)$. Höfum þá bæði
  $\models_{M} \mathbf{A}(s)$ og ekki $\models_{M} \mathbf{A}(s)$, sem er mótsögn.
\end{proof}


\begin{setn}[Fylgisetning]
 Látum $\mathcal{T}$ vera kenningu fyrstu stéttar á máli $\mathcal{L}$. Ef M er
 mynztur fyrir $\mathcal{L}$ þ.a. sérhver eiginleg frumsenda í $\mathcal{T}$ sé sönn
 í M, þá er sérhver setning í $\mathcal{T}$ sönn í M, þá er sérhver setning í 
 $ \mathcal{T} $ sönn í M.
\end{setn}

\begin{skgr} Látum $\mathcal{T}$ vera kenningu fyrstu stéttar á máli $\mathcal{L}$ þ.a. 
  sérvher eiinleg frumsenda í $\mathcal{T}$ sé sönn í mynztrinu. Ef $\mathcal{T}$ er samsemdarkenning,
  þá er \emph{samsemdarlíkan} fyrir $\mathcal{T}$ líkan fyrir $\mathcal{T}$ sem er 
  samsemdarmynztur.
\end{skgr}

\begin{ath}
  Ef $\mathcal{T}$ er samsemdarkenning og M er líkan fyrir $\mathcal{T}$, þá eru
  venzlin$=_{M}$ jafngildisvenzl. Látum $M'$ vera mengi jafngildisflokkana. Þá verður $M'$.
  sjálfkrafa að samsemdarlíkani.
\end{ath}

Fyrir n stætt fallatákn $\mathbf{f}$ skilgreinum við
\[\mathbf{f}_{M'}([a_1], \dotsc, [a_n]) := \mathbf{f}_{M} (a_1, \dotsc, a_n) \]
Þar sem $[a_k]$ er jafngildisflokkur $a_k$;
Þetta er ekki vel skilgreint, því að
 \[\vdash_{\mathcal{T}} \mathbf{a}_1 = \mathbf{b}_1 \wedge \mathbf{a}_2 = \mathbf{b}_2
 \wedge \dotsb \wedge \mathbf{a}_n 
= \mathbf{b}_n \rightarrow \mathbf{f}\: \mathbf{a}_1 \dotsb \mathbf{a}_n = \mathbf{f}\: \mathbf{b}_1 \dotsb \mathbf{b}_n \]

svo að fyrir $a_1, \dotsc, a_n, b_1, \dotsc, b_n$ þ.a. 
$a_k =_{M} b_k$ fyrir $k = 1, \dotsc, n$ er $\mathbf{f}_{M} (a_1, \dotsc, a_n) =_{M} \mathbf{f}_{M} (b_1, \dotsc, b_n)$A
Eins með umsagnartáknin.


\begin{daemi}
  Samsemdarlíkan fyrir kenninguna $\mathcal{O}_1$ ( með tákn $'='$ og $'\leq'$
  er raðað mengi. (\emph{Líkan} fyrir $\mathcal{O}_1$ er ``\emph{forraðað}'' mengi.)

  Samsemdarlíkan fyrir kenningarnar $A_1, A_2$
  fyrir víxlgrúpur er víxlgrúpa; samsemdarlíkön fyrir $G_1, G_2$ eru 
  grúpur.

  Hvenær hefur kenning fyrstu stéttar líkan?
  Nauðsynlegt skilyrði er að hún sé \emph{samkvæm}; Ef hún er samkvæm, þá er 
  hún samkvæm m.t.t. kennigarinnar ($\mathbf{A} \rightarrow (\lnot \mathbf{A} \rightarrow \mathbf{B})$ er setningargrip)
  en yrðingarnar $\mathbf{A}$ og $\lnot \mathbf{A}$ geta ekki báðar verið sannar í sama mynztri.
\end{daemi}

Stenum á að sanna:

\begin{setn}[Meginsetning]
  Kenning fyrstu stéttar hefur líkan þþaa hún sé samkvæm
\end{setn}

\begin{setn}[Fullkomleikasetning Gödels]
  Látum $\mathcal{T}$ vera kenningu fyrstu stéttar á máli $\mathcal{L}$.
  Yrðing á málinu $\mathcal{L}$ er setning í $\mathcal{T}$ þþaa
  hún sé sönn í sérhverju líkani fyrir $\mathcal{T}$.
\end{setn}

Til að sjá að FG sé afl. af meginsetningu:

\begin{setn}[Hjálparsetning]
 Lokuð yrðing $\mathbf{A}$ er setning í $\mathcal{T}$ þþaa
 kenningin $\mathcal{T}[\lnot \mathbf{A}]$ sé ósamkvæm.
\end{setn}




\begin{proof}
 Ef $\mathbf{A}$  er setning í $\mathcal{T}$, þá er $\mathbf{A}$ og $\lnot \mathbf{A}$
 setningar í $\mathcal{T}[\lnot \mathbf{A}]$, svo að
 $\mathcal{T}[\lnot \mathbf{A}]$ er ósamkvæm. Ef hinsvegar
 $\mathcal{T}[\lnot \mathbf{A}]$ er  ósamkvæm, þá er $\mathbf{A}$
 setning í $\mathcal{T}[\lnot \mathbf{A}]$, svo að
 $\lnot \mathbf{A} \vdash_{\mathcal{T}} \mathbf{A}$.

 Þar sem $\mathbf{A}$ er lokuð, gefur afleiðsluseting að

 \[\vdash_{\mathcal{T}} \lnot \mathbf{A} \rightarrow \mathbf{A},\]
 en \[ \vdash_{\mathcal{T}} ( \lnot \mathbf{A} \rightarrow \mathbf{A}) \rightarrow \mathbf{A}\]

 skv. sísönnusetningu; svo að $\vdash_{\mathcal{T}} \mathbf{A}$ skv. \textbf{MP}.

\end{proof}

\begin{proof}[Sönnun á Fullkomleikasetningu Gödels]
  G.r.f. að $\mathbf{A}$ sé ekki setning í $\mathcal{T}$ og látum $\mathbf{A}'$
  vera lokun $\mathbf{A}$. Þá er $\mathbf{A}'$ ekki heldur seting í
  $\mathcal{T}$. Þá er $\mathcal{T}[\lnot \mathbf{A}']$ samkvæm skv. HS og hefur
  þá líkan M skv. meginsetningu.

  En þá er M líkan fyrir $\mathcal{T}$ þ.a. $\lnot \mathbf{A}'$ sé sönn í M
  og þá er $\mathbf{A}'$ ósönn í M.
  
\end{proof}




\vdots


\begin{ath}
  Henkin: Fyrir sérhverja lokaða yrðingu \( \exists \mathbf{x}\mathbf{A}\), þá
  er til fasti \(\mathbf{c}\) í $\mathcal{T}$ þ.a. $\vdash  \exists \mathbf{x}\mathbf{A} \rightarrow  \mathbf{A}_{\mathbf{x}}[\mathbf{c}]$

  Meginsetning:
  Kenning hefur líkan þþaa hún sé samkvæm.
  Þ.e. hún hefur ekki bæði $\mathbf{A}$ og $\lnot \mathbf{A}$
\end{ath}


\begin{setn}
  Látum $\mathcal{T}$ vera fullkomna henkin-kenningu
  með stafróf $\mathcal{S}$. Þá hefur $\mathcal{T}$ líkan M
  þ.a. $|M| = \#\mathcal{S}$;
 % ef hún er samsemdarkenning þá hefur hún samsemdalíkan $M$ þ.a. $|M| \leq \#\mathcal{S}$.

  \begin{proof}
    Aðeins er eftir að sanna fyrri fullyrðinguna.
    Látum $|M|$ vera mengi allra ``lokaðra heita'' (þ.e. heiti án breyta) á málinu
    $\mathcal{L}$. Fyrir sérhvert n-stætt fallatákn $\mathbf{f}$
    látum við $\mathbf{f}_M$ vera fallið $|M|^{\alpha} \rightarrow |M|$
    þ.a.

    \[\mathbf{f}_M ( \mathbf{a}_1, \dotsc,\mathbf{a}_n ) := \mathbf{f} \mathbf{a}, \dotsc, \mathbf{a}_n ,\]
    
    og fyrir n-stætt umsagnartákn $\mathbf{p}$ látum við $\mathbf{p}_M$ veera mengi allra
    n-unda $(\mathbf{a}_1, \dotsc, \\mathbf{a}_n)$ þ.a. 
    \(\mathbf{a}_1, \dotsc, \mathbf{a}_n \)
    þ.a. $\vdash_{\mathcal{T}} \mathbf{p} \mathbf{a}_1, \dotsc, \mathbf{a}_n$.
    Okkur nægir að sýna að fyrir lokaða yrðingu $\mathbf{A}$ gildir
    \[ \models_M \mathbf{A} \text{ þþaa } \vdash_{\mathcal{T}} \mathbf{A}; \]
    því að fyrir frumsendur $\mathbf{A}$ í $\mathcal{T}$. 
    Gildir $\vdash_{\mathcal{T}} \mathbf{A}'$; þar sem $\mathbf{A}'$ er lokun
    $\mathbf{A}$; en af því leiðir $\models_M \mathbf{A}'$ þá er líka $\models_M \mathbf{A}$.
    Svo að M er líkan fyrir $\mathcal{T}$. Þrepun yfir lengd yrðinga:
    \begin{enumerate}[(1)]
    \item Ef $\mathbf{A}$ er grunnyrðing, þá er þetta bein afleiðing af skilgr.
    \item $\mathbf{A}$ er $\lnot \mathbf{B}$. Ef $\models_M \mathbf{A}$, þá er ekki
      $\models_M \mathbf{B}$ og því ekki $\vdash_{\mathcal{T}} \mathbf{B}$ skv. þf.
      þar sem $\mathcal{T}$ er fullkomin og $\mathbf{B}$ lokað er
      $\vdash_{\mathcal{T}} \lnot \mathbf{B}$, þ.e. $\vdash_{\mathbf{A}}$. En ef
      ekki $\models_{M} \mathbf{A}$, þá er $\models_M \mathbf{B}$ og því
      $\vdash_{\mathcal{T}} \mathbf{B}$ skv. þf. En þar sem $\mathcal{T}$ er 
      samkvæm er ekki $\vdash_{\mathcal{T}} \lnot \mathbf{B}$ og því ekki
      $\vdash_{\mathcal{T}} A$.
    \item $\mathbf{A}$ er $\mathbf{B} \vee \mathbf{C}$: Ef $\models_M \mathbf{A}$,
      þá er annaðhvort $\models_M \mathbf{B}$ eða $\models_M \mathbf{C}$,
      svo að $\vdash_{\mathcal{T}} \mathbf{B}$ eða $\vdash_{\mathcal{T}} \mathbf{C}$ skv. þf.
      í báðum tilvikum er $\vdash_{\mathcal{T}} \mathbf{A}$ skv. sís.
      Ef ekki $\models_M \mathbf{A}$, þá er hvorki $\models_M \mathbf{B}$
      né $\models_M \mathbf{C}$ og því hvorki $\vdash_{\mathcal{T}}$ né 
      $\vdash_{\mathcal{T}} \mathbf{C}$.
      En $\mathcal{T}$ er fullkomin svo að $\vdash_{\mathcal{T}} \lnot \mathbf{B}$ 
      og $\vdash_{\mathcal{T}} \lnot \mathbf{C}$ og þá ekki $\vdash_{\mathcal{T}} \mathbf{A}$ skv. sís.

    \item $\mathbf{A}$ er $\forall \mathbf{x} \mathbf{B}$. Ef
      $\mathbf{B}$ er lokað, þá fæst $\models_M \mathbf{A}$ þþaa $\models_M \mathbf{B}$ sem skv.
      þf. gildi þþaa $\vdash_{\mathcal{T}} \mathbf{B}$ og það er jafngilt $\vdash_{\mathcal{T}} \mathbf{A}$A

      G.r.f. að $\mathbf{B}$ sé ekki lokað, þá er $\mathbf{x}$ eina breytan sem er frjáls í $\mathbf{B}$.A

      \begin{enumerate}[(a)]
      \item  G.r.f. að $\models_M \mathbf{A}$ en  ekki $\vdash_{\mathcal{T}} \mathbf{A}$.
        Skv. fullkomleika er $\vdash_{\mathcal{T}} \lnot \mathbf{A}$ og því
        $\vdash_{\mathcal{T}} \exists \mathbf{x} \lnot \mathbf{B}$. Þar sem
        $\mathcal{T}$ er hengin er til fasti $\mathbf{c}$ þa.a. 
        $\vdash_{\mathcal{T}} \lnot \mathbf{B}_{\mathbf{x}} [\mathbf{c}]$.
        En vegna $\models_M \forall \mathbf{x} \mathbf{B}$ svo að
        $\models_M \mathbf{B}_{\mathbf{x}} [\mathbf{c}]$ fyrir öll
        $\mathbf{c}$ í M.
        Skv. þf er $\vdash_{\mathcal{T}} \mathbf{B}_{\mathbf{x}} [\mathbf{c}]$. 
        Þetta er mótsögn, því að $\mathcal{T}$ er samkvæm.
      \item  G.rf. að $\vdash_{\mathcal{T}} \mathbf{A}$ en ekki $\models_M \mathbf{A}$.
        Vegna ekki $\models_M \forall \mathbf{x} \mathbf{B}$ er til úthlutun s
        þ.a. ekki $\models_M \forall \mathbf{x} \mathbf{B} (s)$ og því til $\mathbf{c}$ úr
        $|M|$ þ.a. ekki $\models_M \mathbf{B}_{\mathbf{x}} ( s_{\mathbf{x}}[\mathbf{c}])$;
        en það jafngildir $\models_M \mathbf{B}_{\mathbf{x}}[\mathbf{c}]$. En 
        $\vdash_{\mathcal{T}} \forall \mathbf{x} \mathbf{B}$ og því
        $\vdash_{\mathcal{T}} \mathbf{B}_{\mathbf{x}} [\mathbf{c}]$ og því
        $\models_M \mathbf{B}_{\mathbf{x}} [\mathbf{c}]$ skv. þf. 

        Þetta er mótsögn. Því sést:
        Ef $\vdash_{\mathcal{T}} \mathbf{A}$, þá $\vdash_{\mathcal{T}} \mathbf{A}$,
        þá $\models_M \mathbf{A}$.
      \end{enumerate}
    \end{enumerate}
    \end{proof}
\end{setn}

\begin{setn}[Löwenheim-Skolem-setningin]
Ef kenning fyrstu stéttr hefur teljanlegt stafróf,
þá hefur hún teljanlegt líkan.
\end{setn}

\begin{skgr}
  \begin{enumerate}[(1)]
  \item  Segjum að kenning $\mathcal{T}$ sé \emph{hluti}
    af kenningu $\mathcal{T}'$ ef $\mathcal{T}$ og $\mathcal{T}'$ hafa
    sama mál og sérhver eiginleg frumsenda í $\mathcal{T}$ er frumsenda 
    í $\mathcal{T}'$
  \item Segjum að kenning $\mathcal{T}$ sé \emph{endanleg frumsenduð}
    ef fjöldi eiginlegra frumsenda í $\mathcal{T}$ er endanleg.
  \end{enumerate}
\end{skgr}

\begin{ath}
 Hér teljum við \textbf{Eq 1} og \textbf{Eq 2} til rökfrumsenda fyrir
 samsemdarkenningar.
\end{ath}

\begin{setn}[Þjöppunarsetning]
  Yrðin í kenningu $\mathcal{T}$ er sönn í $\mathcal{T}$ ef (þ.e. sönn
  í sérhverju líkani fyrir $\mathcal{T}$ ) þþaa
  hún sé sönn í endanlega frumsenduðum hlut af $\mathcal{T}$.
  Sama gildir fyrir samsemdarkenningu
\end{setn}

\begin{proof}
  Höfum $\mathcal{T} \models \mathbf{A}$ þþaa $\vdash_{\mathcal{T}} \mathbf{A}$
  skv. fullkomleika setningu. En í sönnun $\mathbf{A}$ eru aðeins endalega
  margar eiginlegar frumsendur notaðar, svo að $\vdash_{\mathcal{T}} \mathbf{A}$
  jafngildir því að til sé endanlega frumsendaður hluti $\mathcal{T}_{1}$
  af $\mathcal{T}$ þ.a. $\vdash_{\mathcal{T}_1} \mathbf{A}$, en það jafngildir
  $\mathcal{T}_1 \models \mathbf{A}$
\end{proof}
\begin{setn}[Fylgisetn]
  Kenning $\mathcal{T}$ hefur líkan þþaa sérhver endanlega
  frumsendaður hluti af $\mathcal{T}$ hafi líkan.
  
  Samsemdarkenning hefur samsemdarlíkan þþaa sérhver endanlega frumsendaður
  hluti af $\mathcal{T}$ hafi samsemdarlíkan
\end{setn}

\begin{proof}
  Notum þjöppunarsetningu á yrðingu $\mathbf{A}$ af gerðinni
  $\mathbf{B} \wedge \lnot \mathbf{B}$. Kenning hefu líkan þþaa
  hún sé samkvæm þþaa ekki gildir $\vdash_{\mathcal{T}} \mathbf{B} \wedge \lnot \mathbf{B}$
  sérhver endanlega frumsendaður hluti af $\mathcal{T}$ hafi líkan.
\end{proof}

\begin{daemi}
  Notum þetta á kenningu $\mathcal{F}$ fyrir svið (stundum kallaðir kroppar);
  hún hefur svo fasta 0 og 1 og tvö tvístæð fallatákn $+$ og $\cdot$, er samsemdar
  kenning og hefur eiginlegar frumsendur.
\end{daemi}

\end{document}
